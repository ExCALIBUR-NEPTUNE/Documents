\nep \  will require the numerical instantiation of systems of equations to deliver
a solution that efficiently and accurately captures
key aspects of the collective physical processes expected to 
operate in the tokamak edge. The allowable systems will treat a range of 
interacting charged and neutral particle species as phase fluids, that is to 
say that functional dependence in velocity space is allowed as well as the more 
usual dependence on space and time. However, only at most two velocity space 
variables should be employed, typically the third will have been removed by 
averaging over the gyro-orbit. Averaging over velocity space 
dependencies to provide fluid-type models only dependent on spatio-temporal variables will be needed.
Plasma species will not only feel the electromagnetic 
field, but may modify its behaviour, which effect must be accounted for. (It 
should be borne in mind that other task packages will address treatment of 
charged and neutral species using discrete particle models and hydrodynamical approaches
involving classical Newtonian fluids, cf.\ Navier-Stokes Equations).

The collective physical processes (regimes) important 
for tokamak edge physics should be determined based on the length scales and timescales observed 
experimentally or derivable from estimates of the number density, temperature 
and magnetic field in the edge, as discussed above for attached L-mode plasmas, and separately
determined in other configurations. It is expected that 
relevant processes for each species separately will be selected on the basis of 
a small number of key parameters, of which obviously the first is number 
of particles of each species in the Debye sphere that should be large to ensure that a plasma 
treatment is appropriate. The second will be magnetisation parameter~$\delta_s=\rho_{ts}/L$ of a 
representative species particle, others will include the normalised collision 
frequency~$\nu^{*}=L/\lambda_{smfp}$, the drift-ratio, the
width of the SOL normalised to~$L$ and the plasma~$\beta$.
Different models will be appropriate according 
to the relative sizes of these latter key parameters, and the magnetic 
geometry, that is to say whether field lines are (i) straight (slab 
geometry), (ii) twisted, (iii) toroidal, (iv) twisted and toroidal. Note the 
term ``model" includes not only say special collision operators appropriate 
to interaction between different species including neutral species, but also 
the boundary conditions appropriate to the particular physical case, or for the 
latter it may be acceptable to describe how the representation changes to 
particle and/or hydrodynamic fluid on a bounding surface or over an overlap 
volume. In all cases, the limitations of each model need to be clearly stated, including a 
discussion of the ease of its numerical implementation at the Exascale.

The identification of key edge physical processes and regimes will prioritise 
the theoretical developments needed to produce models capable of treating them 
efficiently at the Exascale. Ideally it should be possible to transition smoothly 
from one model to another as key parameters vary, both as time varies and as a 
function of position. It is expected that the widely differing timescales for 
electron and ion species dynamics will present a particular challenge, and 
interaction is expected with other tasks for designing software for multiscale 
applications and researching surrogate models. Theorists are also expected to 
collaborate with numerical experts not only in respect of ease of implementation, 
but also to produce algorithms for estimating likely errors, both within a 
single species model and in its interactions with other species models 
including discrete particle and hydrodynamical fluid, in order to guide refinements 
to the equations of the system describing the given regime.
