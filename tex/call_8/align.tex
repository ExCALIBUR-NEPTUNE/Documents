There is a need to capture accurately the physics where the plasma interacts 
with the surfaces of the plasma-facing components~(PFCs) of the tokamak,
including particularly the divertor region 
and limiters, and ultimately the other parts of the first wall of the machine, 
such as duct entries and antenna shielding. Such a model is necessary to form 
the building block of simulations of an actionable, multiphysics,
Exascale-ready plasma boundary code that takes into account the plasma species, 
including impurities, and neutrals by coupling closely to models produced in FM-WP3.
For accuracy, it will be necessary to represent the outer regions of hot confined plasma
although main focus will be on the SOL, where indicative constraints on outputs are
\begin{enumerate}
\item particle and SOL plasma radiated power deposition on the PFCs
accurate to within 10\% as peak value and qualitative agreement on pattern.
\item radiation spectra for selected species identifiable in comparison with 
observation.
\item turbulent statistics giving mean transport suitable for input to transport
models, accurate to within 10\%, with at least qualitative agreement on power spectral
distribution.
\end{enumerate}

Properties of the SOL plasma, though they do vary, do not seem to change by more than
order of magnitude from discharge to discharge and between tokamaks. 
Estimates for L-mode in a medium-sized tokamak may be taken from~\cite{Mi13Expe}, 
which gives experimentally determined
edge values for ion temperature~$T_i\approx20$\,eV, electron temperature~$T_e\approx10$\,eV,
and density~$n_e\approx 3\times10^{18}$\,m$^{-3}$, with a SOL thickness
of several centimetres. Speeds of $10^3$\,m/s normal to the field
are seen corresponding to filament motion, whereas power balance arguments
suggest flows of $U \approx 10^5$\,m$s^{-1}$ along the field.
%, corresponding to $\lambda_q$ in the range $1-3$\,cm.

The above data imply that the electron Debye length $\lambda_D \approx 10^{-5}$\,m 
so that a plasma approximation is appropriate, and the collision
frequency may be estimated as $\nu_e \approx 3$\,MHz, ie. \
a fluid treatment
could be conceived of on timescales~$\tau_s >> \tau_e \approx  3 \times 10^{-7}$\,s.
However, the mean free path for electrons
$\lambda_{emfp} \approx 1$\,m (parallel to field) compared to a
connection length which may be estimated as~$10$\,m, and
for a magnetic field $B_T \approx 1$\,T, the Larmor radius
%of electrons~$\rho_e \approx 5 \times 10^{-5}$\,m, and
of Deuterium ions~$\rho_i  \approx 3$\,mm.
Further, the bulk plasma collision frequency for ions with SOL neutrals, assumed to
have a density $n_0=n$, is $\tau_{i0}= 2 \times 10^{-5}$\,s, so that the number
of collisions experienced by a typical SOL ion before it hits a PFC is small.
This lack of collisions with neutrals in this case, and the closeness of the plasma
scalelengths to the SOL dimensions suggests a fluid treatment may not always be
accurate and motivates more detailed treatment of the dynamics. It is worth noting
that $n_0$ may vary by orders of magnitudes when there is detachment,
gas puff, pellet injection or divertor pumping.

