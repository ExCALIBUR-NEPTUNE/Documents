The work of ref~\cite{Hu18Glob} is important for \nep\ in that simultaneous use is made
of both a 2-D and a 3-D fluid model, referred to as MFMC~(Multi-Fidelity Monte-Carlo, \Sec{mfmc}).
Samples from different planes are used to produce a low-fidelity 2-D model with stochastic parameters
as described in ref~\cite[\S\,4.1]{Hu18Glob}, using Bayesian Inference as described in \Sec{Bayes}.
There is an important physical underpinning that the flow modelled by Huan et al~\cite{Hu18Glob}
is primarily in a single direction, variation in which is neglected to produce
the 2-D model. This is encouraging for \nep, as similar arguments relating to
variation along the magnetic field are used to justify use of 2-D edge codes for
the tokamak scrape-off layer~(SOL).

Indeed, for SOL application, it is conceivable that multi-fidelity work involving 1-D
fluid models may be efficient, when it is remembered that the so-called `Onion-skin' models~\cite{St97Code},
have found a good deal of utility in SOL physics. The `onion' is the plasma
imagined to consist of separate skins, ie.\ layers delineated by equilibrium flux surfaces.
Typically experiment provides boundary conditions at the wall for transport along
the flux tubes of which each such layer may be supposed to be composed. Thanks to the axisymmetry 
of the magnetic field it is only necessary to consider a 1-D problem for each tube. The solutions
of these 1-D problems for the different layers are then combined to produce a 2-D
`onion-skin' solution for which agreement to within~$20$\,\% of an explicitly 2-D fluid  model
is obtained in many cases~\cite{St97Code}.


The 3-D fluid models of \nep\ approximate in turn 5-D gyro-averaged or 6-D full phase-space dynamics.
These more complex models may be needed particularly when the plasma collisionality
is low, but then there is a change in physical emphasis in that the influence of the
boundaries may be directly transmitted throughout the low collisionality region.
For the spatially~5-D/6-D models, even at higher collisionality,
theoretical analysis produces series of correction terms to be added into the fluid models.
These are in addition to other physical effects such as plasma-neutral particle interaction,
radiative loss etc., all adding up to a substantial additional challenge for usage of MFMC in \nep.

The works of Najm and collaborators~\cite{Hu18Glob, Ge19Prog} have delivered a workflow 
primarily aimed at producing at an optimal design under uncertainty:
\begin{enumerate}
\item Global sensitivity analysis, to identify key parameters, see \Sec{stage1}
\item Forward UQ using adaptive sparse sampling, see \Sec{stage2}
\item PCE surrogate for inverse UQ and OUU, see \Sec{stage3}
\end{enumerate}
%\item Full and reduced fidelity models randomly sampled (MFMC, possibly also MLMC) to produce 
%PCE surrogate for the QoIs and/or 
%\begin{enumerate}
%\I Surrogate propounded as a reduced fidelity model with PCE for parameters only
%\I Surrogate is MCMC sampled using Bayesian inference to compute PCE
%\end{enumerate}
%\item PCE reduced by use of LASSO to give sparse PCE surrogate
%\item Surrogate subject to adaptive sparse sampling (also MLMC if stochastic) to evaluate Sobol indices
%\item Surrogate subject to OUU to produce optimal design 
%\end{enumerate}

Whilst the above workflow has very sophisticated components, it does not cover all \nep\ applications.
For instance, the physicist seeking to gain better understanding of the SOL may not
be immediately interested in a particular optimisation. Such usage might require
only relatively simple scans in one or two parameters to compare with theory, for which
a local sensitivity approach (ie.\ examining the effect of small changes to other parameters
one at a time) could provide adequate UQ.

More demanding is to improve understanding of the often highly nonlinear
dynamics of the SOL, where it is possible for small-scale structures such as internal layers
or local hot-spots to form. An example of the former might be the radiative fronts
which develop as the plasma detaches from the first walls, where it could conceivably
be critical for the numerical solution to resolve these fronts. One approach to reduce
the uncertainty as to whether the discrete representation is adequate is `bifurcation-tracking'
whereby a branch of solutions is followed from where it emerges by initial linear 
instability on well-characterised physical scales, into the nonlinear regime.
The physicist user may thus, after examining surrogate behaviour, want \nep\ to produce
many more higher fidelity solutions. These solutions could usefully provide feed-back that
the parameters of the
surrogate require modification, and hence this user's particular workflow might become very involved
as it seeks to `home in' on detachment processes.

Hence there are different potential workflows aimed at optimising designs, improving physical
understanding, and indeed to produce optimal surrogates for use in device control~\cite{bruntonkutz}.
The final decisions as to what should be implemented in \nep, await the production
of the Milestone Reports~M2.4.2 and~M2.5.2 scheduled for August~2021.
