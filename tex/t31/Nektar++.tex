\subsection{Nektar++}\label{sec:nektar}


Nektar++ is described (\cite{nektarwebsite}):
\begin{quote}
{\it `Nektar++ is a tensor product based finite element package designed to allow one to construct efficient classical low polynomial order h-type solvers (where h is the size of the finite element) as well as higher p-order piecewise polynomial order solvers.'}
\end{quote}
The code is opensource under the MIT licence.  
It is modern and object-oriented, with the initial release dating from 2006.

\paragraph{Physics and Scope}

Nektar++ is a spectral / hp element framework for a range of PDEs, which can be hyperbolic, parabolic, or elliptic.
The code includes pre-written solvers for acoustic, advection-diffusion-reaction, cardiac electrophysiology, compressible flow, incompressible Navier-Stokes, linear elasticity, pulse wave, and shallow water problems.
The underlying method can be continuous or nodal discontinuous Galerkin.
There are the options of implicit, explicit, or IMEX time-stepping (though availability depends on solver choice).

Nektar++  does not currently support particles or any Boltzmann-type equations,
although for example the treatment of particle motion in spectral / hp elements
is described in the textbook~\cite{karniadakissherwin}.

%The current physics scope does not extend to particles or Boltzmann-type equations (rather, it is a generic PDE framework supporting up to 1 time and 3 space).

The solvers can be coupled to provide multiphysics capability.  
An MPI coupling implementation consists of a smooth interpolating field layer that is receiver-centric in that the receiver is provided with pointwise field data from the sender (then to be resolved to the local basis).  
Data transfer is handled using the CWIPI library~\cite{cwipiwebsite,Ca19Test}.  
This framework can couple different Nektar++ solvers, or couple one such to a separate application.

\paragraph{Framework} 

Nektar++ uses a modern object-oriented C++ framework \cite{Ca15Nekt}.  Multi-threading is via MPI.

The framework is multi-layered in that there is a structured hierarchy of C++ components (much of the structure is evident in the directory structure of the downloaded source code) and the code structure mirrors the mathematical formulation (see library descriptions below).  
The time-steppers are agnostic to the FEM/solver details and the solvers share much of the underlying FEM codebase.  
Solvers inherit from base classes, for example one for unsteady flows.

The implementation is partitioned into six libraries, organized into utilities (parallel communications, DFT, maths routines), reference elements, physical elements, domain geometry and mesh, global field data operations and solver base classes.

There is no GUI; input is via XML file (mesh, solver, parameters, boundary conditions); output is via file (HDF5 is supported).  
Tools are provided for converting input meshes from popular formats ({\it NekMesh}, which also can make a mesh `higher order', meaning the inclusion of curved-sided elements) and for converting the output to popular formats ({\it FieldConvert}).  
Note that the latest version \cite{Mo20Nekt} supports input meshes in HDF5 format in order to avoid the need for a single process to read the entire mesh during setup.

The code is designed to run on anything from a single desktop up to many thousands of processor cores.

The development workflow involves a monthly stable release.  
There is an extensive testing framework.

The focus of the user community seems primarily engineering and biomedical.  
The entry barrier is lowered by the availability of a precompiled binary in addition to the source code.  
Documentation includes a User's Guide and API description extracted using {\it doxygen}.
The code is under active development by groups at the University of Exeter, Imperial College London
and the University of Utah.


\paragraph{Summary}

Nektar++ has a well-designed framework, is written in a clean way in a modern, object-oriented language, and it looks relatively easy to add new solvers.  
The option to download a ready-built code (as alternate to the full source) provides an easy entry point for new users for whom there is no need to modify the code.  Nektar++ has proven scaling up to tens of thousands of processors.

The caveat is that the existing framework requires new
physics to be described by PDEs in up to 1+3 dimensions, ie.\ maximally time-dependent
in $3$ space dimensions, and
solved using by the restricted set of supported finite-element methods.

There does seem to be some `price of admission' in terms of user expertise; for example, it is not clear what guidance user gets regarding the maximum timestep in explicit methods to avoid violating the Courant stability limit.

%\paragraph{Documentation}
%User's Guide.  API description extracted using {\it doxygen}.
%
%\paragraph{Workflow / release cycle}
%Stable releases monthly.  Extensive testing framework.
%
%\paragraph{Community}
%Focus seems to be engineering / biomedical.  Under active development by groups at Imperial College London and the University of Utah.  Community blog.  Precompiled binary available in addition to source / {\it CMake}.



%


