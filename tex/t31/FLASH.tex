\subsection{FLASH}\label{sec:flash}


FLASH \cite{flashwebsite} is described:
\begin{quote}
{\it `The FLASH code is a publicly available high performance application code 
which has evolved into a modular, extensible software system from a collection 
of unconnected legacy codes.'}
\end{quote}

Dubey~et~al~\cite{Du09Exte} add to the description {\it `... a Massively-Parallel,
Multiphysics Simulation Code'}.  FLASH is a community astrophysics code; it has been a clear 
success in terms of uptake since its inception in Y2K - there are now over 1000 
citations on the FLASH website~\cite{flashwebsite}.


The code formed when legacy solvers (written in Fortran 77) were refactored 
into a formal software engineering framework.  Subsequent additions to the code 
have been driven by the physics requirements of the users; Dubey~et~al~\cite{Du09Exte} 
references in its title the code's {\it `Extensible Component-Based 
Architecture'}.

\paragraph{Physics and Scope}

Capabilities include compressible hydrodynamics, MHD, nuclear burning, 
equations of state, radiation, laser drive, fluid-structure interactions, and 
also particles, which may simply be non-interacting `tracer' particles.
Kinetics (represented by the Boltzmann equation) and 
gyrokinetics seem absent from the codebase.  Gravity (an always-additive and 
therefore long-range force) is clearly a pressing matter for astrophysics but 
is irrelevant for \nep; likewise any cosmolological / relativity physics. 
The codebase uses only rectilinear computational meshes.


\paragraph{Framework} 

The core algorithms are written in Fortran~90 and wrappers are written in~C.

The code is structured into components called Units (many of which implement a 
specific component of the physics capability; an example is particles) and the 
code for an individual Unit is demarcated using the Unix directory structure.  
A build includes only the Units needed for the problem in question.  The top 
level of a Unit's directory defines the API of the Unit.  Unit-private data is 
passed laterally by reference using accessor functions as needed.  Subunits 
allow alternative implementations and also give scope to shared data, avoiding 
proliferation of accessor functions.  Simulations are specified via a 
collection of config files included at various places in the directory 
structure: files lower down the hierarchy can override ones inherited from 
higher up.  Users can incorporate novel physics by adding a new Unit.  

The framework is object-oriented and encapsulates data; however, this is 
accomplished using the directory structure and a set of parsed config files 
(written in a domain-specific language), rather than using an explicit 
object-oriented language.  It is stated in \cite{Du09Exte} that this may be 
beneficial for portability.

Each simulation would appear to necessitate compilation of a new executable; 
the object-oriented character is strictly at compile (i.e. build) time and it 
seems the code must be rebuilt to change what the simulation does (excepting 
parameters read in by config files).

There is no GUI; user interface is via text files / command line.  Multiple I/O 
libraries are supported e.g. HDF5 and parallel netCDF (see \cite{Du09Exte}).

The workflow includes a test suite, which is run nightly.
There is an integrated unit test framework and a self-test suite for regression testing.  
An ensemble of quick-running examples has been proven by a user survey to be a popular feature.  

There is rigorous gatekeeping for non-internal extensions to the release version of the code, 
including a requirement for a specific unit test and a commitment to provide support.

The user community consists of mainly astrophysicists, but there are some other areas of application.
There is an email users' group, with support provided by experienced users and the developers.
The code is free on request; users may modify but not redistribute their own copies.
Documentation includes a User's Guide and also API description extracted using {\it robodoc}.
One interesting aspect has been user surveys, which
have provided insight into how the code was used and also 
indicated the most widely appreciated aspects of the code.
\cite{Du09Exte} cites feedback from close to 300 respondents, which gives
the top reasons for FLASH usage as
\begin{enumerate}
\item Adaptive mesh refinement (64\%)
\item Flexible (47\%)
\item  Good documentation (47\%)
\item  Availability of examples (43\%)
\item  Computationally efficient (37\%)
\item  Easy to Use (36\%)
\end{enumerate}


\paragraph{Summary}

FLASH is a true multiphysics code and incorporates fields and particles.  Its 
authors claim efficient operation up to tens of thousands of processors 
(\cite{Du09Exte} includes an example of 16 million particles running on 32,768 
nodes of IBM BG/L machine).  %Our anticipated scaling requirements exceed this.



Disadvantages are associated with the rather amphibious nature of the code: it 
uses build-time methods to achieve object-oriented features from 
non-object-oriented languages and so must be largely recompiled for each 
simulation; also, the code is written in more than one language.  Writing in 
Chapter~1 of \cite{carverhong}, the authors of the code state that the 
framework was chosen because the original physics capability was in Fortran and 
to refactor in an object-oriented language was not an option; it is also 
conceded that {\it `... FLASH ... has a big challenge in adapting for 
future heterogeneous platforms'}.  It appears the use of the code may have 
`peaked': there are only five publications listed from 2019 and ten from 2018; 
cf. 93 in 2010 \cite{flashwebsite}.


%\paragraph{Documentation}
%User's Guide.  API description extracted using {\it robodoc}.
%
%\paragraph{Workflow / release cycle}
%Nightly test suite.  Gatekeeping for non-internal extensions to Release.  
%Integrated unit test framework.  Self-test suite for regression testing.  Suite 
%of quick-running examples (proven, by a user survey, a popular feature).
%
%\paragraph{Community}
%Mainly astrophysics, but some other.  Email users' group (experienced users and 
%devs provide support).  User surveys.  Code free on request; users may modify 
%but not redistribute own copies.
%%


