The milestone report~M3.1.1~\cite{y1re311} highlighted the design challenge 
represented by project \nep. The present work investigates the state-of-art
in software design for multiphysics applications potentially relevant to
the project, focusing on `software frameworks'. In this document, suitability
for deployment at the Exascale is not an over-arching requirement, there is at least
equal interest in techniques employed both to ensure a flexible software
design, and also those which make the resulting code attractive to others so as to build
a wider user and developer community.

%Wikipedia: framework dictates flow of control, extensible but not modifiable (Wikipedia
%entry closely equivalent to Sommerville 15.2).

Frameworks are defined by Sommerville~\cite[\S\,15.2]{sommerville} (after
Schmidt et al~\cite{Sc04Leve}) as\\
\emph{an integrated set of software artefacts (such as classes, objects and components) that
collaborate to provide a reusable architecture for a family of related applications}\\
(In this context, component implies a collection of objects~\cite[\S\,2.1.3]{sommerville}
whereas class and object are apparently synonymous.)\\
Sommerville treats frameworks in a chapter on software re-use, consistent
with the idea that saving labour is the essence of their attractiveness to others.
Sommerville confuses matters slightly by stating first that ``frameworks are
language-specific", then that a ``framework can incorporate other frameworks"
in the context where it is clearly indicated that the encompassing
framework might be in a different programming language to the others. 
It does not therefore seem unreasonable to drop the language-specific requirement.
It can easily be agreed~~\cite[\S\,15.2]{sommerville} that ``Frameworks are not libraries
in that they provide a skeleton architecture for the application", but while
normally frameworks will be ``implemented \ldots in an object-oriented programming language",
this does not seem to be strictly necessary.

There will be overlap with M3.3.2 work on design patterns in that the skeleton
architecture will usually be a pattern. The pattern however has to be one which
dictates the flow of control, and which is extensible. The definition of skeleton
architecture extends to demand that the pattern ``not be modifiable", whereas it
would seem reasonable to allow changes within an overall pattern.

``Multiphysical" software would be better nomenclature than ``multiphysics" since there
is no restriction to the field of physics
in what appears to be a defining bibliography for multiphysics frameworks by Babur~et~al~\cite{Ba15Mult}.
In 2015, there were $144$ entries in categories listed by ref~\cite{Ba15Mult} which
included 'Chemistry and Chemical Engineering' and 'Life and Health Sciences', thus
fewer than~$100$ would be relevant to this report. Clearly much
of the software listed was likely to have become obsolete even by~2015,
although there were a few examples where one framework had morphed into another.

Potentially relevant multiphysics software listed in~\cite{Ba15Mult} includes
(all C++ unless different language stated in parenthesis after name), with originating
organisation before brief description and website:
\begin{itemize}
\item PETSc  -- Argonne (ANL), Portable, Extensible Toolkit for Scientific Computation \url{https://www.mcs.anl.gov/petsc/}
\item FEniCS  -- (Python, uses PETsc) International, computational mathematical modeling \url{https://fenicsproject.org/}
\item Firedrake  -- (Python, uses PETsc) Imperial College, solution of partial differential equations using the finite element method \url{https://www.firedrakeproject.org/}
\item Trilinos  -- Sandia, project for parallel solver algorithms and libraries \url{https://github.com/trilinos/Trilinos}
\item OpenFOAM  -- ESI group, Computational Fluid Dynamics software \url{https://openfoam.com/}
\item Nektar++, see \Sec{nektar}  -- UK academic, spectral/hp element \url{https://www.nektar.info/}
\item waLBerla  -- Germany, widely applicable Lattice Boltzmann from Erlangen \url{https://www.walberla.net/index.html}
\item LAMMPS  -- Sandia, Large-scale Atomic/Molecular Massively Parallel Simulator \url{https://lammps.sandia.gov/}
\item preCICE  -- Stuttgarts \& TUM, Precise Code Interaction Coupling Environment \url{https://www.precice.org/}
\item POOMA  -- Los Alamos (LANL), library supporting element-wise, data-parallel, and stencil-based physics computations using one or more processors  \url{http://savannah.nongnu.org/projects/freepooma} last updated 2005.
%\item HLA  --  IEEE standard for distributed simulation, combining several simulations
\item FLASH, see \Sec{flash}  -- Chicago, radiation magnetohydrodynamics~(MHD) simulation code for plasma physics and astrophysics, \url{http://flash.uchicago.edu/site/index.shtml}
\item MOOSE  -- Idaho, Multiphysics Object Oriented Simulation Environment \url{https://moose.inl.gov/SitePages/Home.aspx}
%\item CSMP 
\item XMSF  -- US, Extensible Modeling and Simulation Framework using web services \url{https://sourceforge.net/projects/xmsf/}
\item OASIS4 -- France, exchanges of coupling information between numerical codes representing different components of the climate system \url{https://portal.enes.org/oasis} last updated 2019 
%\item MCT
\end{itemize}
Selected codes are described in more detail in the sections indicated.
Qualifications for inclusion in the list are that source code is available without significant fee,
(although it may be necessary to register with the authors prior to download)
and that significant updates were made to the code repository in~2020.
(The web-sites listed were current as of June~2020.)
FLASH is also described in Carver et al~\cite[\S\,1]{carverhong} along with Amanzi/ATS, which
employs the Arcos framework, see \cite{Co16Mana} discussed in more detail in \Sec{arcos}.
%mentioned in Carver et al~\cite[\S\,1]{Du16Soft}

Babur et al also mention HLA, High Level Architecture, which evolved to 
become an IEEE standard for distributed simulation, leaving behind its origins as a single piece of software.
Although OASIS~\cite{Va13OASI} is present, ESMF, the Earth System Modeling
Framework~\cite{Hi04arch} and NEMO, Nucleus for European Modelling of the Ocean NEMO~\cite{nemowebsite},
are missing from the bibliography, presumably on the 
grounds that they are only used by Earth scientists, although both coordinate modelling of many
different physical processes. Similarly there are examples of astrophysics
codes besides FLASH, such as the Pencil~Code~\cite{pencilwebsite} which
is centred on compressible magnetohydrodynamics
and the `other'~NEMO~\cite{othernemowebsite} (Not Everybody Must Observe) which is a stellar dynamics toolbox,
which are not mentioned, although they have grown to include a lot of extra physical processes.
The paper by Theurich~et~al~\cite{Th16eart} indicates that
ESMF was important for subsequent developments,
and together with the recent survey by Groen et al~\cite{Gr19Mast} that
covers multiscale computing software, will be discussed further in \Sec{summ}.

%VORPAL (particles only) replaced Cary's FACETS (1)?
%SAMRAI - adaptive -> grid gen

Regrettably, investigation of the usage of the term ``framework" in
talks at the SIAM~PP20 meeting indicated wide usage as a synonym for library.
Here, consideration will only be given to software which represents
significantly more complexity than a library plus a test harness.
Thus, although there are approximately
$50$~separate mentions of the term ``framework" in the online programme~\cite{pp20},
the number of true frameworks in smaller, and the number that are multiphysics
is smaller still. The following are noteworthy (author(s) and session(s) at end of line):
\begin{itemize}
%\item CabanaMD  --  ECP, Molecular dynamics proxy application based on Cabana https://github.com/ECP-copa/CabanaMD
%\item MaMiCo  -- particle+continuum (0)
\item OpenFPM  -- Dresden, particles and mesh simulation \url{http://openfpm.mpi-cbg.de/}
, last release 2/19 . Incardona, MS22
\item Legion  -- Stanford, data-centric parallel programming system for writing portable high performance programs targeted at distributed heterogeneous architectures \url{https://legion.stanford.edu/}. Shipman, MS36
%\item COSMO  framework for regional weather prediction in Europe, being superseded by ICON
%not opensource \url{https://code.mpimet.mpg.de/projects/iconpublic}
\item AMReX -- ECP, massively parallel, block-structured adaptive mesh refinement \url{https://amrex-codes.github.io/amrex/}. Gott, MS12, Almgren, MS46
\end{itemize}
There were other mentions of Adaptive Mesh Refinement (AMR) frameworks at PP20, notably by Shimokawabe in MS25, but most concerned AMReX, which
is both the name of the software and the Exascale Computing Project~(ECP) co-design centre~\cite[\S\,3]{Al20Exas}. It is notable
that the only other mention of framework within the ECP programme~\cite{Al20Exas} is 
the Multiscale Modeling Framework~(MMF) which appears exclusively
in the context of E3SM, a cloud-resolving version of CESM, the Community Earth System Model
which is compared to the ESMF in ref~\cite{Th16eart}.

%Cabana is a performance portable library for particle-based simulations. 
%Cabana is developed as part of the Co-Design Center for Particle Applications (CoPA) within the Exascale Computing Project (ECP) under the U.S. Department of Energy.
%macro-micro-coupling tool (MaMiCo) was developed to ease the development of and modularize molecular-continuum simulations, retaining sequential and parallel performance.

The precise criteria for identifying software to be ``multiphysics" could be debated.
Babur~et~al include only one current code, OMFIT, relevant to magnetically confined
nuclear fusion, consistent with
the idea that fusion basically constitutes a single area of physics. Indeed,
since  \nep \ will be concentrating on plasma physics, the use of the term multiphysics 
in the context of the project might be queried. However, in magnetic fusion work, there
seems to be an empirical definition as software which combines the function of two or more
existing separate codes. Thus the SOLPS software~\cite{Bo16Pres} couples a fluid
code~(B2) for the ionised component of the plasma with a particle code~(EIRENE) for
the neutral component. A better definition might be as modelling software
capable of treating coupled effects using different physical representations 
and solution algorithms for the interacting components.
Project \nep \ software should not become ``multiphysics" only when 
it incorporates plasma chemistry effects.

The following is a representative although far from exhaustive list of
frameworks relevant to fusion, again C++ unless stated otherwise in parenthesis:
\begin{itemize}
\item OMFIT  -- (Python) General Atomics (GA), One Modeling Framework for Integrated Tasks \url{https://omfit.io/}
\item BOUT++, see \Sec{bout}  -- York, Plasma fluid finite-difference simulation code in curvilinear coordinate systems  \url{http://boutproject.github.io/ }
%Me15Inte
\item ETS  -- (Mixed) EU,  European Transport Simulator within Integrated Tokamak Modelling~(ITM).  IMAS repository %St18Towa
\item JOREK, see \Sec{jorek} --  (Fortran) EU, MHD including more detailed fluid plasma models \url{https://www.jorek.eu/}
\item SMARDDA, see \Sec{smardda}  -- (Object-oriented Fortran) UK, Surface interactions of rays and particles. IMAS repository as SMITER kernel
\item Alya  -- Spain, High Performance Computational Mechanics \url{https://www.bsc.es/research-development/research-areas/engineering-simulations/alya-high-performance-computational }
\item VECMAtk -- EU, includes MUSCLE~3, formerly ComPat \url{https://www.vecma-toolkit.eu/}
\end{itemize}
BOUT++, JOREK and SMARDDA will be described in more detail in the sections indicated.
The IMAS repository is not publicly available, but is usually accessible on request by  
members of the fusion community to ITER.
The Alya software is free subject to a collaboration agreement. Alya~\cite{Va16Alya} and VECMAtk
have been included as more general frameworks which have been used for fusion work,
respectively reactor modelling~\cite{Gu18Newh} and multiscale tokamak core plasma modelling~\cite{Lu19Comp}.
Neither has yet seen any uptake in fusion outside the collaborators in the EU-funded
projects to develop them. The tool-kit VECMAtk is directed towards Uncertainty Quantification
and will be described in detail elsewhere.

OMFIT~\cite{Me15Inte} has attracted and continues to attract a very wide range of users.
Many of the reasons for this are to be found in the first paragraphs of \Sec{attract},
but aside from interactivity, one specially highlighted by
the authors is that it integrates people engaged in previously fragmented efforts in experimental
data analysis, often including multiphysics simulations, into
a community based around the software. It goes almost without saying
that the larger the usage of software, the more reliable it becomes,
provided there is a good management, especially of error reporting.
Much of the OMFIT software relates to ancillary tasks such as reading and
converting between data formats and visualisation, for which users would
otherwise to write their own software, and here OMFIT's comprehensive documentation,
extending to YouTube videos, is clearly vital.

The present work contains in \Sec{taskwork} detailed descriptions of one
obsolescent and six other frameworks still under active development. 
\Sec{compare} provides a tabulated overview of the seven. \Sec{summ} summarises
what has been learnt about what makes software attractive in \Sec{attract},
then \Sec{flex} indicates how
to begin producing complicated designs that nonetheless possess much flexibility,
and lastly \Sec{exa} sets the scene for the transition to the Exascale.
%Frameworks are implemented as a collection of concrete and abstract object
%classes in an object-oriented programming language. Therefore, frameworks are
%language-specific. Frameworks are available in commonly used object-oriented
%programming languages such as Java, C#, and C++, as well as in dynamic languages
%such as Ruby and Python. In fact, a framework can incorporate other frameworks,
%where each framework is designed to support the development of part of the applica-
%tion. You can use a framework to create a complete application or to implement part
%of an application, such as the graphical user interface.

%------------------------------

%OLYMPUS - historical approach
%SMARDDA modules
%raysect \url{https://github.com/raysect} is a library, although described as a framework
%
%
%PP20 programme 87 mentions of word "framework" (/2=43)
%multiphysics 18/2=9
%https://www.siam.org/Portals/0/Conferences/PP20/PP20_PROGRAM_COLOR_V4_with_abstracts.pdf?ver=2020-02-06-095558-400
%CabanaMD
%MaMiCo particle+continuum (0)
%OpenFPM
%AMReX (0+) but is ECP
%AMR framework Shimokawabe
%(Many AMR)
%Shipman Legion Programming System
%Checkout COSMO framework for regional weather prediction in Europe
%
%
%Metrics:
%How good at each user/developer level?
%Long-term maintenance, maintainability
%Range of multiphysics
%Entries in swMATH \url{https://swmath.org/about_contact}
%How robust - "actionable" Are Python, Java robust enough?
%

%\begin{verbatim}
%\end{verbatim}
