Examination of the literature and conference attendance described here,
plus the Birmingham meeting~\cite{y1re111a,y1re111b,y1re121} have indicated
a gap in the knowledge of how to transition efficiently between
a fluid representation and a particle representation and vice versa, particularly when spectral
elements are used to represent the fluid. 

Concerning particle methods, there are several potentially disruptive approaches that need
evaluating before producing a detailed software design. These are
\begin{enumerate}
\item Use of Quasi-Monte-Carlo techniques and related sampling techniques
\item Particle or kinetic enslavement
\item Timestep-robust particle tracking
\end{enumerate}

The aforementioned points require further research and literature analysis, for which
indicative references have been provided in \Sec{taskwork}.

There are also practical issues concerning storage and use of a phase-fluid that overlap with other tasks.
Although many older PIC codes successfully used 32-bit precision,
the use of greatly reduced precision in representing particles is expected to
increase `noise' to unacceptable levels.
Hence, for any usage of particles, particularly
when fluid approximations are being employed for the main species, it is likely that
the major memory cost will be storage of the particles. Hence the demands of particle
storage will determine how \emph{all} field data is assigned to memory.

Other issues relating to use of a phase-fluid approach
seem worthy of examination. It seems that spectral/hp
element might be efficiently used in velocity as well as physical space, and 
therefore evaluating performance here ought to
form part of the general assessment of the performance of spectral elements.

