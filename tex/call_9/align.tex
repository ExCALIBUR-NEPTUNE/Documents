The most important aim of work package FM-WP4 is \ldots to
ensure that the software is fit for its defined purpose, \ldots
by defining suitably flexible code structures and related e-Infrastructure 
for users, ultimately supporting uptake of the new code(s). In order to achieve 
this aim, this work package will coordinate across the other tasks FM-WP1-3, to 
ensure that outcomes are compatible and of high quality.
%FM-WP4 includes management 
%and coordination tasks that will grow as the project matures, connecting with,
%for example, the EUROfusion E-TASC~(TSVV) programme~\cite{etasc}, the currently
%active EPSRC multiscale plasma turbulence programme~\cite{turbtp} and the US~ECP
%programme~\cite{ecp}.
This work package along with FM-WP1 has been prioritised for an early start, in order
that good scientific software engineering practice can be agreed quickly, and well 
documented interfaces to components can be made available rapidly to ensure that best 
practice design is incorporated as early as possible.

This task addresses the following items, at least in part, as enumerated in the
Fusion Modelling System Science Plan~\cite{sciplan}

\begin{enumerate}
%\item Allocation of resource between tasks and setting project priorities. 

%\item[2] Ensuring a consistent choice of definitions (ontology) of objects or equivalently classes. 

\item[3] Definition of common interfaces to components for data input and output. 

\item[4] Design of suitably flexible data structures for common use by all developers. 

\item[5] Establishment, promotion and support of good scientific software engineering practice. 

\item[6] Evaluation and deployment of performance portability tools and DSLs targeting 
Exascale-relevant architectures. 

%\item[7] Integration of the developed software into a VVUQ framework (exploiting common approaches developed under XC-WP1 and XC-WP2). 

%cal6\item Coordination of a benchmarking framework for correctness testing and performance evaluation of the developed software stack. 
\end{enumerate}

