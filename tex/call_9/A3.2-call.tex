%&pdfLaTeX
% !TEX encoding = UTF-8 Unicode
\documentclass{article}
\usepackage{ifxetex}
\ifxetex
\usepackage{fontspec}
\setmainfont[Mapping=tex-text]{STIXGeneral}
\else
\usepackage[T1]{fontenc}
\usepackage[utf8]{inputenc}
\fi
\usepackage{textcomp}

\usepackage{array}
\usepackage{ulem}
\usepackage{amssymb}
\usepackage{fancyhdr}
\renewcommand{\headrulewidth}{0pt}
\renewcommand{\footrulewidth}{0pt}
\usepackage{color}

\pagestyle{fancy}
\rhead{}
\rfoot{}
\chead{}
\cfoot{}
\lhead{             
 \textbf{OFFICIAL -- SENSITIVE}}
\lfoot{\thepage{}}
\definecolor{color02}{rgb}{0.00,0.00,1.00}
\definecolor{color18}{rgb}{0.14,0.23,0.36}
\definecolor{color20}{rgb}{0.14,0.16,0.18}

\begin{document}

\textbf{Call Title:} FM-WP4 Code structure and coordination Activity 3.2 - Investigation 
of DSL and code generator technologies

\textbf{Value:} 6-12? PM (100\% FEC)

\textbf{Grant Funds for the period: }01/09/2020 -- 31/03/2021 (30/09/2021)

\textbf{Aligns to work package [FM-WP4 }Code structure and coordination\textbf{]}

The most important aim of this work package will be to drive user engagement and 
ensure that the software is fit for its defined purpose, first by requirements 
capture, then by defining suitably flexible code structures and related e-Infrastructure 
for users, ultimately supporting uptake of the new code(s). In order to achieve 
this aim, this work package will coordinate across the other tasks FM-WP1-3, to 
ensure that outcomes are compatible and of high quality. There will be management 
and coordination tasks that will grow as the project matures, connecting with the 
EUROfusion E-TASC (TSVV) programme, the EPSRC T.P. Turbulence Programme [20] and 
the US ECP programme etc. 

Coordination tasks will include (but are not restricted to): 

\begin{itemize}
\item Allocation of resource between tasks and setting project priorities. 

\item Ensuring a consistent choice of definitions (ontology) of objects or equivalently 
classes. 

\item Definition of common interfaces to components for data input and output. 

\item Design of suitably flexible data structures for common use by all developers. 
{\color{color18} \textbf{ }}

\item Establishment, promotion and support of good scientific software engineering 
practice. 

\item Evaluation and deployment of performance portability tools and DSLs targeting 
Exascale-relevant architectures. 

\item Integration of the developed software into a VVUQ framework (exploiting common 
approaches developed under XC-WP1 and XC-WP2). 

\item Coordination of a benchmarking framework for correctness testing and performance 
evaluation of the developed software stack. 
\end{itemize}

Along with FM-WP1, this work package will be prioritised for an early start, as 
good scientific software engineering practice needs to be agreed quickly, and well 
documented interfaces to components need to be available early to ensure that best 
practice design is embedded from the start.

\textbf{Objectives:}

The overarching objective of this call is to help \nep \  to codesign its software 
base for exascale hardware and software. It is widely accepted that the pre-exascale 
and exascale generation of HPC systems will use compute nodes with heterogenous 
designs whose hardware architectures and programming models could vary significantly 
amongst vendors. Therefore, achieving performance portability by separation of 
concerns is the best option for the user applications.

There are several approaches to this aim:  

\begin{itemize}
\item domain specific language (DSL) which offers an abstraction level suitable for 
the domain specialist to express specialised algorithms, and which generates automatically 
the code for the target hardware. As examples we mention: PSypclone, GridTools, 
PyOP2, etc 
\item Low level programing models focused on performance portability at the numerical 
kernel level such ad Kokkos, RAJA, OneAPI, HIP, etc 
\item At an intermediate level there are frameworks (such as PETSc, Trilinos) or libraries 
which use generic mathematical abstraction and algorithms, and which cover multiple 
hardware architectures internally.
\end{itemize}

The following specific objectives are put forward to support \nep \ : 

\begin{itemize}
\item An in-depth and continuously updated analysis of the available and upcoming 
hardware and the associated software solutions. 

\item Performance evaluations and tests for the algorithms of interest. For \nep \  
two main branches need to be explored: higher order finite elements for fluids 
and particle methods. Other algorithms could be found to be of interest as other 
parallel activities of \nep \  are progressing. 

\item Design elements: we need to explore the relationships between the data layouts, 
communication patters needed by a given code component and the abstraction offered 
by the DSL or similar technologies described above.
\end{itemize}

\textbf{Anticipated outputs or results:}

\begin{itemize}
\item
Deliver a report on performance and portability of various approaches and their 
suitability for \nep \  algorithms.

\item
Deliver a report on designs for data layouts and internode communication suitable 
for the abstraction used by various DSL and code generators.

\item
Proof of concept codes in the initial stage and contribution to the proxi-apps 
develop by parallel calls in the later stage.
\end{itemize}

For Background:
Successful completion of this activity will require close communication with the 
UKAEA. As well as monthly highlight reports, we envisage that this will include 
three meetings for the monitoring of the project [a kick off meeting (in person 
or video-conference); a mid-term review meeting (in person); and a final meeting 
(in person)] and at least monthly conference calls with UKAEA employees. 

%\vspace{12pt}
%\textbf{Evaluation Weighting: }
%
%\vspace{12pt}
%\begin{tabular}{|>{\raggedright}p{35pt}|>{\raggedright}p{158pt}|>{\raggedright}p{29pt}|>{\raggedright}p{73pt}|}
%\hline
% ~ & C\textbf{riteria (5 points available per question)} ~ & \% ~ & M\textbf{arked 
%against document} ~\tabularnewline
%\hline
% ~ & Q\textbf{uality of Research Plan} ~ & 8\textbf{0\%} ~ & R\textbf{esearch 
%Plan} ~\tabularnewline
%\hline
% \textbf{80\% QUALITY} ~ &  a. Plan and Approach  ~\linebreak{}
% ~ & 4\textbf{5\%} ~\linebreak{}
% ~ & sections~4, 6 and 7~\tabularnewline
%\hline
% &  b. Benefits and alignment to Work Package objectives  ~ & 2\textbf{0\%} ~ & section~2~\tabularnewline
%\hline
% &  c. Relevant experience  ~ & 1\textbf{0\%} ~ & section~7.1 ~\tabularnewline
%\hline
% &  d. Dependencies and Risks  ~\linebreak{}
% ~ & 5\textbf{\%} ~\linebreak{}
% ~ & sections~3, 5,~8~and 9 ~\linebreak{}
% ~\tabularnewline
%\hline
% ~ & V\textbf{alue for money } ~ & 2\textbf{0\%} ~ & R\textbf{esearch Plan} ~\tabularnewline
%\hline
%2\textbf{0\% VFM} ~ & Value for money is exceptionally important to all grant 
%funding. In this section, the bid needs to demonstrate efficiencies and value-added 
%activities considering time, quality and cost.  ~ & 2\textbf{0\%} ~ & section 
%10\tabularnewline
%\hline
%\end{tabular}

\textbf{GLOSSARY: }

\begin{tabular}{|>{\raggedright}p{74pt}|>{\raggedright}p{240pt}|}
\hline
T\textbf{erm or Acronym} & D\textbf{efinition}\tabularnewline
\hline
\nep \  & Code name for the ``Fusion Modelling System'' use case commissioned by 
Met Office to UKAEA under \exc \   programme.\\
\hline
PSyclone & PSyclone is a code generation system that generates appropriate code 
for the PSyKAl code structure developed in the GungHo project.\linebreak{}
{\color{color02} \emph{https://github.com/stfc/PSyclone}}\\
\hline
GridTools & The GridTools framework is a set of libraries and utilities to develop 
performance portable applications in the area of weather and climate.\linebreak{}
{\color{color02} \emph{https://github.com/GridTools/gridtools}}\\
\hline
PyOP2 & Framework for performance-portable parallel computations on unstructured 
meshes http://op2.github.com/PyOP2\\
\hline
Kokkos & Kokkos C++ Performance Portability Programming EcoSystem\linebreak{}
{\color{color02} \emph{https://github.com/kokkos/kokkos}}\\
\hline
RAJA & R{\color{color20} AJA Performance Portability Layer (C++)}\linebreak{}
{\color{color02} \emph{https://github.com/LLNL/RAJA}}\\
\hline
OneAPI & A Unified, Standards-Based Programming Model, {\color{color02} \emph{https://software.intel.com/en-us/oneapi}}\\
\hline
HIP & C++ Heterogeneous-Compute Interface for Portability {\color{color02} \emph{https://gpuopen.com/compute-product/hip-convert-cuda-to-portable-c-code/}}\\
\hline
PETSc & Portable, Extensible Toolkit for Scientific Computation Toolkit for Advanced 
Computation {\color{color02} \emph{https://www.mcs.anl.gov/petsc/}}\\
\hline
Trillinos & Object-oriented software framework for the solution of large-scale, 
complex multi-physics engineering and scientific problems\linebreak{}
{\color{color02} \emph{https://trilinos.github.io/}}\\
\hline
\end{tabular}

\newpage

\end{document}
