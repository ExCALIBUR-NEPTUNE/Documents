\documentclass[12pt]{article}
\usepackage{graphicx}
\usepackage{color}
\newcommand{\Sec}[1]{Section~\ref{sec:#1}}
\newcommand{\Fig}[1]{Figure~\ref{fig:#1}}
\begin{document}
\centerline{\bf\LARGE A Rough  Guide to Software Development}
\vfill
\centerline{\Large Wayne Arter \today}
\vfill
\section{Introduction}

This document is based on the ECSS 
(European Cooperation for Space Standardization) E-40 standard for
writing software.
The fundamental principle of this Standard is the customer-supplier relationship, assumed for all software developments.

Basically there is planned in \Sec{timel} a series of meetings with well defined
aims and hopefully outcomes:
\begin{enumerate}
\item SRR system requirements review
\item PDR preliminary design review.
\item CDR critical design review.
\item QR qualification review.
\item AR acceptance review.
\end{enumerate}

The papers/documents needed for each meeting  are described in \Sec{docs}.

\section{Timeline}\label{sec:timel}

\subsection{System engineering processes related to software}

The system engineering
processes produce the information for input to system requirements review (SRR).
This establishes the functional and the performance requirements baseline (RB)
of the software development. \\
\emph{Customer says what she wants}


\subsection{Software requirements and architecture engineering process}
The software
requirements and architecture engineering process consists of the elaboration of
the technical specification (TS), which is the supplier's response to the requirements baseline.
This process can start in parallel or after the elaboration of the requirements baseline.
The software product tree is defined by this process.
The technical specification contains a precise and coherent
definition of functions and performances for all levels of the software to be developed.
The preliminary interface control document (ICD) is generated by this process.
During the software requirements and architecture engineering process, the result of all significant trade-offs, feasibility analyses, make-or-buy decisions
and supporting technical assessments are documented in a design justification file (DJF).

The software requirements and architecture engineering process is completed by
the preliminary design review (PDR).
The input to the PDR is the technical specification, preliminary ICD and the DJF. \\
\emph{Supplier agrees with customer  this is what we are going to do and why.}

\subsection{Software design and implementation engineering process}
This process does not start before the SRR.
It can start before the PDR, but it is after the PDR, when the results of
the requirements and architecture engineering process are reviewed and
baselined, that are used as inputs to the design and implementation engineering process.
The main output of this process is the design of the software items identified in the software product tree.
It is provided in response to the technical specification and the design
justification file.
All elements of the software design are documented in the design definition file (DDF).
The rationale for important design choices, and analysis and test data that
show that the design meets all requirements, is added to the DJF by this
process.
The results of this process are the input to the critical design review(CDR).
The CDR signals the end of the design activities. \\
\emph{Supplier agrees with customer  precisely what code we are going to write and why.}

Finally this process produces also the coding, unit testing and integration testing of the software product
All elements of the testing activities are documented in the design justification file (DJF). \\
\emph{Supplier writes code.}

\subsection{Software validation process}
The software validation process can start any time after the SRR.
This process is intended to confirm that the requirements baseline functions
and performances are correctly and completely implemented in the final
product.
The result of this process is included in the DJF.

This process includes a qualification review (QR) with the DJF as an input. \\
\emph{Agree a series of test cases that software will be able to do.}

\subsection{Software delivery and acceptance process}
The software delivery and
acceptance process can start after the CDR and when the software validation
activity with respect to the technical specification is complete.
This process includes an acceptance review (AR), with the DJF as input.
The acceptance review is a formal event in which the software product is
evaluated in its operational environment.
It is carried out after the software product is transferred to the customer
and installed on an operational basis.
Software validation activities terminate with the acceptance review. \\
\emph{Customer checks the software does the test cases.}

\subsection{Software operation process}
The operation process can start after
completion of the acceptance review of the software. \\
\emph{(Customer uses software in anger.)}


\subsection{Software maintenance process}
This separate process is started after the
completion of the AR but the process implementation is started before the QR.
This process is activated when the software product undergoes any modific
ation to code or associated documentation as a result of correcting an error,
a problem or implementing an improvement or adaptation.
The process ends with the retirement of the software product. \\
\emph{Keep a list of changes, bugs and fixes.}

\section{Documents}\label{sec:docs}

\subsection{Requirements baseline (RB)}
The RB expresses the customer's requirements.
The Interface requirements document (IRD) expresses the customer's interface requirements for the software
to be produced by the supplier.
This document is part of the requirements baseline.
\emph{This may be either taken as, or derived from, the customer's invitation to tender}

\subsection{Technical specification (TS)}
The TS contains the supplier's response to the
requirements baseline, and is the primary input to the PDR review process.
The Interface Control Document (ICD) is the supplier's response to the IRD, and is part of the TS.
\emph{This may be either taken as, or derived from, the supplier's proposal (Quality Plan)}

\subsection{Design definition file (DDF)}
The DDF is a supplier-generated file that
documents the result of the design engineering processes.
The DDF is the primary input to the CDR review process and it contains
all the documents called for by the design engineering requirements.
\emph{This describes the methods and algorithms to be used in software.}

\subsection{Design justification file (DJF)}
The DJF is generated and reviewed at all stages of the development and review processes.
It contains the documents that describe the trade-offs, design choice justifications, verification plan, validation plan, validation testing specification,
test procedures, test results, evaluations and any other documentation called for to justify the design of the supplier's product.
\emph{This describes why the methods and algorithms were chosen.}

\subsection{Management file (MGT)}
The MGT is a supplier generated file
that describes the management features of the software project (for instance,
organizational breakdown and responsibilities, work activities breakdown,
selected life cycle, deliveries, milestones and risks).
\emph{This was not required.}


\subsection{Maintenance file (MF)}
The MF is a maintainer generated file
that describes the planning and status of the maintenance, migration and retirement activities.
\emph{This was not required.}


\subsection{Operational documentation (OP)}
The standard is not easy to understand here , but seems to make the point that 
the user's experience of the software feeds back into the instructions
as to how to use the software.
\emph{This describes a test case for the particle advance.}
%The operation process is a system level activity, defined by the customer's requirements for the space system.
%The corresponding software engineering process is therefore not independent
%engineering activity, but is a support process at system level.
%Hence, the outputs of the process are contributions to system level outputs,
%and the outputs below are therefore either integrated with the software
%development documentation, or controlled and developed as part of a system documentation tree.


\end{document}
