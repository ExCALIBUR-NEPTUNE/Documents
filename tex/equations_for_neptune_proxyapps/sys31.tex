The neutrals are represented as super-particles that travel ballistically after introduction
and are lost when they strike the first wall.  In this context, neutral
particle may include photons.
Super-particles have label~$p$, weight~$w_p$ and sample the
point~$({\bf x}_p, {\bf v}_p)$ in 6-D position and velocity space.
At introduction, all particle quantities are defined by sampling
from specified probability distributions.

\subsection{Prerequisites}\label{sec:31prereq}
\subsubsection{Parameters}\label{sec:31param}
It is necessary to have parameters describing initial and boundary conditions.
There must be a means of tying boundary conditions to specific parts
of the surface and volume geometry. See discussion of definitions
of objects/classes for \nep \ in web pages.

\subsubsection{Random number generator}\label{sec:RNG}
It is generally important to test the properties of a random number generator
to ensure there is an absence of bias, typically by producing histograms of
the output and comparing with expected curves. If other routines are found to be unsuitable,
a technique based on the `Mersenne Twister' should give a satisfactory sequence
of pseudo-random numbers. 

For a range of applications where
functions or distributions do not vary on very small scales, Quasi-Monte Carlo
sampling~\cite{niederreiter} may be preferable to Monte Carlo. Note that although the place-name
Monte-Carlo has a hyphen, the name of the mathematical technique does not by convention.
For parallel computation, it will generally be best to compute a block of numbers at a time.

\subsubsection{Sampling from a specified distribution}\label{sec:specdist}
\paragraph{Generally}\label{specdistgen}
Textbooks such as Kalos and Whitlock~\cite{kaloswhitlock} (notable
for its treatment of radiation transport in \S\,6)  explain how to
generate samples of a given distribution $f(x)$ 
from random numbers uniformly sampled on the unit interval. Suppose that
$\xi$ is such a random number, then the corresponding value of $x$ is given by
solution of
\begin{equation}\label{eq:invdistbn}
\xi=1-\int_0^x f(x')dx'
\end{equation}
\Eq{invdistbn} may be solved explicitly for~$x$ in many important cases

\paragraph{Gaussian distribution}\label{sec:specdistgaussian}
This may be sampled using the Box-Muller method~\cite[\S\,3.1]{kaloswhitlock}.
Given two random numbers $\xi_1$ and $\xi_2$ uniformly sampled on the unit
interval, then two samples of a Gaussian distribution $f(x) \propto \exp(-x^2/2)$
are given by
\begin{eqnarray}\label{eq:gaussamp}
&&\sqrt{[-2(\ln(1-\xi_1)]} \cos 2\pi \xi_2 \\
&&\sqrt{[-2(\ln(1-\xi_2)]} \cos 2\pi \xi_1
\end{eqnarray}

\paragraph{Knudsen cosine}\label{sec:specdistknud}
Sample by accept-reject from the Knudsen cosine distribution below~\Eq{distknud} to launch
particle trajectories.

Let $f_{max}$ and $s_{max}$ be the maximum of the distribution function
in $[0,6v_{th,i,Kn})$ and the associated speed, respectively
where $v_{th,i,Kn}=\sqrt{2kT_{Kn}/m_{i}})$. Then provide
pairs of random numbers $R_{f}\in[0,f_{max}]$ and $v_{R}\in[0,6v_{th,i,Kn}]$
(and other components of velocity similarly sampled for the tangential
components (but with a negative velocity ranges)). Keep the particle
with normal speed $v_{R}$ if $R_{f}<f_{n,Kn}(v_{R})$.

\subsection{Sources and sinks of neutrals}\label{sec:31sources}
\subsubsection{Knudsen distribution}\label{sec:Knudsen}
As suggested in the TN-07 Neptune report by Parra, Barnes and Hardman~\cite{2047357-TN-07}
%(\url{https://github.com/ExCALIBUR-NEPTUNE/Documents/blob/main/reports/2047357/TN-07.pdf}
equations (5.9)-(5.13), the source of neutrals emitted from a wall
can be described by the Knudsen cosine distribution
\begin{equation}\label{eq:distknud}
f_{n,Kn}\left(\boldsymbol{v};\boldsymbol{n}\cdot\boldsymbol{v}>0\right)=\frac{3}{4\pi}\left(\frac{m_{i}}{kT_{Kn}}\right)^{2}\frac{\boldsymbol{n}\cdot\boldsymbol{v}}{\left|v\right|}\exp\left(-\frac{m_{i}v^{2}}{2kT_{Kn}}\right)
\end{equation}
where $\boldsymbol{n}$ is the normal to the wall, the Knudsen distribution
is used for outgoing neutrals for which $\boldsymbol{n}\cdot\boldsymbol{v}>0$,
and $kT_{Kn}$ is a parameter that controls the temperature
of the emitted neutral distribution. The Knudsen cosine distribution appears with a factor of
a particle flux in Equation~(5.9) of~\cite{2047357-TN-07}, so
despite the "$f$"~notation, it has different units to the other distributions~$f$.

%Propose that for simplicity we also use a Knudsen cosine distribution
%for the neutrals coming from a gas puff, if needed, at least as a
%starting point.

\subsubsection{Volumetric distribution}\label{sec:voldist}
As a starting point, the source of neutrals should be fixed
in time and given a prescribed spatial distribution around the edge
of the simulation domain. A point source, ie.\ delta function in space, 
would be a valid representation of a `gas valve'. Volumetric sources  consisting
of eg.\ a Maxwellian at a given temperature) could be considered for testing purposes.

\subsubsection{Recycling}\label{sec:recyc}
This cannot be properly treated without coupling to a plasma model,
hence is treated in  \Sec{33recyc}.

\subsubsection{Particle sink}\label{sec:volsink}
An explicit volume pumping region,
where neutrals are absorbed when they reach it. This might be
specified by a set of finite element identifiers~$e$.

\subsection{Boundary condition for neutral particles}\label{sec:bcs}
The particle simulation domain need not coincide with the simulation domain, 
since there may be finite elements where a species is treated as a fluid.
\subsubsection{Perfect Absorption}\label{sec:bcsperf}
Any particles that reach a domain boundary are deleted.
\subsubsection{Reflection}\label{sec:bcsrefl}
These conditions are very useful for testing eg.\ energy conservation. 
Perfect specular reflection might be needed to handle symmetry. Supposing
the unit surface normal is~$\boldsymbol{n}$, if the incident velocity is~$\bf{v}_p$,
then the reflected particle has velocity
\begin{equation}\label{eq:bcsreflvel}
\bf{v}_p=\bf{v}_p-2 (\bf{v}_p.\boldsymbol{n}) \boldsymbol{n}
\end{equation}
\subsubsection{Periodic boundaries}\label{sec:bcsperio}
These conditions are very useful for testing purposes, and required for full
or repeat sections of toroidal geometry. Particles simply leave one end
of the domain and re-enter at the other. In the case of a rectilinear
grid, periodic boundary conditions are easily applied by 
taking the modulus of the coordinate value with respect to the period length.
\subsubsection{Imperfect reflection}\label{sec:bcsimperf}
These conditions are probably most appropriate for photons.
This may be achieved by arranging that a fraction~$(1-R_p)$ of incident
particles are absorbed.  If the particles are allowed different weights, then
simply reduce the weight~$w_p$ of each reflected particle by the particle recycling factor~$R_p$
and its energy consistent with recycling factor~$R_E$.
Sputtering is the ejection of surface atoms by impact of both
energetic ions and neutrals. However, the rates of sputtering by
energetic neutrals are relatively low below~$100$\,eV ~\cite[\S\,9.7]{wesson} and may be neglected
in an initial investigation.
%Suggest making the choice in the first instance for ease of implementation.

%\subsection{Numerical algorithm\label{subsec:neutral-bc-numerics}\label}\label{sec:XX}


%\subsection{Numerical algorithm}\label{sec:XX}
%
%Already part of plasma fluid model. Future `explicit pumping' option
%would be covered by the boundary conditions for neutral particles,
%section~\ref{subsec:neutral-bc-numerics}\label.
