The following generic transport equation~\cite[\S\,1]{duderstadtmartin} applies to


all particle-based models for the time
evolution of the density distribution function~$f (x,{\bf v},t)$
\begin{equation} \label{eq:transport}
\frac{\partial f}{\partial t}
+ {\bf v} \cdot \frac{\partial f}{\partial {\bf x}}
+ {\bf a} \cdot \frac{\partial f}{\partial {\bf v}} 
= S_C(f)
=\left(\frac{\partial f}{\partial t}\right)_C + s_{exp}({\bf x},{\bf v},t)
\end{equation}
where ${\bf a} = d^2 {\bf x}/dt^2$ is the acceleration experienced by a particle
at position~${\bf x}$ with velocity~${\bf v}$. This represents scalar
advection in a 6-D space with an explicit source~$s_{exp}({\bf x}, {\bf v},t)$
and a source due to other inter-particle interactions that is conventionally written
$(\partial f/\partial t)_C$ when it is localised and usually depends linearly on~$f$.

Complete specification of the problem even for $S_C=0$ and a single species of particle
requires a force law such as that for particles of species~$\alpha$
with charge~$q_\alpha$ and mass~$m_\alpha$
\begin{equation}\label{eq:forcelaw}
{\bf F} = m_\alpha \frac{d^2 {\bf x}}{dt^2} = q_\alpha ({\bf E} + {\bf v} \times {\bf B})
\end{equation}
and equations for the evolution of the electromagnetic fields~${\bf E}({\bf x},t)$ 
and ${\bf B}({\bf x},t)$ such as Maxwell's equations, neglecting displacement current.
For neutral particles, when often ${\bf a}={\bf 0}$, interest attaches to $S_C$
which for 2-particle interactions is often the Boltzmann operator for 
different species $Q(f_\alpha, f_\beta)$ where $\alpha,\beta$ are species labels.

It will be seen that \Eq{transport} is a statement that the $7-D$ Lie derivative of~$f$
vanishes, and it is hoped that in the longer term, the properties of the Lie derivative 
might be exploited.

For current purposes, it is necessary to state a multi-species version of \Eq{transport}
complete with appropriate source terms to represent the physics thought critical for modelling the
tokamak edge. Inevitably choice of $S_C$ is a function of lengthscale and timescale.
On fast timescales in a strong electromagnetic field, the effect of collisions can
be ignored (collective effects are felt through say the electrostatic field).


