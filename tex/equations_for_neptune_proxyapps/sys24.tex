The following simple model is after Taitano~et~al~\cite{Ta13Deve,Ch11Ener}
\begin{eqnarray}
\frac{\partial f_e}{\partial t} +  v_{ex}\frac{\partial f_e}{\partial  x} +  \frac{q_e}{m_e} {\bf E} \cdot \frac{\partial f_e}{\partial  {\bf v}} &=& 0 \nonumber \\
\frac{\partial f_i}{\partial t} +  v_{ix}\frac{\partial f_i}{\partial  x} +  \frac{q_i}{m_i} {\bf E} \cdot \frac{\partial f_i}{\partial  {\bf v}} &=& 0  \label{eq:taitano3} \\
\epsilon_0 \frac{\partial {\bf E}}{\partial t} + \sum_\alpha q_\alpha n {\bf u}_\alpha - \overline{\sum_\alpha q_\alpha n {\bf u}_\alpha} &=&0
\nonumber
\end{eqnarray}
Equations~\ref{eq:taitano3} are the electron and ion Vlasov equations and 
Ampere's equation respectively.
The quantities $m_e$, $m_i$, $f_e$, $f_i$, ${\bf v}_e$, ${\bf v}_i$,
$q_e$, $q_i$, ${\bf E}$, $\epsilon_0$, and~$n{\bf u}_\alpha$ are the
electron and ion masses, electron and ion distribution functions, electron and ion
velocities, electron and ion charges, the electric
field, permeability constant of vacuum, and the momentum of species $\alpha=i,e$, respectively.
Note that \Eq{taitano3}represents a generalisation of the system in ref~\cite{Ta13Deve},
where for vector quantities, the $x$-component is always implied, in the usual notation
the original system is $1d1v$ rather than~$1d3v$ as above, where particles move
according to
\begin{eqnarray}
\frac{dx}{dt} &=& v_{\alpha x} \nonumber \\
\frac{d {\bf v}_\alpha}{dt} &=& \frac{q_\alpha}{m_\alpha} {\bf E} \label{eq:xvtaitano}
\end{eqnarray}
(Motion in $(y,z)$ is neglected, the 3-D electromagnetic version of \Eq{xvtaitano}
appears in \Sec{sys3-1}).

The $\overline{\sum \cdot}$ term denotes for example a spatially averaged summed
quantity and is included to enforce Galilean invariance. The solutions~$f_e$ and~$f_i$
of the Vlasov equations are functions of space variable~$x$, velocity~${\bf v}$, and time.
Ampere's equation is solved for the self-consistent electric field~${\bf E}$, which is a function of
space variable~$x$ and time.

The boundary conditions used are periodicity in~$x$, and zero at infinity in~$|{\bf v}|$.
Initial conditions which might be used for the distribution functions
are from ref~\cite{Ta13Deve},
\begin{eqnarray}
f_0(x,{\bf v},{\bf u}_0,T_0)&=& \frac{n_0(x)}{\sqrt{2\pi k T_0/m}}\exp\left(-\frac{m({\bf v}-{\bf u}_0)^2}{2kT_0}\right)  \label{eq:taitano3ics} \\
n_0(x)=n(t=0,x)&=&1 +\alpha_n \cos(k_w x ) \nonumber
\end{eqnarray}
where $n_0$, ${\bf u}_0$ and  $T_0$ are the 
initial number density, initial fluid velocity and initial temperature respectively. The
parameter~$\alpha_n$ is the perturbation amplitude, $k_w$ is its wave vector, and $m$ is the
species mass. It follows that the (scaled) momentum is given formally as
\begin{equation}\label{eq:mtmdefn}
n {\bf u}_\alpha(x) = \int {\bf v} f_\alpha (x,{\bf v},t) d{\bf v}
\end{equation}
where $f_\alpha$ is the distribution function for species~$\alpha$ at time~$t$.
(In practice the integral would be replaced by a sum over particles.)

\emph{Note} that periodic boundary conditions are of limited
value in practice, and attention should be given to minimal modifications
of the above  problem where there is
\begin{enumerate}
\item a flux of momentum across the domain (inflow
and outflow boundary conditions)
\item reflection of particles at the boundaries
\item a source of plasma within the domain, and outflow boundaries
\item and where the spatial dimension corresponds to arc length~$s$ along a fieldline
(implies $n$ replaced by $n/|{\bf B}|$, cf.\  \Sec{sys23plas}).
\end{enumerate}


