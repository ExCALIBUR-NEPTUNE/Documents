Rather than attempting to develop a fully 3-D Exascale targeted plasma edge (or 
boundary) code from day one, project \nep \   will first focus upon the 
development of ``\papp s"~\cite{proxies}, developed by partners across 
the project through a series of Grant calls. These must be designed and encoded 
to pave the way to the fully 3-D, actionable and performant \nep \   code (or 
codes) outlined in the Science Plan. As such, all \nep \   \papp s must 
capture the functionality and performance/scalability characteristics of the 
eventual infrastructure as much as possible. In addition, all of the solutions 
across the \nep \   programme must eventually be synergistic, leading to an  
integrated solution for the eventual code(s) -- this will require close 
cooperation through co-design across all partner organisations. The baseline 
\papp s for the initial years of the project are described briefly in the 
Science Plan~\cite{sciplan}, and expanded upon below to give the
``baseline plan" model equations, geometry and boundary conditions.
Note, the baseline plan does not preclude additional 
functionality that any bidder may deem useful (or even essential) to the 
project. Bidders are encouraged in their response to calls to be creative 
and ambitious and to describe their own ideas and plans for delivering above 
and beyond core scope,  provided the aim is 
to increase impact, quality, reduce risk and/or accelerate delivery (and that 
deliverables are fully aligned with the goals of the \nep \   Science Plan~\cite{sciplan}).

%For each \papp\ , bidders should describe their approach (including their 
%initial thoughts around provisional selection for algorithm and library 
%options), programming language preference etc. (which will be down selected in 
%negotiation with UKAEA) and should indicate how their solution will in the 
%eventual \nep \   code deliver:
%\begin{itemize}
%\end{itemize}
%\item[a)] Code sustainability.
%\item[b)] Easy to deploy, easy to use and easy to develop code.
%\item[c)] Performance (both strong and weak scaling).
%\item[d)] Performance portability (targeting multiple candidate Exascale 
%architectures).
%\item[e)] Actionable solutions (UQ).
%\item[f)] Quality, through rigorous Verification and Validation procedures.
%\item[g)] The ability to generate surrogate models derived from the code (eg.\ 
%through Model Order Reduction).

At baseline, \papp s target $x86$ (and ideally IBM POWER and ARM 
CPU) architectures (multi-core and multiple node) for scalability to first 
generation Exascale hardware. \Papp s might also target other 
Exascale candidate architectures (eg.\ GPGPU) and/or demonstrate a capability to 
explore the use of novel hardware as it becomes available to the \exc \   
project as part of the novel test-bed programme. In order to execute
efficiently on parallel architectures, \papp s are expected to examine
use of MPI, OpenMP or some other software technology
(ideally with a focus upon performance portability).

%If a bidder determines there 
%is insufficient resource to do so (or if they do not possess the relevant 
%skills), they should instead indicate that they are willing to work closely 
%with other \nep \   team members who have demonstrated skills in accelerator 
%exploitation.
%In order to target the chosen architecture(s), it will be necessary 
%indicate whether they intend to deploy MPI and OpenMP or some other 
%-- including a clear explanation as to why (and be prepared to negotiate with 
%UKAEA).

Supporting information regarding Braginskii's transport
coefficients for plasma in a strong magnetic field appears in \Sec{bragin}.
A description of sources of atomic and molecular radiation is given as an
annex in \Sec{atomic}.
