The following generic transport equation~\cite[\S\,1]{duderstadtmartin} applies to
all particle-based models for the time
evolution of the density distribution function~$f (x,{\bf v},t)$
\begin{equation} \label{eq:transport}
\frac{\partial f}{\partial t}
+ {\bf v} \cdot \frac{\partial f}{\partial {\bf x}}
+ {\bf a} \cdot \frac{\partial f}{\partial {\bf v}} 
= S_C(f)
=\left(\frac{\partial f}{\partial t}\right)_C + S_{exp}({\bf x},{\bf v},t)
\end{equation}
where ${\bf a} = d^2 {\bf x}/dt^2$ is the acceleration experienced by a particle
at position~${\bf x}$ with velocity~${\bf v}$. This represents scalar
advection in a 6-D space with an explicit source~$S_{exp}({\bf x}, {\bf v},t)$
and a source due to other inter-particle interactions that is conventionally written
$(\partial f/\partial t)_C$ when it is localised and usually depends linearly on~$f$.
Ultimately it will be necessary to solve a multi-species version of \Eq{transport}
complete with appropriate source terms to represent the physics thought critical for modelling the
tokamak edge.

Complete specification of the problem even for $S_C=0$ and a single species of particle
requires a force law such as that for particles of species~$\alpha$
with charge~$q_\alpha$ and mass~$m_\alpha$
\begin{equation}\label{eq:forcelaw}
{\bf F} = m_\alpha \frac{d^2 {\bf x}}{dt^2} = q_\alpha ({\bf E} + {\bf v} \times {\bf B})
\end{equation}
and equations for the evolution of the electromagnetic fields~${\bf E}({\bf x},t)$ 
and ${\bf B}({\bf x},t)$ such as Maxwell's equations, neglecting displacement current.
Inevitably choice of $S_C$ is a function of lengthscale and timescale.
On fast timescales in a strong electromagnetic field, the effect of collisions can
be ignored (collective effects are felt through the electromagnetic field),
when the Particle-in-Cell or PIC approach~\cite{hockneyeastwood} is effective.
Note that strictly local particle-particle interactions should be accounted for,
but these are expected to have negligible effect in a plasma (although not in
a gravitating system).

For neutral particles, when often ${\bf a}={\bf 0}$, interest attaches to $S_C$
which for 2-particle interactions is often the Boltzmann operator for 
different species $Q(f_\alpha, f_\beta)$ where $\alpha,\beta$ are species labels.

%It will be seen that \Eq{transport} is a statement that the $7-D$ Lie derivative of~$f$
%vanishes, and it is hoped that in the longer term, the properties of the Lie derivative 
%might be exploited

\subsection{Particle-in-Cell (PIC)}\label{sec:picdetails}
Although PIC codes are conceptually simple to implement, in practice there is often 
a problem with statistical effects, \emph{aka} noise. Noise is generally found to be
reduced when the scheme is momentum conserving, which is usually
achieved~\cite[\S\,5-3-3]{hockneyeastwood} by use of (1) the same function in both
charge assigment and interpolation of force onto particles, and (2) space-centred
approximations to derivatives. In the \nep\ symbols, assignment of charge to nodes 
and force interpolation to particles, share a weighting function~$W$, such that
\begin{eqnarray}\label{eq:Wfunction}
\Delta \rho_c({\bf x}_m)&=&\frac{q_\alpha}{V^e} W({\bf x}_p-{\bf x}_m) \\
{\bf F}({\bf x}_p)&=&=\Sigma_m q_\alpha W({\bf x}_p-{\bf x}_m) {\bf E}({\bf x}_m)
\end{eqnarray}
where ${\bf x}_p$ is the position of the particle, ${\bf x}_m$ is the location
of a finite element node. For consistency with the Nektar++ basis 
\begin{equation}\label{eq:basisagree}
W({\bf x})= \phi_{e,\xi}({\bf x})
\end{equation}


The fundamental import of (2) is that 
\begin{equation}\label{eq:forcefromcharge}
{\bf E} ({\bf x}_m) = \Sigma_m' V^e G({\bf x}_m, {\bf x}_m') \rho_c({\bf x}_m')
\end{equation}
where $G({\bf x}_m, {\bf x}_m')= -G({\bf x}_m', {\bf x}_m)$. The antisymmetry of~$G$
then ensures vanishing of the particle self-force and that the forces exerted by
one particle on another are equal and opposite, since eg.\
\begin{equation}\label{eq:forceiszero}
{\bf F}({\bf x}_p)=q_\alpha^2 \Sigma_m \Sigma_m' W({\bf x}_p-{\bf x}_m) V^e G({\bf x}_m, {\bf x}_m') 
W({\bf x}_p-{\bf x}_m')
\end{equation}
Only the electrostatic field~${\bf E}$ is shown in \Eq{Wfunction}
but the above analysis holds more generally for the Lorentz force.
