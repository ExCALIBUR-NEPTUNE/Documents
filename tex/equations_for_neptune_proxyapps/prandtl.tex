
\subsection{Prandtl Numbers}\label{sec:prandtl}
The above expressions (except for the resistivity) apply strictly only when there
are separate equations for ion and electron transport, 
so decisions have to be taken about how to combine the transport coefficients
to treat the plasma as a single fluid. For the thermal transport, since 
pressures $p_e\approx p_i$, it is sufficient to add the $\kappa_\alpha$. However, the values for ions and electrons
are so disparate because $m_p\gg m_e$ that one or other might be neglected, 
assuming~$B$ is of order unity (in Tesla) and $T_e\approx T_i$, 
thus $\kappa_{e\|} \gg \kappa_{i\|}$ and hence 
$\kappa_{\|}\approx \kappa_{e\|}$, since 
\begin{equation}\label{eq:rbrat}
\left(\frac{x_i}{x_e}\right)^2 = \frac{2m_e}{Z^2 A m_p} \left(\frac{T_i}{T_e}\right)^3
\end{equation}
It also follows that
\begin{equation}\label{eq:krat}
\frac{\kappa_{e\perp}}{\kappa_{i\perp}} = 0.078 \left(\frac{T_i}{T_e}\right)^{1/2}
\frac{1}{\sqrt{A}}
\end{equation}
thus $\kappa_{\perp}\approx \kappa_{i\perp}$.
There is the caveat that if $T_i$ is approximately spatially constant radially,
then $\kappa_{e\perp}$ might become relevant.

As for viscosity, since the ion momentum is so much greater than the electron momentum, then $\nu\approx\nu_i$.

For interchange motions where flows are perpendicular to the field, take 
$\kappa=\kappa_{i\perp}$, then on the \emph{Cambridge} definition, the magnetic Prandtl number is
\begin{equation}\label{eq:zeta}
\zeta=\frac{\eta}{\kappa_{i\perp}}=\frac{0.765}{\sqrt{2}}\frac{1}{\mu_0}\sqrt{\frac{m_e}{m_p}} \cdot
\frac{B^2}{Z N\sqrt{A}} \frac{(kT_i)^{1/2}}{(kT_e)^{3/2}}
\end{equation}
which evaluates as ($T_\alpha$ in~$eV$, $N$ in units of $10^{18}$\,m$^{-3}$)
\begin{equation}\label{eq:zetan}
\zeta=\frac{\eta}{\kappa_{i\perp}}=62\,700 \cdot \frac{B^2}{Z (N/10^{18})\sqrt{A}} \frac{(T_i)^{1/2}}{(T_e)^{3/2}}
\end{equation}
It may be argued that it is more appropriate to use the `anomalous'  value of $1\,m^2 s^{-1}$,
in which case \Eq{resisn} without units gives the `Cambridge'  magnetic Prandtl number.

The usual (viscous) Prandtl number is
\begin{equation}\label{eq:sigma}
Pr=\frac{\nu_{i\perp}}{\kappa_{i\perp}}=0.23
\end{equation}
Note that P.H.Roberts~\cite{roberts} defines the magnetic Prandtl number
as $Pr_M=\nu/\eta=Pr/\zeta$, and his definition is more widely used.

