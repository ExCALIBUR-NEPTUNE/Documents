\subsection{Fluid model}\label{sec:fluid1D}
For a multispecies plasma,
there is a system of Boltzmann equations to be solved, one for each species, each of form in
$3$ spatial dimensions
\begin{equation}\label{eq:mboltz}
\mathcal{L}_7 f_\alpha = \sum_\beta Q(f_\alpha, f_\beta) +s_\alpha
\end{equation}
where $\mathcal{L}_7$ is the  7-D Lie derivative (space, velocity-space and time make up the
$3+3+1=7$ dimensions, $\alpha,\beta$ are species labels and $Q$ is the Boltzmann collision operator.
The multispecies equations are derived following Grad~\cite[\S\,6]{zhdanov}
by substituting in \Eq{mboltz}
\begin{equation}\label{eq:ifaexpan}
f_\alpha=\exp(-\lambda H_\alpha) \mathcal{F}_\alpha (x,{\bf v},t)
\end{equation}
where the flow of the Lie derivative is given by the Hamiltonian~$H_\alpha$
and $\mathcal{F}_\alpha$ is a functional of moments of~$f_\alpha$, to include
(dropping the suffix on~$f$)
\begin{equation}\label{eq:moments}
n=\int f d{\bf v},\;\; {\bf u}_0 = \int f {\bf v} d{\bf v},\;\; 
T=\int f v^2/2 d{\bf v}
\end{equation}
The resulting system is linearised and solved by iteration to give the multispecies
plasma fluid equations in Zhdanov~\cite[\S\,6]{zhdanov}. There are believed to be
typographical errors in Zhdanov, so cross-checking is needed.

To see Grad's approach applied to classical fluids see
for example~\cite[\S\,8]{thompson}.

\subsection{Coupling to particles}\label{sec:coupart1D}
Other, less collisional species are to be treated as particles as in \Sec{sys2-4} and
coupled via~$s_\alpha$. Mathematical forms for $s_\alpha$ will be guided by 
the emerging results from particles' method research.
