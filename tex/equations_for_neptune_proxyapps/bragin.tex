Braginskii's transport coefficients are widely used in tokamak edge modelling.
Object-oriented Fortran code to compute the Braginskii coefficients is available
at \\
{\tt https://github.com/wayne-arter/smardda-misc.git}.

\subsection{General}\label{sec:general}
Note that $k_B T(\mbox{units of }K) = |e| T(\mbox{units of }eV)$ where $k_B$ is Boltzmann's constant and $|e|$ is the
absolute value of the charge on the electron. Otherwise suffix~$B$ denotes a quantity
from the Plasma Formulary~\cite{NRLpf07}. See also Braginskii's paper~\cite{Br65Tranwarv}.
The subsequent corrections by Epperlein and Haines, and by Mikhailovski and Tsypin are not relevant to this work.

In a magnetic field, the direction of which is given by unit vector~${\bf b}$,
Goedbloed and Poedts~\cite{goedbloedpoedts} define three auxilliary vectors for
a vector~${\bf v}$, viz.
\begin{equation}\label{eq:vaux}
{\bf v}_{\parallel}= {\bf b} ({\bf b} \cdot {\bf v}), \;\; {\bf v}_\wedge= {\bf b} \times {\bf v}
\;\;\mbox{and}\;\; {\bf v}_\perp=({\bf b} \times {\bf v} ) \times {\bf b}
\end{equation}
If ${\bf v} = (v_1, v_2, v_\parallel)$ and ${\bf b}$ is aligned with the 3-axis
in a Cartesian coordinate system, then
\begin{equation}\label{eq:vpt}
{\bf v}_{\parallel}= (0,0,v_\parallel), \;\; {\bf v}_\wedge= (-v_2,v_1,0)
\;\;\mbox{and}\;\; {\bf v}_\perp=(v_1,v_2,0)
\end{equation}
It may be shown that a tensor~$\mathcal{T}$ which is symmetric under rotation
about~${\bf b}$ has the form (in Cartesians) 
\begin{equation}\label{eq:tensor}
\mathcal{T}=
\begin{pmatrix} \mathcal{T}_\perp & - \mathcal{T}_\wedge & 0 \\
 \mathcal{T}_\wedge &  \mathcal{T}_\perp & 0 \\
 0 & 0 & \mathcal{T}_\parallel \\
\end{pmatrix}
\end{equation}
so that
\begin{equation}\label{eq:TV}
\mathcal{T} \cdot {\bf v} = \mathcal{T}_\parallel {\bf v}_\parallel + \mathcal{T}_\wedge {\bf v}_\wedge + \mathcal{T}_\perp {\bf v}_\perp
\end{equation}

\subsection{Conduction, Viscous and Resistive Coefficients}\label{sec:coefficient}
The electron parallel thermal conductivity in the Braginskii theory is given as~\cite{NRLpf07}
\begin{equation}\label{eq:kapbae}
\kappa^e_{B\|}= 3.2 \frac{Nk_BT_e}{m_e} \tau_e
\end{equation}
a formula valid in either cgs or SI units, where $\tau_e$ is 
the electron relaxation time (measured in seconds), defined below.
The notation is standard, with $N$ the number density of electrons, approximately 
the same as the number density of ions, $m_e$ the electron mass, and $T_\alpha,\;\;s=i,e$
the temperature of species~$\alpha$.
The perpendicular electron thermal conductivity satisfies
similarly
\begin{equation}\label{eq:kapberpe}
\kappa^e_{B\perp}= 4.7 \frac{Nk_BT_e}{m_e} \tau_e \cdot \frac{1}{(\omega_{ce}\tau_e)^2}
\end{equation}
where the electron cyclotron frequency
\begin{equation}\label{eq:cyce}
\omega_{ce}= \frac{e}{m_e}\cdot B
\end{equation}
Equivalent expressions for ions are
\begin{equation}\label{eq:kapbai}
\kappa^i_{B\|}= 3.9 \frac{Nk_BT_i}{m_i} \tau_i
\end{equation}
\begin{equation}\label{eq:kapberpi}
\kappa^i_{B\perp}= 2 \frac{Nk_BT_i}{m_i} \tau_i \cdot \frac{1}{(\omega_{ci}\tau_i)^2}
\end{equation}
where the ion cyclotron frequency
\begin{equation}\label{eq:cyci}
\omega_{ci}= \frac{ZeB}{m_i}= \frac{e}{m_p} \cdot \frac{ZB}{A}
\end{equation}
where $Z$ is the charge state of the ion and $A$ its atomic mass.
The definitions above have to be interpreted in the context of the
equations given in~\cite{NRLpf07}, so that thermal diffusivities are
obtained by dividing by~$3n_\alpha/2$ where $\alpha=i,e$ is the species index.
It is also convenient to introduce the dimensionless factors
\begin{equation}\label{eq:xe}
x_e= \omega_{ce}\tau_e
\end{equation}
\begin{equation}\label{eq:xi}
x_i= \omega_{ci}\tau_i
\end{equation}

Kinematic viscosities in the Braginskii theory may be taken as
\begin{equation}\label{eq:nuparae}
\nu^e_{\|}= 0.73 Nk_BT_e \tau_e /(N m_e) = 0.73 \frac{k_BT_e}{m_e} \tau_e
\end{equation}
\begin{equation}\label{eq:nuperpe}
\nu^e_{\perp}= 0.51 Nk_BT_e \tau_e /(N m_e) \frac{1}{x_e^2}= 0.51 \frac{k_BT_e}{m_e} \tau_e\frac{1}{x_e^2}
\end{equation}
\begin{equation}\label{eq:nuparai}
\nu^i_{\|}= 0.96 Nk_BT_i \tau_i /(N m_i) = 0.96 \frac{k_BT_i}{m_i} \tau_i
\end{equation}
\begin{equation}\label{eq:nuperpi}
\nu^i_{\perp}= 0.3 Nk_BT_i \tau_i  /(N m_i)\frac{1}{x_i^2} = 0.3 \frac{k_BT_i}{m_i} \tau_i\frac{1}{x_i^2}
\end{equation}


Key quantities in the calculation of all these terms are $\tau_\alpha$, $\alpha=i,e$.
The first step in their calculation is to convert their formulas,
usually given in cgs, to SI units, giving
\begin{equation}\label{eq:tauesi}
\tau_e=6 \sqrt{2\pi^3} \frac{\epsilon_0^2\sqrt{m_e}}{e^4} \frac{(k_BT_e)^{3/2}}{Z^2 N \Lambda}
=3.44 \times 10^{-7} \frac{(T_e)^{3/2}}{Z^2 (N/10^{18}) \Lambda}
\end{equation}
\begin{equation}\label{eq:tauisi}
\tau_i=12 \sqrt{\pi^3} \frac{\epsilon_0^2\sqrt{m_p}}{e^4} \frac{(k_BT_i)^{3/2} \sqrt{A}}{Z^4 N \Lambda}
=2.09 \times 10^{-5} \frac{(T_i)^{3/2} \sqrt{A}}{Z^4 (N/10^{18}) \Lambda}
\end{equation}
where the notation is standard, except possibly the use of $\Lambda$ for
the Coulomb logarithm. The above check with expressions in Wesson~\cite[\S\,14]{wesson}.
Note that $Z^2 \tau_i$ differs from~$\tau_e$ in being larger by  a factor of $\sqrt{2m_i/m_e}\approx 60 \sqrt{A}$
(also substituting $T_i$ for~$T_e$ is necessary). The factors in~$Z$ are taken from the
original Braginskii paper~\cite{Br65Tranwarv}.

It follows that the $x_\alpha$ factors may be conveniently written
\begin{equation}\label{eq:xen}
x_e =6.05 \times 10^{4} \frac{(T_e)^{3/2} B}{Z^2 (N/10^{18}) \Lambda}
\end{equation}
\begin{equation}\label{eq:xin}
x_i =1997 \frac{(T_i)^{3/2} B}{Z^3 (N/10^{18}) \sqrt{A} \Lambda}
\end{equation}
The large coefficients in \Eqs{xen}{xin} explain why classical transport is
so anisotropic.

Substituting the explicit expression for $\tau_e$ in \Eqs{kapbae}{kapbai} gives
respectively, the thermal parallel diffusivities are
\begin{equation}\label{eq:kappae}
\kappa_{e\|}= 13 \sqrt{2\pi^3} \frac{1}{\sqrt{m_e}}\frac{\epsilon_0^2}{e^4} \cdot
\frac{(k_BT_e)^{5/2}}{Z^2 N\Lambda}
\end{equation}
\begin{equation}\label{eq:kappai}
\kappa_{i\|}= 16 \sqrt{\pi^3} \frac{1}{\sqrt{m_p}}\frac{\epsilon_0^2}{e^4} \cdot
\frac{(k_BT_i)^{5/2}}{Z^4 N\Lambda \sqrt{A}}
\end{equation}
and the ratios are
\begin{equation}\label{eq:xee}
x_e=\frac {6 \sqrt{2\pi^3}\epsilon_0^2} {\sqrt{m_e}e^3} \cdot
\frac {(k_BT_e)^{3/2}B} {Z^2 N\Lambda}
\end{equation}
\begin{equation}\label{eq:xie}
x_i=\frac {12 \sqrt{\pi^3}\epsilon_0^2} {\sqrt{m_p}e^3} \cdot
\frac{(k_BT_i)^{3/2}B} {Z^3 N\Lambda \sqrt{A}}
\end{equation}
An expression for the perpendicular ion conductivity, maintaining
the fixed physical factors is of interest
\begin{equation}\label{eq:kperpie}
\kappa_{i\perp}=\frac{e^2\sqrt{m_p}}{9 \sqrt{\pi^3}\epsilon_0^2} \cdot
\frac{Z^2 N\Lambda \sqrt{A}}{(k_BT_i)^{1/2}B^2}
\end{equation}
Assuming $T_i$ is measured in~$eV$, and $N$ in units of $10^{18}$\,m$^{-3}$, then
\begin{equation}\label{eq:kperpin}
\kappa_{i\perp}=6.67 \times 10^{-4} \cdot
\frac{Z^2 (N/10^{18}) \Lambda \sqrt{A}}{(T_i)^{1/2}B^2}\;\;m^2 s^{-1}
\end{equation}
and
\begin{equation}\label{eq:kperpen}
\kappa_{e\perp}=5.26 \times 10^{-5} \cdot
\frac{Z^2 (N/10^{18}) \Lambda}{(T_e)^{1/2}B^2}\;\;m^2 s^{-1}
\end{equation}

The plasma resistivity is taken as
\begin{equation}\label{eq:resis}
\eta=\eta_B/\mu_0 =\frac{0.51\sqrt{m_e}e^2}{6 \sqrt{2\pi^3}\mu_0\epsilon_0^2} \cdot
\frac{Z \Lambda}{(k_BT_e)^{3/2}}
\end{equation}
Assuming $T_e$ is measured in~$eV$, then
\begin{equation}\label{eq:resisn}
\eta=\eta_B/\mu_0 =41.9\cdot\frac{Z \Lambda}{(T_e)^{3/2}}\;\;m^2 s^{-1}
\end{equation}


\subsection{Prandtl Numbers}\label{sec:prandtl}
The above expressions (except for the resistivity) apply strictly only when there
are separate equations for ion and electron transport, 
so decisions have to be taken about how to combine the transport coefficients
to treat the plasma as a single fluid. For the thermal transport, since 
pressures $p_e\approx p_i$, it is sufficient to add the $\kappa_\alpha$. However, the values for ions and electrons
are so disparate because $m_p\gg m_e$ that one or other might be neglected, 
assuming~$B$ is of order unity (in Tesla) and $T_e\approx T_i$, 
thus $\kappa_{e\|} \gg \kappa_{i\|}$ and hence 
$\kappa_{\|}\approx \kappa_{e\|}$, since 
\begin{equation}\label{eq:rbrat}
\left(\frac{x_i}{x_e}\right)^2 = \frac{2m_e}{Z^2 A m_p} \left(\frac{T_i}{T_e}\right)^3
\end{equation}
It also follows that
\begin{equation}\label{eq:krat}
\frac{\kappa_{e\perp}}{\kappa_{i\perp}} = 0.078 \left(\frac{T_i}{T_e}\right)^{1/2}
\frac{1}{\sqrt{A}}
\end{equation}
thus $\kappa_{\perp}\approx \kappa_{i\perp}$.
There is the caveat that if $T_i$ is approximately spatially constant radially,
then $\kappa_{e\perp}$ might become relevant.

As for viscosity, since the ion momentum is so much greater than the electron momentum, then $\nu\approx\nu^i$.

For interchange motions where flows are perpendicular to the field, take 
$\kappa=\kappa_{i\perp}$, then on the \emph{Cambridge} definition, the magnetic Prandtl number is
\begin{equation}\label{eq:zeta}
\zeta=\frac{\eta}{\kappa_{i\perp}}=\frac{0.765}{\sqrt{2}}\frac{1}{\mu_0}\sqrt{\frac{m_e}{m_p}} \cdot
\frac{B^2}{Z N\sqrt{A}} \frac{(k_BT_i)^{1/2}}{(k_BT_e)^{3/2}}
\end{equation}
which evaluates as ($T_\alpha$ in~$eV$, $N$ in units of $10^{18}$\,m$^{-3}$)
\begin{equation}\label{eq:zetan}
\zeta=\frac{\eta}{\kappa_{i\perp}}=62\,700 \cdot \frac{B^2}{Z (N/10^{18})\sqrt{A}} \frac{(T_i)^{1/2}}{(T_e)^{3/2}}
\end{equation}
It may be argued that it is more appropriate to use the `anomalous'  value of $1\,m^2 s^{-1}$,
in which case \Eq{resisn} without units gives the `Cambridge'  magnetic Prandtl number.

The usual Prandtl number is
\begin{equation}\label{eq:sigma}
\sigma=\frac{\nu_{i\perp}}{\kappa_{i\perp}}=0.23
\end{equation}
Note that P.H.Roberts~\cite{roberts} defines the magnetic Prandtl number
as $\nu/\eta=\sigma/\zeta$, and his definition is more widely used.

