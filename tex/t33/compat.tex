\hyphenation{gyro-period sub-models create hetero-geneous gyro-kinetic}% specify allowed hyphen insertions (e.g. prevents gy-roperiod, submod-els)

\subsubsection{ComPat}\label{sec:compat}

The ComPat project \cite{compatwebsite,compatswwebsite} is an extreme scaling
design pattern targeting the multiscale modelling of magnetic confinement
fusion (MCF) plasmas.  
Models of MCF plasmas span lengthscales from the electron gyroradius $\sim
10^{-6}$ m to the device size $\sim 10$ m, and timescales from the electron
gyroperiod $\sim 10^{-10}$ s to the confinement time $\sim 10$ s.
Including even smaller length- and timescales from atomic and sheath physics
would be desirable.  The different physical effects within an MCF plasma are
also characterized by models of different dimensionality, with five-dimensional
gyrokinetic turbulence, two- or three-dimensional edge transport, and
one-dimensional core transport models.

The ComPat project adopts a submodel approach to modelling MCF plasmas, using
single-scale submodels provided by different codes, and coupling these together
by passing relevant information in an agreed format through middleware.  
This approach has a number of benefits.  
It allows the separation of concerns: using existing codes as submodels means
that the components are developed by domain specialists, and are therefore
better tested and validated.  
Discrete submodels are also easier to maintain and optimize.  
Further, it allows users to change which code is used for each submodel, so
that a user could, for example, choose between continuum and particle-based
solvers for the gyrokinetic turbulence, or indeed, between gyrokinetic and
gyrofluid models.
The drawback of separating codes in this manner is that the process of coupling
codes introduces complexity and runtime overheads.

ComPat is a framework and software suite designed to facilitate studying
multiscale fusion plasmas on HPC systems.  
It has two major objectives \cite{Lu19ComP}:
(1) to provide a collection of methods and software to aid the development of
component-based multiscale simulations; and 
(2) to create generic transformation and optimization methods to ameliorate
coupling overheads and improve the simulation's runtime (and/or other
performance metric) on a targeted set of execution platforms.
Different workflow configurations may be used depending on what the user wishes
to optimize, for example, minimizing wall clock time, minimizing total energy
consumption, or maximizing scaling efficiency.


\paragraph{Implementation}

The ComPat software stack has several components.
Firstly, it uses a domain-specific language, the Multiscale Modelling Language
(MML) and its representation in XML (xMML) to allow developers to give a
high-level description of submodels and their interactions.
It uses the tool jMML to generate a topology, task-graph and skeleton
configuration file for the coupling framework.
The coupling framework used is MUSCLE2 (Multiscale Coupling Library and
Environment).
This itself has two parts, a library for data exchange between submodels, and a
runtime environment to manage each submodel's execution on (possibly)
distributed resources.
The library has APIs in C, C++, Fortran, Python and Java, with functions
allowing users to query parameters, send and receive data, and log and stage
files.
This component-based approach means that the framework is agnostic towards
which submodels are actually used.
Submodels (called \emph{kernels} in MUSCLE2 nomenclature) therefore may be
swapped in and out, and different multiscale models treated within the same
framework.

ComPat uses QCG middleware \cite{qcgwebsite}, as a resource broker to set-up
and execute jobs.
The QCG client provides a simple interface to machines through batch scripts or
XML files.
This allows cross-cluster jobs to be configured with a single script and to be
monitored from a single workstation.

\paragraph{Performance}

Alowayyed \emph{et al.}\ \cite{Al17Mult} identify three Multiscale Computing Patterns (MCPs):
{\it Heterogeneous Multiscale Computing} (HMC),
{\it Replica Computing} (RC) And
{\it Extreme Scaling} (ES).
ComPat follows the {\it Extreme Scaling Pattern}, as the computing cost of one
submodel (called the \emph{primary}) dominates that of all other submodels
(called the \emph{auxiliaries}).
In ComPat, the primary is the model for microscale turbulence.
Through its use of multiple computing resources, the extreme pattern introduces
an extra layer of parallelism at the code coupling level, meaning that coupling
overheads may be hidden, and the coupled code may even outperform a monolithic
implementation.

In Luk \emph{et al.}\ \cite{Lu19ComP}, a further optimization is found by solving
the submodels in an ``asynchronous'' fashion.
A simple first analysis of the data dependencies in ComPat showed that the
different models needed to be executed sequentially, and that consequently the
primary microturbulence submodel was often idle, waiting for other submodels to
finish.
However, by considering the timescales of the auxiliary submodels, Luk \emph{et
al.}\ argue that the longer-timescale equilibrium and transport solvers do not
change their input to the turbulence solver significantly from one time step to
the next.
Therefore, they introduce an "asynchronous" mode, where the microturbulence
model takes equilibrium and transport data from the \emph{previous} timestep. 
This removes the data dependency, and allows the submodels to run in parallel.
This asynchronous workflow decreases the total wall clock time by around $6\%$
without appreciably changing the simulation results.

Despite this improvement, the runtime is still limited by the scaling
performance of microturbulence submodel, which only scales well to 16 cores per
flux tube.
This emphasises that in the {\it Extreme Scaling Pattern} it is necessary to optimize
both the workflow and the raw scaling performance of the submodels (that are on
the critical path) in order to obtain good scaling performance at Exascale.
