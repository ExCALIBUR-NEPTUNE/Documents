\subsection{Software design patterns for Call 8 Proxyapps}\label{sec:call8}

%Description of proxyapps

The Contract Ref.\ T/NA085/20 focusses on the development of a referent
gyrokinetics model.
This work package will investigate a novel approach to deriving the gyrokinetic
equations, where the local three-dimensional fluid quantities (density,
momentum and energy) are embedded into the five-dimensional particle
distribution function.
This leads to a version of gyrokinetics with both a kinetic equation for the
modified distribution function and a set of fluid equations for the local
moments.
This allows one to use a single model to study both the core (where one solves
the whole equation system) and for the scrape-off layer (where one solves 
only the fluid equations).
Alternatively, the model allows a very natural coupling of the gyrokinetic
equations to any set of fluid equations for the scrape-off layer.

Because of the novelty of this approach, the work package will first derive
drift kinetic equations (the simplified, long-perpendicular wavelength
limit of gyrokinetics).
It will also develop proxyapps to compare the numerical performance of the new
model to a more standard drift kinetic model for a variety of problems.
It will determine potential bottlenecks in the numerical treatment of the drift
kinetic models, and, in addition, develop appropriate numerical algorithms for
treating the wall boundary.

%Context of proxyapp

The proxyapp will be developed from scratch using Julia, which combines rapid
prototyping in a high-level language and access to computational libraries,
with performance that approaches that of statically-typed languages. 
While there is no design pattern yet specified for the proxyapp, the developers
are also early contributors to the gyrokinetics code GS2 \cite{GS2}.
It therefore seems likely that the proxyapp will share design features with
GS2.

GS2 is written in Fortran 90, using a modular design pattern (see section
\ref{sec:modular}).
It utilizes lazy initialization (see section \ref{sec:lazy_init}) as a
technique to passively manage module dependencies.
The modular design is a result of the age of the code.
Object-oriented techniques have been introduced, but in an incremental
fashion that allows modular design to coexist with encapsulated objects.
To achieve this, dependency injection (section \ref{sec:dependency_injection})
was adopted, with module-level variables being phased out in favour of large
\texttt{state} objects.
Having a small number of objects with many members has proven convenient for developers,
and it seems likely this proxyapp will retain this approach from GS2,
with perhaps a few \texttt{state} objects representing the solution, the diagnostic outputs
and the simulation parameters.

%%%GS2 is written in Fortran 90, using a modular design pattern.
%%%It utilizes lazy initialization -- rather than handling dependencies between modules explicitly,
%%%each module has an initialization subroutine \texttt{init} which calls the \texttt{init} routine of all the module's dependencies before initializing itself.
%%%A similar approach is taken to uninitialization.
%%%In keeping with the modular design, GS2 makes heavy use of module-level
%%%variables for objects like the solution arrays and grid resolutions.
%%%These variables may be read and modified by all subroutines within the module;
%%%when given the \texttt{public} attribute, such variables may be edited by \emph{all} modules.
%%%The module-level variables arise in legacy code, and the newer code
%%%incorporates object-oriented designs.  
%%%However, this change was made incrementally, with the object-oriented code
%%%coexisting with the module-level variables.
%%%Therefore, the object-oriented approach uses a dependency injection design pattern, 
%%%where so far as possible, objects that were previous module-level variables
%%%are instead wrapped into a large \texttt{state} object.
%%%The \texttt{state} is then the input/output of subroutines, allowing the
%%%module-level variables to be incrementally removed.
%%%Having a small number of objects with many members has proven convenient for developers,
%%%and it seems likely this proxyapp will retain this approach from GS2,
%%%with perhaps a few \texttt{state} objects representing the solution, the diagnostic outputs
%%%and the simulation parameters.
