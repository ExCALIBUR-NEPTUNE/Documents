This report has presented a subset of known object-oriented software design
patterns that are expected to be important in \nep\ \papp s as currently
envisaged, using facts derived from existing codebases on which the
\papp s will draw.  
It is of note at this point that the progenitor frameworks fall into two
categories: modern, C++-based approaches,
and procedural Fortran codes.  
For the former, the salient design patterns are principally those that enable
the reuse of components in complex, modular codebases: the Template Method,
Abstract Factory, Strategy, and Flyweight patterns from the Gang of Four
\cite{gammahelmjohnsonvlissides}.
As for the latter, the potential for extensive code re-use is evidenced in 
existing codes, stemming primarily from the use of modular design.
The \F{SMARDDA} framework described in this report is a concrete example of 
a Fortran code constructed by the aggregation of smaller objects, 
in a similar manner to that anticipated for \nep.
Here, the use of the layered architectural pattern involves a shared subroutine library
for common functionality e.g. error logging, supporting re-use.  
Higher in the framework, the fact that the modules are agnostic to each others' 
existence means that the various physics models combine in a manifestation of 
the puppeteer pattern, though some limitations of the current code 
(e.g. the existence of a single central control file) will need to be overcome 
in \nep, perhaps with the aid of a graph-based dependency mapping approach. 

A further section has surveyed overarching patterns, though this
section stops short of prescribing an overall integration framework.  
There is thus scope for exploration of, for example, optimal coupling using
prototype versions of the initial \papp s, with a view to address issues
encountered in earlier multiphysics frameworks \cite{y2re332, compatwebsite,
vecmawebsite}.
Again, certain of the time-honoured Gang of Four design patterns, for example,
Adapter, are of use.

Further design inspiration may be gained by study of the game
engine structure in the first annex.
The second  annex has presented a set of guidelines for a series of \papp s for
benchmarking exascale computing platforms, designed by the US Department of
Energy's Exascale Computing Project.  
This provides an evolving set of quality standards and best practices which
might usefully be considered during \nep\ \papp\ development, as well as
possibly providing a more formal basis for future development within ExCALIBUR.
