\section{Morning}

The meeting was opened by Rob Akers and the agenda was shown by Wayne Arter:

\begin{verbatim}
Wayne Arter, UKAEA    9.05         Introduction
Ben Dudson,  York     9.10         Plasma fluid referent model via exploratory Proxyapps
Steven Wright, York   9.35         Investigate DSL and code generation techniques
Dave Moxey, Exeter    9.50-10.05   Performance of Spectral Elements
           ---Break---
Felix Parra, Oxford   10.25        Referent model for plasma edge region
Peter Coveney, UCL    10.45        Study of Uncertainty Quantification (UQ) techniques
Sue Thorne,  STFC     11.00        Investigate matrix-preconditioning techniques
Ben McMillan, Warwick 11.15-11.30  Optimal Use of Particles
        ---Short break---
Discussion            11.40
Marta Barrabino       12.00-12.15  Your admin/financial questions answered
        ---Lunch Break---
13.30  Anisotropic transport/elliptic solvers. 
   Patrick Farrell, Oxford
   Chris Cantwell/Spencer Sherwin, Imperial
14.30 Surrogate models for data compression/turbulence 
   Ed Threlfall, UKAEA
   Tim Dodwell, Exeter
   Ben McMillan, Warwick 
15.30-16.00  Discussion and Close
\end{verbatim}

\subsection{PO 2047356: Plasma referent model via exploratory \papp s, Ben~Dudson / Peter Hill (York)}

Ben~Dudson showed a slide with his timeline; the current work is 2D / 3D 
elliptic solver and test cases and 1D fluid solver.

Environment set-up has been done, including a Slack channel and a github 
repository which now contains initial \papp s MinEPOCH and Nektar-Driftwave 
(the latter is \F{Nektar++} with Hasegawa-Wakatani equations).  Documentation 
is on ReadTheDocs (it uses reStructuredText from github repo).

Peter Hill described the initial plans for performance testing framework to 
target FLOPS, memory performance and networking.  The plan is to track 
performance progression rather than to do detailed profiling i.e. be able to 
detect if changes to the code result in performance loss (e.g. spoiling cache 
locality).  Framework is initially a github repo but the intention is to do a 
database.  Output is to be a structured text file format, probably JSON.  There will be 
automated tests as well as ad-hoc manual capability.  This is currently being 
scoped, with target delivery in c.6 months.

Ben~Dudson showed work for 1.2 elliptic solver, treating vorticity equation on a `2D 
helix' (i.e. excluding direction parallel to magnetic field; for the shape 
think of the Riemann surface of the logarithm).  The shape is a space-filling 
curve as it does not close on itself and the equation is basically $\nabla^2 
\Phi = \Omega$.  New Hypre implementation (GPU), tested on Lassen (LLNL); $10^6 
- 10^7$ sized grids. There was a clarification that slides focussed on operator inversion - 
`equation' was a typo. One simple test is that operator compounded with  inverse returns 
identity. He showed some GPU tests e.g. 4 GPU (NVIDIA V100) executes faster 
than 40 CPU cores on node-for-node comparison.  A code profile showed a large 
speed-up in the solve step but that the set-up and data copy stages were slow 
for the GPU - these parts scale badly.

Ben~Dudson showed 2.1 1D fluid model, based on SD1D and Hermes-3 (the latter developed partly for
STEP).  Core plasma on the left, first wall on right and region of plasma / 
neutrals recycling near the wall with atomic reactions there.  Component tests 
included conservation of matter and comparison to analytic solution for 
nonlinear heat model.  He showed output of test case with outflow boundary 
condition on one side and no-flow on the other.  Much scope for introducing 
complexity e.g. separate electron and ion temperatures, multiple ion species, 
reactions and collisions.

There was a question from Felix Parra about whether equation for elliptic 
solver $\nabla^2 \Phi = \Omega$ was actually the correct physics: sometimes 
it's OK e.g. ballooning modes but in other cases the potential may actually be 
determined by other physics along magnetic field lines.  There was discussion 
about extending the method to full 3D, allows bigger timestep but Ben~Dudson 
mentioned problems with Ohm's Law in this case.  Felix Parra mentioned a paper 
using the COGENT code trying to do this in 3D that had confirmed the difficulty 
of the problem.

\subsection{PO 2047358: Investigate DSL and code generation techniques, Steven Wright (York)}

Stephen W showed the timeline for this 9 month project.  There is one 
deliverable for this FY: D1.1 a report on the state of current and future (5-10 
year) hardware and techniques for evaluating hardware, programming models and 
code portability, which he summarized.  Current and forthcoming pre / post 
Exascale systems are all heterogeneous (except Fugaku, which is now getting 0.5 
ExaFLOPS).  Near-future GPU tech is NVIDIA (Lovelace / Einstein / Hopper?), AMD 
Instinct, and Intel Xe.  US DOE has four forthcoming supercomputers, all 
SlingShot fabric Cray Shasta systems; Europe situation is similar.  UK Archer 2 
is slower (28 PetaFLOPS), many Tier-2 systems inc Isambard 2.  Report also 
summarizes pros and cons of OpenMP, RAJA, SYCL etc.

D1.2 DSLs.  These can be low-level e.g. OP2 or high-level like UFL or \F{BOUT++}. 
Optimized data representation and movement: data managers are Kokkos 
Views, UMPIRE, Intel SIMD Data Layout Template~(SDLT)
(goal is hardware-agnostic data configurations).

D1.3 data representation and I/O.  Mentioned formats (HDF5, netCDF) and 
I/O acceleration libraries (TyphonIO, SILO, ADIOS).  York are looking for apps 
to test these.  New approaches include NVMe, Veloc, Rabbit, burst buffers.  
Noted that rewriting existing codes nasty so goal is \emph{new} future-proofed codes, 
agile / flexible, with hardware-agnostic data management and a user interface 
and/or DSL for scientists.

Task 2 testbeds and representative \papp\ s.  Noting that \F{Nektar++},
\F{BOUT++} and {\it EPOCH} likely too large to assess portability, the goal is to 
perform this analysis using \papp\ s.


Rob Akers wanted to determine the extent to which \nep\  partners are aware of 
cross-cutting ExCALIBUR themes e.g. DSL.  Gihan Mudalidge replied that he is on 
two of the relevant workgroups: Gen~X and turbulent flow simulations.
Low-level OP2 / OPS were developed at Warwick.  Wayne Arter commented 
that there is an upcoming \nep\  discussion session on DSLs to which many of 
the partners want to contribute.  Rob mentioned that there is a Cambridge / 
Intel / Dell `openlab' group from which partners might obtain assistance (e.g. 
for OneAPI).

\subsection{PO 2048465: Performance of spectral elements, David Moxey (Exeter)}

Task 1: High-order mesh generation

David M made the point that high-order elements need high-order representations 
of the geometry (nomenclature: isoparametric means geometry represented to same 
polynomial order as used in spectral elements; there are also sub- and 
super-parametric - standard definition of isoparameteric is that geometry is represented
with the same basis functions as the fields).
Accurate surfaces representation is specifically D1.3.  He 
explained the need to bend element edges / faces and spread out internal nodal 
points and showed a strain-energy minimization scheme used in {\it NekMesh} 
(using a BFGS optimization strategy), an incremental process that deforms 
linear mesh, optimizes edge nodes then optimizes internal nodes.  Figures of 
merit for surface meshes include accuracy of normal approximation and 
evaluation of $L_2$~distance from the CAD surface.  Exeter are working with 
sample geometries provided by UKAEA as well as other known challenging examples.

Looking ahead to D2.1 (anisotropic heat transport solver, steady and 
time-dependent cases) David M has recruited two Imperial College students.  
D1.2 is a quadrilateral-based cross-field solver.  For D3.2 he has solicited 
interest in aspects of \F{Nektar++} from \nep\ partners.


Discussion on meshing: Rob Akers asked if it was worth UKAEA collating CAD 
representations e.g. MAST, ITER first walls during the coming months.  David M 
reported that he has already received some `clean' CAD models of JET / MAST / 
ITER from Wayne Arter.  Cleaning up CAD was identified as a bottleneck; David M 
plans to continue working with the developers of {\it CADfix} software which 
does precisely this.  Wayne Arter said he was encouraging interaction with
the \F{CADfix} people.

Discussion on particles: Rob Akers mentioned the problem of `losing' particles 
when tracking them in a mesh (e.g. in neutronics) if they traverse a vertex.  
David M mentioned that there is a prototype particle-tracking branch of \F{Nektar++}
implemented by a student; it works but is not efficient.  Rob worries 
about scaling behaviour of particle tracking if there are e.g. billions of 
particles.  David M concurred that this is a problem.

After this talk there was a question from Chris Cantwell about allowing 
his student to join the afternoon sessions: it was decided that it was fine for 
affiliated parties to join in the sessions.  Wayne Arter also raised the 
question of how widely David M's \F{Nektar++} training might be publicized.

\subsection{PO 2047357: Referent model for plasma edge region, Felix Parra (Oxford)}

Gyrokinetic-averaged kinetic referent model (work done with Michael Barnes and 
Michael H. Hardman).

3 Milestones, M1.1 1D equations.  Felix P presented a final set of parallel 
dynamics equations defined on the 1D subspaces along the magnetic field lines.  
The moment approach for the system has been deduced and a linear test for 
checking the implementation has been derived.  The main problem appears to be 
solving for the potential (hence the moment approach).

M2.1 1D code with periodic boundary conditions - this is now on the git repo.  
It has the option to use finite differences or Chebyshev polynomials and it 
handles ions, electrons, and neutrals.  It has been validated against an 
analytic linear model.  Felix P showed computational  results as
wave dispersion relations, which exhibited a novel direct (non-propagating)  mode associated 
with the presence of neutrals  in the plasma.

M1.2 report on plasma edge model is to be delivered this week.  It reviews 
problems needing kinetic treatment and a systematic treatment of the issue of 
which particular finite Larmor radius effects need be retained and when; also 
the matters of including neutrals into kinetic codes and boundary conditions at 
the wall.

Rob Akers commented on the pleasing progress and it was noted that the 
treatment of neutrals in 1D is new physics.  Rob is interested in 
collisionally-coupled massive impurities.

There was a discussion of the non-propagating mode, which was once thought 
unimportant for simulations but now is believed to be relevant for turbulence 
and how turbulence cascades; it will be present in 2D and 3D simulations also.  
Rob Akers asked whether it was possible to measure this mode experimentally.  
Wayne Arter mentioned he and Patrick Farrell will soon start supervising a PhD
student to investigate problems of this type (systems with an time-varying mode and a `DC' mode)
as they transition into the nonlinear regime.
  


\subsection{PO 2047352\_1: Study of UQ techniques, Maxime Vassaux (UCL) on behalf of Peter Coveney}

Study of Uncertainty Quantification (UQ) techniques for \nep.

Activity 1: Report on UQ.  Part 1 concise recommendations for UQ - done.  Part 
2 architecture of UQ workflows for June; they are working on active subspaces and 
semi-intrusive UQ for a chaotic (molecular dynamics) code.

Activity 2: Workshops.  Part 3 EasyVVUQ workshop was done (18 January) and was 
followed by a VECMAtk hackathon (19-22 June) in which it was noted that the 
\F{BOUT++} developers participated.  Part 4, a second workshop including an 
explanation of recommended UQ methods is scheduled to occur in early July.  
There will be another hackathon also (21-23 April).

High-throughput UQ reporting: the focus is on non-intrusive UQ (so as to avoid 
human-intensive activities e.g. code rewriting).  Enhanced sampling techniques 
were mentioned, including Monte-Carlo for high-dimensional parameter spaces, 
adaptive sparse grid and stochastic collocation.  Active subspace methods 
rotate to the strongest variability directions.

Semi-intrusive methods are for encapsulated coupled models.  Maxime V showed an 
example from materials modelling with a cheap macromodel and an expensive 
(molecular dynamics) micromodel; the latter was replaced with a surrogate.  The 
example used MUSCLE3.

The conclusions from the first hackathon were presented. A questionnaire
indicated that there were a wide range (six) of 
single codes and one coupled model.  It was mentioned that one of the codes was 
\F{Nektar++}.

Rob Akers wanted to make sure that the UQ group were getting what they needed 
from other partners.  Maxime V said that this was in progress; specifically, he 
has started interacting with Serge Guillas' group on surrogates.  Peter Coveney 
said partners should actively request help from his group.  Maxime V mentioned 
that there is a UQ-specific channel on the Slack.  Peter Coveney advised that 
Maxime V will be collaborating with Deyu Ming now that the latter is active.  
Rob Akers mentioned there is a joint UKAEA PhD student with Serge G and also an 
additional Met Office PhD.  Rob Akers said all partners should think about 
attracting PhD students to \nep\  in order to expand the project. The next UKAEA
round however will be geared towards students' starting in Autumn 2022.

Peter Challenor was concerned that there was no Gaussian Process content in the 
non-intrusive UQ, as this approach is more scalable than e.g. stochastic 
collocation.  Peter Challenor said this was supposed to come from Serge G.  
Maxime V asked Peter Challenor whether GP can be used for adaptive sampling; 
Peter Challenor replied that he has just submitted a paper and it works better 
than MICE.  `Sequential Design' for computer experiments was mentioned and Rob 
Akers agreed that this is needed for STEP; Rob Akers is very interested in 
opensource approaches and Peter Challenor is going to send him a preprint.


\subsection{PO 2047353: Investigate matrix preconditioning techniques, Sue Thorne (STFC)}

Key PDEs have been identified.  There was a presentation last week to \nep\  
partners by Anton Lebedev of STFC.

Deliverables: Literature surveys 1.1 elliptic problems, 2.1 hyperbolic, and 3.1 
sparse approximate inverses.  These are contained in one report and were 
discussed in Anton's talk.  4.1 implicit factorization preconditioners to 
include their proposed preconditioners and theoretical convergence results.
Work started early  on  D6.1-6.4 comparison report and D5.1-5.2 prototype code.

Sue T thanked Ben~McMillan for supplying equations: elliptic ones are simplified 
Grad-Shafronov and non-Boussinesq vorticity; hyperbolics 1D fluid solver and 2D 
Hasegawa-Wakatani.

Sue T discussed constraint-style preconditioners, and how STFC have extended 
the work of Keller, Gould, Wathan (2001) and Dollar, Gould, Schilders, Wathen 
(2006) to the non-symmetric case.  (Note that Dollar is Sue's former surname.)
D4.1 non-symmetric extension gives new 
(publishable) theory results, applicable beyond \nep.

\subsection{PO 2047355: Optimal use of particles, Ben~McMillan (Warwick)}

Ben~McMillan is focussing on Lagrangian particle approach (numerically favoured when 
velocity space is 3D, distribution function is multi-component or there is a 
non-optimal spatial grid; there is no Courant limit.  There is however a 
timestep limit from the explicit field solver (field moves faster than 
particles), also existing techniques are noisy and do not integrate well with 
fluid models.  For these reasons, Ben~McMillan believes massive improvements are 
possible for the simulation of particles in the plasma edge.

Milestones: M4 \papp\  with advanced particle orbit solver, M5 low-noise 
control variates implementation, M6 report on co-design.  Also M3 implicit 
methods (for field solve) is underway.

Gyrokinetic PIC: standard technique, done in ORB5.  Basic drift-kinetic / 
currents added to {\it EPOCH} \papp.  The aim is to provide a pathway to 
advanced electromagnetic algorithms and mixed schemes.  Minor improvements 
include some refactoring and adding particle sub-stepping in anticipation of 
the implicit field solver.

Connection with fluid methods: control-variates to reduce noise: this approach 
involves using a background distribution plus a perturbation.  It works well in 
the core, usually referred to as~$\delta f$, and is being adapted for the edge,
e.g. to allow large fluctuations, 
to make sure the no-fluctuation limit recovers the fluid behaviour (e.g. 
setting the first three moments of the perturbation to zero).

Ben~McMillan showed a test of the {\it EPOCH} \papp\ ; an evolving collisionless 
particle distribution.  It exhibited simple 1D plasma bunching.  Control 
variates can reduce the Monte Carlo variance by a factor of c.100.

Rob Akers asked if the density stays strictly positive; Ben~McMillan replied yes under 
certain assumptions but there are potential issues here.

M6 co-design with spectral / hp elements: Ben~McMillan has done some work on particle 
tracing in Hasegawa-Wakatani and seed particles in Nektar-driftwave.  This was 
on a regular quad grid; he proposes distorting the mesh to test particle 
tracing.  There are unanswered questions here so there will need to be dialogue 
with \F{Nektar++} developers.

Rob Akers expressed again his concern re.\ the issue of losing particles (this 
issue needs to be fixed, in a way that is scaleable).  He asked Ben~McMillan whether 
he was getting everything he needs; Ben~McMillan replied he would like a bit of steer 
on his next step e.g. algorithm in the \papp\  and curvilinear elements.  Rob 
replied that Ben's remit was to provide this steer himself, and also 
recommended talking to others in the particle community e.g. Simon 
McIntosh-Smith re.\ the latter's work in collaboration with Aldermaston on 
co-design for upcoming architectures.  Rob made clear that a continuing grant 
would involve particles on curvilinear meshes at the Exascale, and coupling to 
spectral / hp - to be co-designed, rather than attempting to fuse existing 
codes.

\subsection{Answers to admin and financial questions, Marta Barrabino, UKAEA}

A general discussion preceded Marta B's arrival.

Wayne mentioned problems getting a contract for Peter Challenor / Tim Dodwell
under PO 2048906.

Rob asked whether everyone was happy with the communication aspects of the 
project e.g. Slack, email.  Spencer Sherwin did not receive the agenda for this 
meeting.  Rob mentioned that the forthcoming ExCALIBUR website could have an 
email sign-up button that could work in a `granular' way i.e. sign up to 
various ExCALIBUR themes.

Spencer S mentioned the challenge of getting new starters up to speed on 
\nep\  activities.  It emerged that, in order to mitigate this, there might be 
a small amount of flexibility in Met Office deliverable deadlines.  There is 
also the possibility of more informal meetings for novices; Wayne Arter 
mentioned that dividing the group into smaller parts (e.g. divide by 3) made 
these more accessible.

Rob Akers asked whether the invoicing process was clear (and noted that some 
invoices had incorrectly been sent directly to him).  Peter Coveney noted the 
difficulties faced by finance departments at this time of year, proximity to 
Easter, and the overlap with other proposals.  Rob emphasized that we need to 
satisfy the Met Office that the deliverables have been done, by 20 March.  
Outstanding invoices need to have been received by then or there is the 
possibility of having to refund money to BEIS.  It was clear that some PIs were 
not certain whether their invoices had been raised or not due to bureaucratic 
opacity.  Rob emphasized that the optimal procedure is to raise the invoice at 
the same time as submitting the deliverable.

Marta B joined the discussion at this point.
% It emerged that UKAEA could invoice the Met Office even without invoices (!).

Rob Akers reminded \nep\ grantees that they were obliged to produce technical reports as
promised in their bids. The slides from their talks would interim evidence of 
work done and would act as substitutes for the  quarterly reports
due in the current month. \\
Payment timescale: Rob approves invoices once 
received then payment within 3-5 days.  Partners may email Rob if they wish to 
know Marta B's email address for further queries.

Marta B made clear that the UKAEA procurement dept is currently overloaded 
(explaining problems with issuing Peter Challenor / Tim Dodwell contract) and 
to copy her in to any relevant correspondence.

Deferred workshop sessions will occur after Easter: Wayne Arter will arrange a 
Doodle poll.  Rob Akers suggests a shared calendar for the project to automate 
this coordination and asked for ideas for implementing such (e.g. is there a 
Slack feature?).

\section{Afternoon}

\subsection{Anisotropic transport / elliptic solvers}

Chris Cantwell reported on D2.1 attempts to solve the anisotropic diffusion 
problem using \F{Nektar++}: a rectangular domain with a discrete step of 
`source', then homogeneous Dirichlet boundary on a second side and the 
remaining two sides using a Neumann condition.  Even with $p=8$ high-order 
elements the system displays the Gibbs (Runge) phenomenon of oscillations due 
to the representation of a steep-sided function in polynomials.  They tried 
aligning the quads with the axes of the diffusion tensor, to little help, then 
switched the top hat function for a `regularized' version (combination of 
hyperbolic tangents), which helped, but still showed a bit of oscillation that 
David M said was probably due to a badly-shaped triangular element.  Further 
refinement of the mesh appeared to remove the residual oscillation, giving a 
nice-looking result.

D2.2 vectorizing matrix-free operators: the issue here was to get the relevant 
data contiguous in memory, which means elements are no longer stored 
sequentially but that the data for a specific element is interlaced with the 
corresponding data for all elements.  A roofline plot showed good performance 
with regular elements, hitting the AVX2 floating point performance roofline; 
curvilinear elements are more memory-bottlenecked due to the need to load a 
Jacobian (i.e. non-affine relation between reference and individual elements).  
Wayne Arter made the point that one goal was to quantify the quality of the 
solution as a function of the regularization parameter used in smoothing the 
boundary data.  Rob said that a criterion for what degree of Runge oscillation 
we can allow is the experimental measurement threshold.
Spencer S asked how we know where to refine the mesh and where to align it with 
the magnetic field.  Patrick Farrell queried whether the problem was ill-posed 
but the consensus was that this was not the case.  John Omotani asked whether 
there was a general method for aligning meshes to field lines in 3D; David M 
said this work was one of his deliverables and that they had code for 
addressing the 2D problem.  One other approach mentioned was simply to 
$p$-refine without trying to align the mesh but John O had little faith in that 
procedure.  X-points bring further complexity.

It was noted that one could do local $p$-adaptivity and local $r$-adaptivity 
(the latter might involve local deformation of the mesh).  Wall {\it and} field 
alignment is a nasty problem due to the small angles.  Wayne Arter mentioned 
that, in the literature, there is up to $10^8$ difference in size of 
diffusivity component; he showed a slide with an illustration of what happened 
to a quadrilateral element on a circuit of a tokamak: returned to original 
position but sheared and no longer congruent with how it started out.  Possible 
solutions: hanging nodes or use tetrahedra / prisms.  Rob asked how long we 
have before needing to decide a meshing strategy here: Wayne Arter said a few 
months.

Patrick Farrell reported on his own work on the anisotropic transport problem.  
A literature review had uncovered examples of asymptotic-preserving schemes by 
Deluzet and Narski for highly anisotropic Laplacians (up to $10^{10}$ 
anisotropy factor in the diffusion tensor components).  This work showed any 
Galerkin method / LU decomposition would fail for such large anisotropies, and 
proposed a strategy for dealing with this issue: introduce an extra variable 
obeying an auxiliary (homogeneous) equation, solved via an iterative scheme.  
Deluzet-Narski proved convergence.  Patrick F showed test cases, some of which 
worked well, but also showed that convergence stalls in some cases (he wants to 
correspond further with the authors re.\ this).  Patrick F showed then a possible 
solution he devised the day before using a vectorial auxiliary variable, so not 
sure whether it always works or indeed whether it is already in the literature, 
though initial tests were promising.  Possible overlap with preconditioners 
work (84) as the linear algebra problem is not symmetric.  Patrick F also 
showed examples of Deluzet-Narski for a 3D problem run on a supercomputer; also 
his work was presented in the DSL of Firedrake (UFL)  which looks very good.

Felix Parra mentioned he had tried something similar but using a scalar 
auxilliary variable (the gradient component parallel to the magnetic field) in 
2009 but had not succeeded in making it work.  Patrick thought the idea was 
good.

Wayne Arter suggested Patrick liaise with Pierre Degond at Imperial, an expert 
on asymptotic-preserving schemes (although he has only a visiting appt).
 



\subsection{Surrogate models for data compression / turbulence}

Ed Threlfall presented slides showing convection: firstly, a 3D case where 
heating from below caused mushroom-like convection structures in water.  He 
then showed 2D convection in a vertical slot heated from the side i.e. the 
double-glazing problem and explained the three main regimes accessed by 
increasing the Rayleigh number: conducting (diffusion-dominated), regular 
convection, and turbulent convection.  A comparison between a turbulent \F{Nektar++}
simulation and a photograph from a 1965 experiment was shown.  Then 
Ed T played a movie of the same turbulent simulation recorded using \F{Nektar++}
that showed the instability to boundary-layer waves.  Here, Spencer S 
suggested boundary layer refinement (Ed T explained that this had been tried 
but then relaxed in order to resolve turbulent structures propagating into the 
core of the tank and their growth).  Spencer S suggested trying the 2.5D (i.e. 
Fourier in $z$) capability of \F{Nektar++} to do a full 3D simulation and 
seeing whether the similarity to the 1965 results was enhanced.  Wayne Arter
indicated that there was more experimental work in the last few years, by 
Daverat, Li et al., but he has not had chance to return to it.  Rob Akers 
suggested Ed T liaise with Andrew Davis (UKAEA) who is simulating swirl tubes 
with rifling structures designed to promote turbulence.

Tim Dodwell presented on constructing Gaussian process (GP) surrogate models of 
the same \F{Nektar++} simulations.  He noted that this problem is a good 
candidate for a surrogate as two inputs (he used Prandtl number and Rayleigh
number $Ra$) but a $42,000$-dimensional model space (the temperature field), 
with full mode runs taking c.4 hours on 24 CPUs for his choice of time step.  
Full-field GP emulator approximates the full output $T$ field.  GPs offer 
forward propagation, Bayesian sensitivity analysis, calibration and data-driven 
preconditioners, so better than polynomial chaos approaches.  Solution 
approximated by $M$ independent Gaussian processes representing the principal 
components (the leading one seems to be a steady convection mode): overlap with 
this mode decayed exponentially as $Re$ was increased.  Code used was {\it mogp-emulator}.
Another consideration is using the UQ techniques to determine 
the maximum stable timestep over the parameter space, as currently this needs 
to be `guessed'.  UQ would need to be extended to handle the failures caused by 
an over-large timestep.  Overall, results were impressive, given that only 20 
model runs were used, and Tim D mentioned that he will probably be able to make 
rapid progress now that David M has returned to work.

Spencer S asked about using machine learning for adaptive meshing.  Rob Akers 
has spoken to Garth Wells re.\ this (the latter not enthusiastic) but also Lee 
Margetts) who has done an extensive literature survey on the subject and who 
seems more keen.

Pater Coveney brought up the issue described in the UQ textbook by RC Smith,
of the many systems that are not governed by 
Gaussian statistics; heavy-tailed distributions make estimation of the mean 
from a few measurements unreliable.  Peter Challenor replied
that it is a common misconception that GP emulators are restricted to the
case of Gaussian errors; they just use a Gaussian interpolant.

Ben~McMillan delivered a presentation on hierarchies of models in tokamak 
turbulence (work with Chris Pringle and Bogdan Teca, Cambridge).  He explained 
the range of models, from zero-dimensional toy model right up to the full 6D 
Boltzmann kinetics.  He focused on the `plasma interchange' model that shares 
some of the features of fluid convective instabilities (interchange instability 
replaces buoyancy instability; again viscosity versus diffusion).  He showed 
turbulent-like structures on a space-time plot and likened these to the 
filaments / blobs seen in tokamak-s.  Also showed that the 1D model could be run 
for sufficiently long so as to capture rate `extreme' events that are clearly 
of great interest in real tokamaks (these are the events that can cause severe 
damage to the first wall).  Wayne Arter drew attention to a the work
of Mohammad Farazmand on extreme events e.g. rogue waves (using heavy-duty numerics).
