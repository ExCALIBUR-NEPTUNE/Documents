For each \Papp, bidders should describe their approach (including their initial 
thoughts around provisional selection for algorithm and library options), programming 
language preference etc.\ (which will be down selected in negotiation with UKAEA) 
and should indicate how their solution will in the eventual \nep \   code deliver: 

\begin{enumerate}
\item Trustworthiness, achieved through rigorous Verification and Validation procedures. 
\item Code sustainability, software remains capable of significant development. 
\item Easy deployment and usage by users with different levels of \nep \ expertise.
\item Good performance as measured by both strong and weak scalings. 
\item Performance portability across a range of candidate Exascale architectures. 
\item Actionable solutions, achieved through Uncertainty Quantification.
\item The ability to generate surrogate models derived from the code (eg.\ through 
Model Order Reduction).
\end{enumerate}

As a baseline, all \Papp s should target $x86$ (and ideally IBM POWER and ARM CPU) 
architectures (multi-core and multiple node) for scalability to first generation 
Exascale hardware. Bidders should feel free to also target other Exascale candidate 
architectures (eg.\ GPGPU) and demonstrate a willingness to explore the use of 
novel hardware as it becomes available to the \exc \   project as part of the 
novel test bed programme. If a bidder determines there is insufficient resource 
to do so (or if they do not possess the relevant skills), they should instead indicate 
that they are willing to work closely with other \nep \   team members who have 
demonstrated skills in accelerator exploitation. In order to target the chosen 
architecture(s), bidders should indicate whether they intend to deploy MPI and 
OpenMP or some other parallelisation technology (ideally with a focus upon
performance portability) -- including a clear explanation as to why
(and be prepared to negotiate with UKAEA).

As Responsible Officer (RO) for developing the ``referent Plasma Fluid model'' 
(see section 1.3) and associated infrastructure, the successful bidder to this 
call will be expected to work closely with all members of the \exc \   \nep \   
community.  In particular, the RO is required to work closely 
with the successful bidder to Call~T/NA078/20 in order to co-design the optimal numerical 
scheme and meshing technology for instantiating the referent fluid model eventually 
within an Exascale performant framework. They will be expected to help UKAEA rapidly 
lay down a code/platform and development infrastructure that is optimal for team-distributed 
software engineering and to help build team cohesion. 

%To this end, all partners in the \exc \   \nep \   project will be expected to 
%sign up to a ``project charter'' -- a live document that they will jointly develop 
%with the rest of the team. The aim of the charter will be to ensure that common 
%standards for software and documentation are laid down and adhered to by all parties 
%and that all contributors work closely together in order to meet the goals of the 
%programme -- to develop a roadmap, knowledge, skills and an emerging coherent 
%infrastructure that will allow the UK to instantiate a state of the art, Exascale 
%performant modelling and simulation capability for the ITER era. A draft charter 
%is provided with this call. The bidder is encouraged to comment upon this draft 
%and should indicate how they intend to ensure that the deliverables of this call 
%are compliant with the charter, (to be maintained by UKAEA) and how they intend 
%to help ensure that the team as a whole converges upon a consensus for the programme 
%(ideally indicating where they have done similar for other large scale/complex 
%projects in the past). The bidder should budget appropriately for T\&S, to ensure 
%that they have adequate resource for working closely with partners distributed 
%across the UK, for face to face meetings, hackathon events etc.

