\begin{enumerate}

\item Identification and classification of at least three different approaches to
UQ that that show promise of suitability for HPC / Exascale implementation
particularly in the context of a spectral element code. It is expected that at least
two of these will be drawn from those described at the end of Section~1.2.2.

\item Potentially suitable approaches will be evaluated by means of 
literature and code surveys and consultation
with UKRI and industry experts.
Relatively small development tasks may initially be undertaken to 
test candidate methods for
accuracy, stability and HPC scalability potential.
At minimum, evaluation should produce an objective assessment as to
whether an approach is suitable for:
\begin{itemize}
\item Immediate implementation in \nep\ defined in the maturing project roadmap
and through co-design with project partners initially in the Y1-2 \papp s.
\item Recognition by the \nep\ design so that deferred implementation is facilitated,
perhaps with suggested training in UQ methods to aid co-deaign.
\item Implementation only in a restricted class of machine architectures, thereby
steering architecture choices and priorities for future infrastructure.
\end{itemize}

\item The bidder should indicate how, over the duration of the grant, they intend to engage
in community building and development, for example by assisting in the production
of one or more \papp s to demonstrate key features of preferred approaches.

\end{enumerate}

%\input{../commitment0}

