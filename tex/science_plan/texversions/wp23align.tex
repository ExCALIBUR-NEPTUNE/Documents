\textbf{FM-WP2} and \textbf{FM-WP3} will concentrate upon development of the 
two close coupled models of the NEPTUNE
programme, specifically FM-WP2 around the inclusion of kinetic effects into 
existing and new edge plasma models, and
FM-WP3 of particle based models for describing the region outside and just 
inside the plasma (neutral atoms/molecules
and partially ionized impurities). Initial exploratory work will be carried out 
using existing codes (as per above) and
via the development of proxy-apps, for example to expose options for exascale 
targeted hardware (GPUs, ARM technology
etc.).

This work package will begin by identifying a referent in conjunction with 
potential users, whereby ``referent'' is
meant a model that establishes the maximum detail and complexity of plasma that 
the software could ever be reasonably
expected to model (beyond exascale). \ Critical features of the referent will 
be identified and prioritised for
implementation as part of the project. This may involve replacing a kinetic 
model by moment-based or fluid models.
Input from the European Boundary Code (EBC) development (funded by EUROfusion 
wherein UKAEA is a core partner) will be
important to this process. In addition, the speed of model execution will be a 
consideration as indicated earlier. A
provisional sequence of developments is as follows (and will be tuned as part 
of the initial NEPTUNE requirements
capture exercise funded in year 1):


\bigskip

\liststyleWWNumx
\begin{enumerate}
\item 2D model of anisotropic heat transport.
\item 2D elliptic solver in complex geometry.
\item 1D fluid solver with simplified physics but with UQ and realistic boundary conditions.
\item Spatially 1D plasma model incorporating velocity space effects.
\item Spatially 1D multispecies plasma model.
\item Spatially 2D plasma model incorporating velocity space effects.
\item Interaction between models of different dimensionality.
\item Spatially 3D plasma kinetic models.
\end{enumerate}

\bigskip

Users from the wider fusion community will be engaged via a workshop in Feb 
2020 to firm up on the plan, to help define
work for defrayment that they can themselves can carry out and will be engaged 
throughout the project to help define
and add new surrogate models through a wide engagement programme and co-design 
(e.g. to treat boundary sheaths) and/or
other important physical effects (e.g. radiation, charge exchange 
recombination, etc.) and compare with existing codes
and experiments. 
This work package will begin by identifying a referent (as above) in 
conjunction with potential users and experts in
atomic physics. Critical features of the referent will be identified and 
prioritised for implementation as part of the
project. This may involve deploying a particle-based method, a moment-based 
model or a fluid model (or even a
combination thereof). Input from the European EBC programme will be important 
to this process, as will the speed of
model execution as indicated earlier. A careful assessment of existing codes 
currently in use {will be
necessary, as will cross validation with the established models (notably 
B2-EIRENE
}\href{https://www.tandfonline.com/doi/abs/10.13182/FST47-172}{\textstyleInterne
tlink{{[19]}}}{).
A provisional }sequence of developments is as follows:


\bigskip

\liststyleWWNumxi
\begin{enumerate}
\item 2D particle-based model of neutral gas \& impurities with critical physics.
\item 2D moment-based model of neutral gas \& impurities.
\item Interaction with 2D plasma model when available.
\item 3D model of neutral gas \& impurities.
\item Interaction with 3D plasma model.
\item Staged introduction of additional neutral gas/impurity physics.
\end{enumerate}

\bigskip

As per FM-WP2, users from the wider fusion modelling community will first be 
engaged through a workshop in Feb 2020, and
will help build a UK team that will add new physics and compare with existing 
codes and experiments, providing detailed
and rigorous verification and validation (V\&V). This is likely to require the 
provision of databases for different
ionisation and excitation reactions, both in the plasma volume and at surfaces.
