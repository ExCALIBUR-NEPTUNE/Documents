\textbf{FM-WP4} will focus upon the design of data structures to interface 
between the different models and generally
ensure best practice in scientific software engineering (being responsible for 
Quality Assurance, co-design,
integration etc.). Aligned with Met Office work {package WC-WP1, this programme 
of work will also
explore the use of a ``separation of concerns'' methodology for the Fusion use 
case critical components (e.g. by
exploring the use of Kokkos
}\href{https://cfwebprod.sandia.gov/cfdocs/CompResearch/docs/Kokkos-Multi-CoE.pd
f}{\textstyleInternetlink{{[12]}}}{
and Domain Specific Language (DSL)
}\href{https://en.wikipedia.org/wiki/Domain-specific_language}{\textstyleInterne
tlink{{[13]}}}{ technologies}).

\subsection[Code structure and coordination (Fusion modelling, work package
FM{}-WP4)]{\textbf{\textcolor[rgb]{0.12156863,0.28627452,0.49019608}{Code 
structure and coordination
}}\textcolor[rgb]{0.12156863,0.28627452,0.49019608}{(Fusion modelling, work 
package
}\textbf{\textcolor[rgb]{0.12156863,0.28627452,0.49019608}{FM-WP4}}\textcolor[rg
b]{0.12156863,0.28627452,0.49019608}{)}}

\bigskip

The most important aim of this work package will be to drive user engagement 
and ensure that the software is fit for its
defined purpose, first by requirements capture, then by defining suitably 
flexible code structures and related
e-Infrastructure for users, ultimately supporting uptake of the new code(s). In 
order to achieve this aim, this work
package will coordinate across the other tasks FM-WP1-3, to ensure that 
outcomes are compatible and of sufficiently
high quality. There will be management and coordination tasks that will grow as 
the project matures,
{connecting with the EUROfusion E-TASC (TSVV) programme, the EPSRC T.P. 
Turbulence Programme
}\href{https://www.york.ac.uk/physics/news/departmentalnews/plasma-fusion/major-
grant-award-supports-fusion-energy-research/}{\textstyleInternetlink{{[20]}}}{
and the }US ECP programme etc.


\bigskip

Coordination tasks will include (but are not restricted to):


\bigskip

\liststyleWWNumxii
\begin{enumerate}
\item Allocation of resource between tasks and setting project priorities.
\item Ensuring a consistent choice of definitions (ontology) of objects or 
equivalently classes.
\item Definition of common interfaces to components for data input and output. 
\item Design of suitably flexible data structures for common use by all 
developers.
\item Establishment, promotion and support of good scientific software 
engineering practice.
\item Evaluation and deployment of performance portability tools and DSLs 
targeting Exascale-relevant architectures.
\item Integration of the developed software into a VVUQ framework (exploiting 
common approaches developed under XC-WP1
and XC-WP2).
\item Coordination of a benchmarking framework for correctness testing and 
performance evaluation of the developed
software stack.
\end{enumerate}

\bigskip

Along with FM-WP1, this work package will be prioritised for an early start, as 
good scientific software engineering
practice needs to be agreed quickly, and well documented interfaces to 
components need to be available early to ensure
that best practice design is embedded from the start.

