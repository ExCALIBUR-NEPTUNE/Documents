% This file was converted to LaTeX by Writer2LaTeX ver. 1.4
% see http://writer2latex.sourceforge.net for more info
\documentclass[a4paper]{article}
\usepackage[latin1]{inputenc}
\usepackage[T1]{fontenc}
\usepackage[english]{babel}
\usepackage{amsmath}
\usepackage{amssymb,amsfonts,textcomp}
\usepackage{color}
\usepackage{array}
\usepackage{supertabular}
\usepackage{hhline}
\usepackage{hyperref}
\hypersetup{pdftex, colorlinks=true, linkcolor=blue, citecolor=blue, 
filecolor=blue, urlcolor=blue, pdftitle=ExCALIBUR Fusion Modelling System 
Science Plan, pdfauthor=Wayne Arter, pdfsubject=Stage of proposal e.g. ITT; PQQ 
Client name and reference number, pdfkeywords=}
\usepackage[pdftex]{graphicx}
% footnotes configuration
\makeatletter
\renewcommand\thefootnote{\arabic{footnote}}
\setcounter{footnote}{1}
\makeatother
\newcommand\textsubscript[1]{\ensuremath{_{\text{#1}}}}
% Text styles
\newcommand\textstyleGuidanceTextChar[1]{\foreignlanguage{english}{\textrm{\text
color[rgb]{0.0,0.1882353,0.33333334}{#1}}}}
\newcommand\textstyleInternetlink[1]{\textcolor{blue}{#1}}
\newcommand\textstylepagenumber[1]{#1}
% Outline numbering
\setcounter{secnumdepth}{2}
\renewcommand\thesection{\arabic{section}}
\renewcommand\thesubsection{\arabic{section}.\arabic{subsection}}
\makeatletter
\newcommand\arraybslash{\let\\\@arraycr}
\makeatother
% List styles
\newcommand\liststyleWWNumvii{%
\renewcommand\labelitemi{[F0B7?]}
\renewcommand\labelitemii{o}
\renewcommand\labelitemiii{[F0A7?]}
\renewcommand\labelitemiv{[F0B7?]}
}
\newcommand\liststyleWWNumvi{%
\renewcommand\theenumi{\arabic{enumi}}
\renewcommand\theenumii{\alph{enumii}}
\renewcommand\theenumiii{\roman{enumiii}}
\renewcommand\theenumiv{\arabic{enumiv}}
\renewcommand\labelenumi{\theenumi.}
\renewcommand\labelenumii{\theenumii.}
\renewcommand\labelenumiii{\theenumiii.}
\renewcommand\labelenumiv{\theenumiv.}
}
\newcommand\liststyleWWNumviii{%
\renewcommand\theenumi{\arabic{enumi}}
\renewcommand\theenumii{\arabic{enumii}}
\renewcommand\theenumiii{\arabic{enumiii}}
\renewcommand\theenumiv{\arabic{enumiv}}
\renewcommand\labelenumi{P\theenumi.}
\renewcommand\labelenumii{\theenumii.}
\renewcommand\labelenumiii{\theenumiii.}
\renewcommand\labelenumiv{\theenumiv.}
}
\newcommand\liststyleWWNumxiii{%
\renewcommand\theenumi{\alph{enumi}}
\renewcommand\theenumii{\alph{enumii}}
\renewcommand\theenumiii{\roman{enumiii}}
\renewcommand\theenumiv{\arabic{enumiv}}
\renewcommand\labelenumi{\theenumi.}
\renewcommand\labelenumii{\theenumii.}
\renewcommand\labelenumiii{\theenumiii.}
\renewcommand\labelenumiv{\theenumiv.}
}
\newcommand\liststyleWWNumix{%
\renewcommand\theenumi{\arabic{enumi}}
\renewcommand\theenumii{\alph{enumii}}
\renewcommand\labelenumi{\theenumi.}
\renewcommand\labelenumii{\theenumii.}
\renewcommand\labelitemi{[F0A7?]}
\renewcommand\labelitemii{[F0B7?]}
}
\newcommand\liststyleWWNumx{%
\renewcommand\theenumi{\arabic{enumi}}
\renewcommand\theenumii{\alph{enumii}}
\renewcommand\theenumiii{\roman{enumiii}}
\renewcommand\theenumiv{\arabic{enumiv}}
\renewcommand\labelenumi{\theenumi.}
\renewcommand\labelenumii{\theenumii.}
\renewcommand\labelenumiii{\theenumiii.}
\renewcommand\labelenumiv{\theenumiv.}
}
\newcommand\liststyleWWNumxi{%
\renewcommand\theenumi{\arabic{enumi}}
\renewcommand\theenumii{\alph{enumii}}
\renewcommand\theenumiii{\roman{enumiii}}
\renewcommand\theenumiv{\arabic{enumiv}}
\renewcommand\labelenumi{\theenumi.}
\renewcommand\labelenumii{\theenumii.}
\renewcommand\labelenumiii{\theenumiii.}
\renewcommand\labelenumiv{\theenumiv.}
}
\newcommand\liststyleWWNumxii{%
\renewcommand\theenumi{\arabic{enumi}}
\renewcommand\theenumii{\alph{enumii}}
\renewcommand\theenumiii{\roman{enumiii}}
\renewcommand\theenumiv{\arabic{enumiv}}
\renewcommand\labelenumi{\theenumi.}
\renewcommand\labelenumii{\theenumii.}
\renewcommand\labelenumiii{\theenumiii.}
\renewcommand\labelenumiv{\theenumiv.}
}
% Page layout (geometry)
\setlength\voffset{-1in}
\setlength\hoffset{-1in}
\setlength\topmargin{1.27cm}
\setlength\oddsidemargin{2.54cm}
\setlength\textheight{22.281998cm}
\setlength\textwidth{15.920999cm}
\setlength\footskip{2.4390001cm}
\setlength\headheight{1.27cm}
\setlength\headsep{1.169cm}
% Footnote rule
\setlength{\skip\footins}{0.119cm}
\renewcommand\footnoterule{\vspace*{-0.018cm}\setlength\leftskip{0pt}\setlength\
rightskip{0pt plus 
1fil}\noindent{\rule{0.0\columnwidth}{0.018cm}}\vspace*{0.101cm}}
% Pages styles
\makeatletter
\newcommand\ps@Standard{
  
\renewcommand\@oddhead{\textbf{\textcolor[rgb]{0.14117648,0.22745098,0.3647059}{
Commercial in Confidence}}}
  \renewcommand\@evenhead{\@oddhead}
  \renewcommand\@oddfoot{\textstylepagenumber{\thepage}}
  \renewcommand\@evenfoot{\@oddfoot}
  \renewcommand\thepage{\arabic{page}}
}
\newcommand\ps@FirstPage{
  \renewcommand\@oddhead
  \renewcommand\@evenhead
  \renewcommand\@oddfoot
  \renewcommand\@evenfoot
  \renewcommand\thepage{\arabic{page}}
}
\makeatother
\pagestyle{Standard}
\setlength\tabcolsep{1mm}
\renewcommand\arraystretch{1.3}
\title{ExCALIBUR Fusion Modelling System Science Plan}
\author{Wayne Arter}
\date{2020-03-20}
\begin{document}
\clearpage\setcounter{page}{1}\pagestyle{Standard}
\thispagestyle{FirstPage}
[Warning: Draw object ignored]%
%Which tokamak interior is pictured on front page - EAST or WEST or something 
else?
%Wayne Arter
%November 30, 2019 2:53 PM



\bigskip


\bigskip


\bigskip


\bigskip

[Warning: Draw object ignored]

[Warning: Draw object ignored]\ 

\clearpage
\bigskip


\bigskip

{\centering
\textstyleGuidanceTextChar{\textbf{UKAEA REFERENCE AND APPROVAL SHEET}}
\par}

\begin{center}
\tablefirsthead
\tablehead
\tabletail
\tablelasttail
\begin{supertabular}{|m{3.24cm}m{2.333cm}m{2.0149999cm}m{2.553cm}m{0.83400005cm}
m{4.0160003cm}|}
\hline
\multicolumn{2}{|m{5.7730002cm}|}{~
} &
\multicolumn{2}{m{4.768cm}|}{Client Reference:} &
\multicolumn{2}{m{5.05cm}|}{~
}\\\hline
 &
 &
\multicolumn{2}{m{4.768cm}|}{UKAEA Reference:

~
} &
\multicolumn{2}{m{5.05cm}|}{CD/\exc \ -FMS/0001}\\\hhline{~~----}
 &
 &
\multicolumn{2}{m{4.768cm}|}{Issue:} &
\multicolumn{2}{m{5.05cm}|}{1.10}\\\hhline{~~----}
 &
 &
\multicolumn{2}{m{4.768cm}|}{Date:} &
\multicolumn{2}{m{5.05cm}|}{30 November 2019}\\\hline
\multicolumn{6}{|m{15.991cm}|}{~

Project Name: ExCALIBUR Fusion Modelling System.

~

Version 1.10 -- modifications in response to reviewer recommendations.

~
}\\\hline
\multicolumn{1}{|m{3.24cm}|}{~
} &
\multicolumn{2}{m{4.5480003cm}|}{Name and Department} &
\multicolumn{2}{m{3.5870001cm}|}{Signature} &
Date\\\hline
\multicolumn{1}{|m{3.24cm}|}{Prepared By:

~
} &
\multicolumn{2}{m{4.5480003cm}|}{Wayne Arter

Lucian Anton

Debasmita Samaddar

Rob Akers 

~

MSSC} &
\multicolumn{2}{m{3.5870001cm}|}{N/A

N/A

N/A

N/A} &
21/10/19

21/10/19

21/10/19

22/10/19\\\hline
\multicolumn{1}{|m{3.24cm}|}{Reviewed By:} &
\multicolumn{2}{m{4.5480003cm}|}{Ash Vadgama (AWE)} &
\multicolumn{2}{m{3.5870001cm}|}{N/A} &
14/11/19\\\hline
\multicolumn{1}{|m{3.24cm}|}{Modifiedd By:

~
} &
\multicolumn{2}{m{4.5480003cm}|}{Rob Akers

~

MSSC} &
\multicolumn{2}{m{3.5870001cm}|}{ 
\includegraphics[width=2.101cm,height=1.245cm]{CD\exc \ FMS0001110WA-img001.jp
g} } &
27/11/19\\\hline
\multicolumn{1}{|m{3.24cm}|}{Approved By:

~
} &
\multicolumn{2}{m{4.5480003cm}|}{Martin O'Brien 

~

MSSC} &
\multicolumn{2}{m{3.5870001cm}|}{ 
\includegraphics[width=3.33cm,height=1.152cm]{CD\exc \ FMS0001110WA-img002.png
} } &
~
\\\hline
\end{supertabular}
\end{center}

\bigskip


\bigskip


\bigskip


\bigskip


\bigskip


\bigskip


\bigskip


\bigskip


\bigskip


\bigskip


\bigskip


\bigskip


\bigskip


\bigskip


\bigskip


\bigskip


\bigskip

{\selectlanguage{english}
\textbf{\textcolor[rgb]{0.12156863,0.28627452,0.49019608}{ExCALIBUR}}}

{\selectlanguage{english}
\textbf{\textcolor[rgb]{0.12156863,0.28627452,0.49019608}{Fusion Modelling 
System Science Plan }}}

\section[Purpose]{\textcolor[rgb]{0.12156863,0.28627452,0.49019608}{Purpose}}

\bigskip

{\selectlanguage{english}\color{black}
This document outlines the Science Plan for research to be commissioned by the 
Met Office for the ``Fusion Modelling
System'' use case of the SPF 
``\textbf{\textcolor[rgb]{0.12156863,0.28627452,0.49019608}{Ex}}ascale
\textbf{\textcolor[rgb]{0.12156863,0.28627452,0.49019608}{C}}omputing
\textbf{\textcolor[rgb]{0.12156863,0.28627452,0.49019608}{Al}}gorithms and
\textbf{\textcolor[rgb]{0.12156863,0.28627452,0.49019608}{I}}nfrastructures
\textbf{\textcolor[rgb]{0.12156863,0.28627452,0.49019608}{B}}enefiting
\textbf{\textcolor[rgb]{0.12156863,0.28627452,0.49019608}{U}}K
\textbf{\textcolor[rgb]{0.12156863,0.28627452,0.49019608}{R}}esearch'' 
programme (ExCALIBUR). It complements ExCALIBUR
Met Office Science Plan and Activities document [1], setting out a vision for 
the project's contribution to the
overarching ExCALIBUR plan. This document is aimed at stakeholders including 
the SPF ExCALIBUR Programme Board and
Steering Committee, has been informed by consultation with the Met Office, with 
domain experts from across the UKRI
community and from within UKAEA. This Science Plan is intended to complement 
the ExCALIBUR activities that are funded
through the UKRI Research Councils and the Met Office ``Weather \& Climate 
Prediction'' use case.}


\bigskip

{\selectlanguage{english}\color{black}
\textit{The ExCALIBUR programmatic aims described in [1] are common with those 
of the Fusion Modelling System use case
and so are not repeated here.}}


\bigskip


\bigskip

{\selectlanguage{english}\color{black}
\textbf{\textcolor[rgb]{0.12156863,0.28627452,0.49019608}{Fusion Modelling Use 
Case activities}}}

\section[The aim of the ``Fusion Modelling'' use case of ExCALIBUR is, via 
the exploitation of the ExCALIBUR
principles outlined in [1], to develop new algorithms, software and related 
e-Infrastructure that will result in
the efficient use of current Petascale and future Exascale supercomputing 
hardware in order to a) draw insights from
ITER [2] ``Big Data'' and b) to guide and optimise the design of the UK 
demonstration nuclear fusion power plant
STEP [3] and related fusion technology as we approach the Exascale. The aims 
of the work are to deliver expertise in,
and tools for, ``in-silico'' reactor interpretation and design, initially 
with a focus upon the ``edge''
region of the tokamak [4] plasma where hot plasma comes into contact with the 
material walls of the machine (see
project \nep \  below). This challenging, multi-physics, multi-scale 
intersection between plasma physics and
engineering has long been heralded as an ``exascale'' modelling and 
simulation problem and its solution is well
established as critical to the success of commercial fusion energy. Existing 
legacy (and often ``black box'') codes
do not scale and do not contain all the latest physics that is believed to be 
important in the ``burning plasma
regime'' (notably kinetic effects); without a significant investment in this 
area, it will not be possible to design
the ``divertor'' region of future fusion power plants (the region of the 
machine where hot plasma comes into
contact with the surrounding first wall -- see Figure 1 below). 
]{\textmd{{The aim of the ``Fusion
Modelling'' use case of ExCALIBUR is, via the exploitation of the ExCALIBUR 
principles outlined in [1], to develop new
algorithms, software and related e-Infrastructure that will result in the 
efficient use of current Petascale and future
Exascale supercomputing hardware in order to a) draw insights from ITER
}}\href{https://www.iter.org/}{\textstyleInternetlink{\textmd{{[2]}}}}\textmd{{ 
``Big
Data'' and b) to guide and optimise the design of the UK demonstration nuclear 
fusion power plant STEP
}}\href{https://www.gov.uk/government/news/uk-to-take-a-big-step-to-fusion-elect
ricity}{\textstyleInternetlink{\textmd{{[3]}}}}\textmd{{
and related fusion technology as we approach the Exascale. The aims of the work 
are to deliver expertise in, and tools
for, ``in-silico'' reactor interpretation and design, initially with a focus 
upon the ``edge'' region of the tokamak
}}\href{https://en.wikipedia.org/wiki/Tokamak}{\textstyleInternetlink{\textmd{{[
4]}}}}\textmd{{
plasma where hot plasma comes into contact with the material walls of the 
machine (see project \nep \  below). This
challenging, multi-physics, multi-scale intersection between plasma physics and 
engineering has long been heralded as
an ``exascale'' modelling and simulation problem and its solution is well 
established as critical to the success of
commercial fusion energy. Existing legacy (and often ``black box'') codes do 
not scale and do not contain all the
latest physics that is believed to be important in the ``burning plasma 
regime'' (notably kinetic effects); without a
significant investment in this area, it will not be possible to design the 
``divertor'' region of future fusion power
plants (the region of the machine where hot plasma comes into contact with the 
surrounding first wall -- see Figure 1
below). }}}

\bigskip

\section[]{ 
\includegraphics[width=9.223cm,height=10.072cm]{CD\exc \ FMS0001110WA-img003.p
ng} }
{\selectlanguage{english}
\textit{{Figure 1: Schematic diagram of a generic tokamak ``poloidal cross 
section'' showing the areas
of plasma and first wall that will be targeted by project \nep \  (shaded 
circles). Attribution: G. Federici et al. [CC
BY 3.0 
(}}\url{https://creativecommons.org/licenses/by/3.0}\textit{{)]}}\footnote{
\foreignlanguage{english}{Minor modifications to figure.}}\textit{{.}}}

\section[This is an activity that must be performed ``in-silico'' using 
``actionable'' modelling and
simulation as the highly coupled physics in the divertor cannot be created in 
the laboratory with conditions
approaching the reactor regime. It is also well established that there is a 
need for further development of the science
(which will take place in close cooperation with ExCALIBUR numericists) -- 
the codes of the ITER and Exascale era
must therefore be easy to adapt as knew knowledge becomes established (e.g. 
through ITER operations). The programme
itself is designed to exploit commonality of solutions across the disciplines 
and domains represented by ExCALIBUR and
to foster the development of a UK interdisciplinary community (within our 
National laboratories and institutes across
UKRI and throughout Academia).]{\textmd{{This is an activity that must be 
performed ``in-silico''
using ``actionable'' modelling and simulation as the highly coupled physics in 
the divertor cannot be created in the
laboratory with conditions approaching the reactor regime. It is also well 
established that there is a need for further
development of the science (which will take place in close cooperation with 
ExCALIBUR numericists) -- the codes of the
ITER and Exascale era must therefore be easy to adapt as knew knowledge becomes 
established (e.g. through ITER
operations). The programme itself is designed to exploit commonality of 
solutions across the disciplines and domains
represented by ExCALIBUR and to foster the development of a UK 
interdisciplinary community (within our National
laboratories and institutes across UKRI and throughout Academia).}}}

\bigskip

Activities will initially be around an ambitious programme to develop a 
computational model that includes plasma kinetic
effects believed to be essential for a first-principles description of the 
complex dynamics of high temperature fusion
plasma\textcolor[rgb]{0.0,0.5019608,0.0}{. }{Codenamed \nep \ 
(}\textbf{{NE}}{utrals \& }\textbf{{P}}{lasma
}\textbf{{TU}}{rbulence }\textbf{{N}}{umerics for
the }\textbf{{E}}{xascale), work will initially focus upon coupling the 
turbulent
plasma periphery to the surrounding neutral gas and partially ionized 
impurities that exist between the plasma and
plasma facing components, in the presence of an arbitrary tokamak magnetic 
field and full 3D first wall geometry.
Figure 1 shows a schematic of a generic tokamak ``poloidal cross section'', 
highlighting the targeted regions of plasma
and machine, namely the main chamber between core plasma, scrape off layer and 
wall (upper shaded circle) and the so
called ``plasma exhaust'' or }{``divertor'' region (lower shaded circle) 
where heat and particles
come into direct contact with material surfaces. Infrastructure and workflows 
will be developed so that the resulting
close coupled models can be constructed routinely based around a high-fidelity 
representation of the geometry described
by Computer Aided Design (CAD) systems (this likely requiring the development 
of high order meshing technology, e.g.
using Nektar++).}


\bigskip

{Quality control, verification, validation and uncertainty quantification 
(VVUQ, e.g. via intrusive or
ensemble-based methods) will be embedded across all areas of the project to 
ensure numerical predictions are
``actionable''. The initial aim is to develop knowledge and UK capability 
around how to design world leading Exascale
targeted software for the benefit of the UK academic community, UKRI and the UK 
nuclear supply chain. Emphasis will be
placed upon building a connected community, delivering training and skills 
development activities as necessary, aligned
with Pillar 4 of the ExCALIBUR aims (Investing in People).}


\bigskip

Together, the UK plasma and HPC communities have unique expertise required to 
deliver this project. UKAEA will bring
together world-class experts in tokamak edge physics, gyrokinetic theory, and 
highly scalable algorithms, to address
arguably one of the most important unsolved challenges of fusion research -- 
how to design a plasma ``exhaust system''
that can reliably restrict power flux reaching the material surfaces to 
tolerable levels, i.e. no more than
\~10MW/m\textsuperscript{2} in the steady-state. Later in the project, 
further packages of work will be designed to
address other aspects around the ``in-silico'' design of fusion technology. It 
is recognised that the project poses a
significant human resource and project management challenge, notably to develop 
and manage a team across many different
sites and from many different backgrounds in order to build state-of-the-art 
software that can reliably and accurately
account for a plethora of different complex physical phenomena. Further, the 
project must maintain an international
context and connect to the European EUROfusion programme (in which the UK is a 
key player) and the US Exascale
programme (ECP). Crucially, the emerging software environment must be trusted 
and ``actionable'' to guide procurements
potentially involving hundreds of millions of pounds (e.g. in the case of the 
{DEMO
}\href{https://www.euro-fusion.org/programme/demo/}{\textstyleInternetlink{{[5]}
}}{
first wall}), and ultimately to ensure the safe operation of a multi-billion 
pound nuclear plant.


\bigskip

The UK is world-leading in the study of the edge region of tokamak plasmas. The 
flagship EPSRC/EUROfusion MAST-Upgrade
{experiment
}\href{https://reuters.screenocean.com/record/1366443}{\textstyleInternetlink{{[
6]}}}{
recently }commissioned by UKAEA at Culham has been built with the primary goal 
of testing a novel ``Super-X divertor''
design for handling the plasma exhaust. Combined with software developed under 
ExCALIBUR, the result will be a
significant step forward in our understanding of how heat and particle flows 
can be controlled and kept to within
material limits inside a reactor.


\bigskip

\textbf{\textcolor[rgb]{0.12156863,0.28627452,0.49019608}{Project \nep \ }}

\textbf{(}\textbf{\textcolor[rgb]{0.12156863,0.28627452,0.49019608}{NE}}\textbf{
utrals \&
}\textbf{\textcolor[rgb]{0.12156863,0.28627452,0.49019608}{P}}\textbf{lasma
}\textbf{\textcolor[rgb]{0.12156863,0.28627452,0.49019608}{TU}}\textbf{rbulence
}\textbf{\textcolor[rgb]{0.12156863,0.28627452,0.49019608}{N}}\textbf{umerics 
for the
}\textbf{\textcolor[rgb]{0.12156863,0.28627452,0.49019608}{E}}\textbf{xascale) 
- background}


\bigskip

The UK plasma community -- in UKAEA and several universities - currently makes 
extensive use of ``fluid'' codes in its
research into the edge-plasma region of fusion devices. ``Fluid'' implies that 
the plasma can be treated in one sense
like ``the atmosphere'' in the Met Office's Dynamical Core, but with the added 
complication of significant effects due
to the electrically charged nature of the plasma. For example, plasma electrons 
and ions can to an extent be treated as
separate fluids with different temperatures, interacting via the 
electromagnetic field. Plasma fluid codes (such as
{BOUT++
}\href{https://bout-dev.readthedocs.io/en/latest/user_docs/introduction.html}{\t
extstyleInternetlink{{[7]}}}{)
are relatively }efficient up to only \~1000 cores, but there are a large 
range of effects in the tokamak plasma edge
that prevent the plasma from isotropising to a Maxwellian distribution with 
same temperature ion and electron
populations (which would correspond most closely to the atmospheric fluid 
concept). Firstly, the strong imposed and
directed magnetic field leads to anisotropy since charged particles move 
rapidly in tight, spiral orbits (gyro-orbits)
along the closed field lines and only slowly normal to the field. Secondly 
there may be an inadequate number of
collisions (especially in the burning plasma or ``reactor'' regime) for the 
species to equilibrate, leaving significant
tails in the particle velocity distributions. Such tails may be driven by 
external heating and/or by collisions with
relatively rarefied neutral particle species that are themselves 
non-Maxwellian. Inclusion of these effects in the
fluid models introduces significant modelling uncertainty (which would have to 
be quantified if the models are to be
``actionable'' -- this would require a significant investment). Moreover, 
existing simulations show that attempting to
include the required extra physics can significantly add to execution cost 
overhead or a breakdown of scaling, e.g.
when stability/accuracy issues are addressed, a large reduction in allowable 
timestep can ensue. Unfortunately, there
are significant challenges with moving to a straightforward particle-based 
{model (e.g. PIC
}\href{https://en.wikipedia.org/wiki/Particle-in-cell}{\textstyleInternetlink{{[
8]}}}{,
which }could in principle address these problems) - the strength of the 
electric field in the plasma edge exaggerates
the effects of noise when sampling charged particles. Together, all these 
issues conspire to make it impossible to
achieve converged solutions using existing codes within an acceptable %
%The logic was fluid, then gyrokinetics, then 6-D kinetic, so XGC description 
has been moved
%Wayne Arter
%November 30, 2019 3:04 PM
timeframe.

\ (e.g. particle-based simulations that run on SUMMIT, e.g. using the XGC PIC 
code, can take many weeks for just a
single simulation). Further, the simulations themselves are of course almost 
prohibitively expensive.


\bigskip

The next most widely studied level of treatment of velocity space, {namely 
``gyrokinetics''
}\href{https://en.wikipedia.org/wiki/Gyrokinetics}{\textstyleInternetlink{{[9]}}
}{,
averages over gyro-orbits so that velocity space is treated as a 2D problem 
(perpendicular and parallel to the local
magnetic field). This leads to 5D gyrokinetic models, which in principle }are 
detailed enough and accurate enough to
model reality. Ideas around the discretisation of the velocity (phase) space 
dependence include techniques suitable for
dealing with infinite coordinates, such as moment-based methods using truncated 
series and/or mapped finite elements
perhaps combined with spherical harmonics. Alternative, p(e.g. particle-based 
simulations that run on SUMMIT, e.g.
using the XGC PIC code, can take many weeks for just a single simulation.). 
Further, the simulations themselves are of
course almost prohibitively expensive.


\bigskip

Unfortunately, Eexisting 5D gyrokinetic models are currently at or beyond the 
limit of current HPC capability in terms
of scalability.


\bigskip

Kinetic levels of complexity are nonetheless going to be necessary (at least 
locally) for modelling the burning plasma
regime, due to the inherent uncertainty in the fluid codes. The plasma in a 
fusion reactor may well behave
significantly differently to plasma in existing devices because it will in 
general contain two main ionic species
(Deuterium and Tritium), neutral fuel particles and ionised Helium ash (or 
alpha particles), as well as impurity ions
originating from the wall. Further, the plasma will be hotter, reducing 
collisions so that yet more complicated terms
need to be added to the fluid approximations. These additional contributions to 
the fluid models will require tuning in
much the same way as meteorological micro-physical effects, but in advance 
preferably of suitable experimental data
from ITER or other reactors. Kinetic code results will inevitably be needed for 
this ``tuning'' exercise (i.e.
providing kinetic closures to the fluid codes). Each plasma species will 
contribute different levels of uncertainty,
and a different scaling and performance overhead to the workflow made up of 
close coupled codes that will be developed
under FM-WP2 and FM-WP3 (see below and Table 1). Identifying, quantifying and 
mitigating uncertainty and the
computational overhead of adequately modelling each will be a core theme within 
the project.


\bigskip

Evidently the interpretation of data from ITER and consequently the design of a 
DEMO fusion power plant will require a
hierarchy of models, from those that can be deployed upon Exascale hardware 
down to the surrogate models that will be
deployed at scale upon the high throughput computing platforms of the ITER era 
(e.g. for uncertainty quantification and
parametric optimisation). High fidelity, exascale ``hero run'' codes capable of 
modelling all the required physics to
accurately model the edge region of the tokamak plasma will be used to develop 
order reduced ``surrogate'' models that
can be scaled out for testing against large volumes of experimental data and 
for routine uncertainty quantification as
part of the process shown in Figure 2 (left hand branch). Currently, this 
iterative discovery loop is not possible for
the edge plasma problem due to the unacceptable run-time overhead of existing 
codes and lack of flexibility for
introducing new physics. Similarly, the high fidelity and surrogate models of 
the future will be crucial for designing
future fusion plant ``in-silico'' (this is shown as the right-hand branch of 
Figure 2 -- an iterative ``optimization''
loop). 


\bigskip

{\centering  
\includegraphics[width=11.737cm,height=9.363cm]{CD\exc \ FMS0001110WA-img004.p
ng} \par}

\bigskip

\textit{Figure 2: Typical ``Modelling \& Simulation'' workflow for fusion 
applications. The left-hand iterative loop is
largely around model validation against experimental data whereas the 
right-hand loop focuses upon ``engineering
design''.}


\bigskip

The existing code base is currently limited in its ability (for the same 
reasons as the left-hand branch) to make
accurate predictions that will lead to robust engineering solutions (leading to 
the introduction of unnecessary and
expensive engineering overhead). The current paradigm is to make predictions, 
then build expensive prototypes at scale
to confirm that the code predictions were valid -- this will be impossible for 
STEP and DEMO for cost reasons and due
to the fact that we cannot recreate the operating conditions under which 
components will have to survive -- instead,
quantified uncertainty and ``trust'' must be associated with modelling and 
simulation through rigorous, modern VVUQ --
i.e. our codes must be ``actionable''. Lastly, there are also likely to be 
additional demands upon code execution speed
from DEMO operation, for example there will inevitably be a safety-related need 
for surrogate models that can be
deployed as part of future real-time systems to help control the temperature of 
the first wall. These future needs will
all be embedded within the heart of the project requirements throughout the 
development lifecycle of the \nep \ 
software stack. 


\bigskip

The project itself has been designed in stages, so that early work will make 
significant contributions to existing
software capability and to the development of standalone proxy-apps{
}\href{https://proxyapps.exascaleproject.org/app/}{\textstyleInternetlink{{[10]}
}}{
that capture the }scaling characteristics of the full models being targeted in 
the long term. The overall aim, which
will guide the direction of the project and choice of sub-tasks (most of which 
will be defrayed across UKRI and the
Universities), is to build a hierarchy of models that are capable of 
representing edge plasma behavior to within a
specific level of uncertainty, with the option of at least partial 
incorporation into the development of software
across the EUROfusion work programme, notably the TSVV (Theory, Simulation, 
Verification and Validation) activities.
There will be no attempt to refactor or improve the performance of existing 
legacy codes (unless the overhead of doing
so is low). Instead, emphasis will be upon building new insight/knowledge and 
new software and infrastructure from the
ground up using modern software engineering design principles and 
futureproofing those tools to create an agile
platform that can respond to the rapidly changing vision of what the exascale 
will look like, and to the emergence of
new physics understanding. All codes that are capable of making predictions 
will have embedded within them the concepts
of VVUQ, to ensure that they are ``actionable'' to within specified levels of 
uncertainty (e.g. using ``intrusive'' UQ
methods). Existing legacy codes will of course be an invaluable resource for 
guiding development towards the flexible,
scalable and performant codes of the future.


\bigskip

High level aims of the ExCALIBUR Fusion use case, aligned with the Met Office 
Weather \& Climate prediction system use
case include:


\bigskip

\liststyleWWNumvii
\begin{itemize}
\item To apply the principles of ExCALIBUR to deliver the benefits outlined 
above.
\item To develop and deliver cross-cutting research that aligns with the UKRI 
Research Council contribution (i.e. to
help deliver a UK interdisciplinary team that can address aspects of Exascale 
software design that lie in common across
the represented use cases).
\item To help train the software engineers, architecture specialists, computer 
scientists/algorithms specialists and
data scientists of the ITER and Exascale era.
\end{itemize}

\bigskip

UKAEA will work with the SPF ExCALIBUR Programme Board and Steering Committee 
to ensure alignment and a close working
relationship with the UKRI funded research activities, which for example could 
include participation in workshops and
knowledge exchange activities, participation in stakeholder engagement 
exercises etc.


\bigskip

As part of this Science Plan, four work packages will initially be commissioned 
by the Met Office, i.e. FM-WP1/2/3/4. As
outlined in the Met Office Science Plan [1], the Met Office will manage the 
cross-cutting theme work packages XC-WP1/2
(see below). The four initial Fusion Modelling System work packages are:


\bigskip

\liststyleWWNumvi
\begin{enumerate}
\item Numerical representation;
\item Plasma multiphysics model;
\item Neutral gas \& impurity model;
\item Code structure and coordination,
\end{enumerate}

\bigskip

as shown in table 1 (and in [1]). Code Coupling will be an ``Activity'' with 
elements that will thread FM-WP1
(performance) and FM-WP4 (flexibility) as well as through the two referent 
model work packages FM-WP2 and FM-WP3. All
work packages will be built around modern best practice in co-design and there 
will be significant interaction between
the work packages. Each work package will {be supported by Research Software 
Engineers (RSEs)
(including engagement with the Society of RSE
}\href{https://society-rse.org/}{\textstyleInternetlink{{[11]}}} where 
necessary).


\bigskip


\bigskip

\begin{center}
\tablefirsthead
\tablehead
\tabletail
\tablelasttail
\begin{supertabular}{|m{3.151cm}m{2.603cm}m{3.261cm}m{2.8739998cm}m{2.779cm}|}
\hline
\multicolumn{5}{|m{15.467999cm}|}{{\selectlanguage{english} \textbf{PSRE Use 
Cases}}}\\\hline
\multicolumn{1}{|m{3.151cm}|}{~

{\selectlanguage{english} Weather \& Climate prediction system:}} &
\multicolumn{2}{m{6.064cm}|}{~

{\centering\selectlanguage{english} \textbf{WC-WP1:}\par}

\centering{\selectlanguage{english} Component model co-design}} &
\multicolumn{1}{m{2.8739998cm}|}{~

{\centering\selectlanguage{english} \textbf{WC-WP2:}\par}

\centering{\selectlanguage{english} System co-design}} &
~

{\centering\selectlanguage{english} \textbf{WC-WP3:}\par}

{\centering\selectlanguage{english} System integration\par}

~
\\\hline
\multicolumn{1}{|m{3.151cm}|}{~

{\selectlanguage{english} Fusion Modelling system:}} &
\multicolumn{1}{m{2.603cm}|}{~

{\centering\selectlanguage{english} \textbf{FM-WP1:}\par}

{\centering\selectlanguage{english} Numerical representation\par}

~
} &
\multicolumn{1}{m{3.261cm}|}{~

{\centering\selectlanguage{english} \textbf{FM-WP2:}\par}

\centering{\selectlanguage{english} Plasma multiphysics model}} &
\multicolumn{1}{m{2.8739998cm}|}{~

{\centering\selectlanguage{english} \textbf{FM-WP3:}\par}

\centering{\selectlanguage{english} Neutral gas \& Impurity model}} &
~

{\centering\selectlanguage{english} \textbf{FM-WP4:}\par}

\centering\arraybslash{\selectlanguage{english} Code structure \& coordination}\\\hline
\multicolumn{5}{|m{15.467999cm}|}{{\selectlanguage{english} 
\textbf{Cross-Cutting Themes}}}\\\hline
\multicolumn{2}{|m{5.9540005cm}|}{~

{\centering\selectlanguage{english} \textbf{XC-WP1:}\par}

\centering{\selectlanguage{english} Common approaches \& solutions}} &
\multicolumn{3}{m{9.314cm}|}{~

{\centering\selectlanguage{english} \textbf{XC-WP2:}\par}

{\centering\selectlanguage{english} Emerging technologies\par}

~
}\\\hline
\end{supertabular}
\end{center}

\bigskip

{\centering
\textit{Table 1: SPF ExCALIBUR Met Office commissioned Work Packages.}
\par}


\bigskip

\textbf{FM-WP1} will focus upon the suitability of available numerical 
algorithms (or the development of new algorithms)
for Exascale targeted plasma modelling. As stated above, elements of this work 
package together with FM-WP4 will also
surround the ``coupling technology'' that will be required to connect the 
edge/pedestal region of the plasma (addressed
by FM-WP2) and the neutral gas/impurity model (FM-WP3). An agile approach will 
be adopted around the legacy codes that
might be necessary for helping to develop a roadmap for the FM-WP1 referent 
models, e.g. through community workshops
(the first being in Feb 2020).


\bigskip

\textbf{FM-WP2} and \textbf{FM-WP3} will concentrate upon development of the 
two close coupled models of the \nep \ 
programme, specifically FM-WP2 around the inclusion of kinetic effects into 
existing and new edge plasma models, and
FM-WP3 of particle based models for describing the region outside and just 
inside the plasma (neutral atoms/molecules
and partially ionized impurities). Initial exploratory work will be carried out 
using existing codes (as per above) and
via the development of proxy-apps, for example to expose options for exascale 
targeted hardware (GPUs, ARM technology
etc.).


\bigskip

\textbf{FM-WP4} will focus upon the design of data structures to interface 
between the different models and generally
ensure best practice in scientific software engineering (being responsible for 
Quality Assurance, co-design,
integration etc.). Aligned with Met Office work {package WC-WP1, this programme 
of work will also
explore the use of a ``separation of concerns'' methodology for the Fusion use 
case critical components (e.g. by
exploring the use of Kokkos
}\href{https://cfwebprod.sandia.gov/cfdocs/CompResearch/docs/Kokkos-Multi-CoE.pd
f}{\textstyleInternetlink{{[12]}}}{
and Domain Specific Language (DSL)
}\href{https://en.wikipedia.org/wiki/Domain-specific_language}{\textstyleInterne
tlink{{[13]}}}{ technologies}).


\bigskip

Scoping work for the Met Office led cross-cutting themes (XC-WP1 and 2) will be 
defined in collaboration with the other
ExCALIBUR partners early in year 2 with a view towards starting work later that 
year. The specifics of the
cross-cutting themes are here assumed to be covered by the two overarching 
themes of ``Common Approaches and
Solutions'' and ``Emerging Technologies'' (discussed further below).


\bigskip


\bigskip

\section[UKAEA Research 
Plans]{\textcolor[rgb]{0.12156863,0.28627452,0.49019608}{UKAEA Research Plans}}

\bigskip

\textbf{\textcolor[rgb]{0.12156863,0.28627452,0.49019608}{Numerical 
representation
(}}\textcolor[rgb]{0.12156863,0.28627452,0.49019608}{Fusion modelling, work 
package
}\textbf{\textcolor[rgb]{0.12156863,0.28627452,0.49019608}{FM-WP1}}\textcolor[rg
b]{0.12156863,0.28627452,0.49019608}{)}


\bigskip

The ideal numerical algorithms for forming the Exascale edge plasma codes of 
the future will have (but will not be
restricted to) the following properties:


\bigskip


\bigskip

\liststyleWWNumviii
\begin{enumerate}
\item %
%Added prefix P to indicate desirable property
%Wayne Arter
%November 30, 2019 3:24 PM
Accurate solution of hyperbolic problems.
\item Ability to deliver efficient and accurate solutions of corresponding 
elliptic problems.
\item Accurate modelling of highly anisotropic dynamics. 
\item Accurate representation of first wall geometry (face normals to within 
0.1$^\circ$\textsubscript{--}), and
correspondingly of complex magnetic field geometries.
\item Accurate representation of velocity (phase) space.
\item Preservation of conservation properties of the underlying equations.
\item Scalability to likely Exascale architectures:
\end{enumerate}
\liststyleWWNumxiii
\begin{enumerate}
\item interaction between models of different dimensionality,
\item interaction between particle and fluid models,
\item dynamic construction of surrogates.
\end{enumerate}
\liststyleWWNumviii
\begin{enumerate}
\item Performance portability to allow rapid deployment upon emerging hardware.
\end{enumerate}

\bigskip

It is difficult at this state to rank the importance of these properties, or to 
identify the cost associated with
achieving each in a timely fashion -- a clearer understanding of immediate 
needs and achievable SMART deliverables will
emerge as the project matures via extensive community engagement and team 
building across the UK partners -- this
exercise will start in Feb 2020 via an open workshop. 


\bigskip

It is clear however that the choice of geometrical representation and numerical 
scheme will have a profound impact upon
almost all areas of the project. Options will be identified by means of 
literature and code surveys and consultation
with UKRI and industry experts. Research will be commissioned where necessary 
to eliminate unsuitable choices as early
as possible. Relatively small development tasks will initially be undertaken to 
test remaining candidate methods for
accuracy, stability and HPC scalability potential. \ This task, along with 
FM-WP4 will have a prioritised start to
provide initial inputs into FM-WP2 and FM-WP3 developments and work to be 
defrayed in year 2. Tasks FM-WP1-3 will
incorporate numerical, finite element and other plasma physics libraries of 
suitable quality and exascale
applicability.


\bigskip

Options, which do not preclude consideration of others, have been tentatively 
identified for initial investigation as
follows:


\bigskip

\liststyleWWNumix
\begin{enumerate}
\item Spectral/hp {element
}\href{https://en.wikipedia.org/wiki/Spectral_element_method}{\textstyleInternet
link{{[14]}}}{,
combined }with Discontinuous {Galerkin
}\href{https://www.sciencedirect.com/topics/engineering/discontinuous-galerkin}{
\textstyleInternetlink{{[15]}}}{,
to }meet P1,P3 and possibly P5 above.
\item Multigrid methods for P2.
\item Nekmesh for P4.
\item For Exascale (P7):

\begin{enumerate}
\item matrix-based approaches, hierarchical geometric structures,
\item kinetic enslavement, multi-index Monte-Carlo methods,
\item physics based Neural Network approaches.
\end{enumerate}
\item {MUSCLE 2 etc. as referenced in
}\href{https://royalsocietypublishing.org/doi/10.1098/rsta.2018.0144}{\textstyle
Internetlink{{[16]}}}{,
MUI
}\href{https://www.sciencedirect.com/science/article/pii/S0021999115003228}{\tex
tstyleInternetlink{{[17]}}}{,
ADIOS
}\href{https://www.olcf.ornl.gov/center-projects/adios/}{\textstyleInternetlink{
{[18]}}}{
etc. for code }coupling P6, P6,7.
\end{enumerate}

\bigskip

As per the algorithm requirements, these will be ranked via the community 
workshop planned for Feb 2020. Activities for
defrayment will then be defined around SMART deliverables together with an 
integrated delivery plan for the entire
project.


\bigskip

\subsection[Plasma multiphysics model (Fusion modelling, work package
FM-WP2)]{\textbf{\textcolor[rgb]{0.12156863,0.28627452,0.49019608}{Plasma 
multiphysics
model}}\textcolor[rgb]{0.12156863,0.28627452,0.49019608}{ 
}\textcolor[rgb]{0.12156863,0.28627452,0.49019608}{(Fusion
modelling, work package
}\textbf{\textcolor[rgb]{0.12156863,0.28627452,0.49019608}{FM-WP2}}\textcolor[rg
b]{0.12156863,0.28627452,0.49019608}{)}}

\bigskip

This work package will begin by identifying a referent in conjunction with 
potential users, whereby ``referent'' is
meant a model that establishes the maximum detail and complexity of plasma that 
the software could ever be reasonably
expected to model (beyond exascale). \ Critical features of the referent will 
be identified and prioritised for
implementation as part of the project. This may involve replacing a kinetic 
model by moment-based or fluid models.
Input from the European Boundary Code (EBC) development (funded by EUROfusion 
wherein UKAEA is a core partner) will be
important to this process. In addition, the speed of model execution will be a 
consideration as indicated earlier. A
provisional sequence of developments is as follows (and will be tuned as part 
of the initial \nep \  requirements
capture exercise funded in year 1):


\bigskip

\liststyleWWNumx
\begin{enumerate}
\item 2D model of anisotropic heat transport.
\item 2D elliptic solver in complex geometry.
\item 1D fluid solver with simplified physics but with UQ and realistic 
boundary conditions.
\item Spatially 1D plasma model incorporating velocity space effects.
\item Spatially 1D multispecies plasma model.
\item Spatially 2D plasma model incorporating velocity space effects.
\item Interaction between models of different dimensionality.
\item Spatially 3D plasma kinetic models.
\end{enumerate}

\bigskip

Users from the wider fusion community will be engaged via a workshop in Feb 
2020 to firm up on the plan, to help define
work for defrayment that they can themselves can carry out and will be engaged 
throughout the project to help define
and add new surrogate models through a wide engagement programme and co-design 
(e.g. to treat boundary sheaths) and/or
other important physical effects (e.g. radiation, charge exchange 
recombination, etc.) and compare with existing codes
and experiments. 

\subsection[]{\rmfamily\bfseries }
\subsection[Neutral gas \& Impurity model (Fusion modelling, work package
FM-WP3)]{\textbf{\textcolor[rgb]{0.12156863,0.28627452,0.49019608}{Neutral 
gas \& Impurity
model}}\textcolor[rgb]{0.12156863,0.28627452,0.49019608}{ (Fusion modelling, 
work package
}\textbf{\textcolor[rgb]{0.12156863,0.28627452,0.49019608}{FM-WP3}}\textcolor[rg
b]{0.12156863,0.28627452,0.49019608}{)}}

\bigskip

This work package will begin by identifying a referent (as above) in 
conjunction with potential users and experts in
atomic physics. Critical features of the referent will be identified and 
prioritised for implementation as part of the
project. This may involve deploying a particle-based method, a moment-based 
model or a fluid model (or even a
combination thereof). Input from the European EBC programme will be important 
to this process, as will the speed of
model execution as indicated earlier. A careful assessment of existing codes 
currently in use {will be
necessary, as will cross validation with the established models (notably 
B2-EIRENE
}\href{https://www.tandfonline.com/doi/abs/10.13182/FST47-172}{\textstyleInterne
tlink{{[19]}}}{).
A provisional }sequence of developments is as follows:


\bigskip

\liststyleWWNumxi
\begin{enumerate}
\item 2D particle-based model of neutral gas \& impurities with critical 
physics.
\item 2D moment-based model of neutral gas \& impurities.
\item Interaction with 2D plasma model when available.
\item 3D model of neutral gas \& impurities.
\item Interaction with 3D plasma model.
\item Staged introduction of additional neutral gas/impurity physics.
\end{enumerate}

\bigskip

As per FM-WP2, users from the wider fusion modelling community will first be 
engaged through a workshop in Feb 2020, and
will help build a UK team that will add new physics and compare with existing 
codes and experiments, providing detailed
and rigorous verification and validation (V\&V). This is likely to require the 
provision of databases for different
ionisation and excitation reactions, both in the plasma volume and at surfaces.

\subsection[]{\rmfamily\bfseries }
\subsection[Code structure and coordination (Fusion modelling, work package
FM-WP4)]{\textbf{\textcolor[rgb]{0.12156863,0.28627452,0.49019608}{Code 
structure and coordination
}}\textcolor[rgb]{0.12156863,0.28627452,0.49019608}{(Fusion modelling, work 
package
}\textbf{\textcolor[rgb]{0.12156863,0.28627452,0.49019608}{FM-WP4}}\textcolor[rg
b]{0.12156863,0.28627452,0.49019608}{)}}

\bigskip

The most important aim of this work package will be to drive user engagement 
and ensure that the software is fit for its
defined purpose, first by requirements capture, then by defining suitably 
flexible code structures and related
e-Infrastructure for users, ultimately supporting uptake of the new code(s). In 
order to achieve this aim, this work
package will coordinate across the other tasks FM-WP1-3, to ensure that 
outcomes are compatible and of sufficiently
high quality. There will be management and coordination tasks that will grow as 
the project matures,
{connecting with the EUROfusion E-TASC (TSVV) programme, the EPSRC T.P. 
Turbulence Programme
}\href{https://www.york.ac.uk/physics/news/departmentalnews/plasma-fusion/major-
grant-award-supports-fusion-energy-research/}{\textstyleInternetlink{{[20]}}}{
and the }US ECP programme etc.


\bigskip

Coordination tasks will include (but are not restricted to):


\bigskip

\liststyleWWNumxii
\begin{enumerate}
\item Allocation of resource between tasks and setting project priorities.
\item Ensuring a consistent choice of definitions (ontology) of objects or 
equivalently classes.
\item Definition of common interfaces to components for data input and output. 
\item Design of suitably flexible data structures for common use by all 
developers.
\item Establishment, promotion and support of good scientific software 
engineering practice.
\item Evaluation and deployment of performance portability tools and DSLs 
targeting Exascale-relevant architectures.
\item Integration of the developed software into a VVUQ framework (exploiting 
common approaches developed under XC-WP1
and XC-WP2).
\item Coordination of a benchmarking framework for correctness testing and 
performance evaluation of the developed
software stack.
\end{enumerate}

\bigskip

Along with FM-WP1, this work package will be prioritised for an early start, as 
good scientific software engineering
practice needs to be agreed quickly, and well documented interfaces to 
components need to be available early to ensure
that best practice design is embedded from the start.


\bigskip

\textbf{\textcolor[rgb]{0.12156863,0.28627452,0.49019608}{Common approaches and 
solutions
}}\textcolor[rgb]{0.12156863,0.28627452,0.49019608}{(Cross-cutting themes,
}\textbf{\textcolor[rgb]{0.12156863,0.28627452,0.49019608}{XC-WP1}}\textcolor[rg
b]{0.12156863,0.28627452,0.49019608}{)}


\bigskip

{The examples listed in [1] are very well aligned with the needs of the Fusion 
use case comprising
initially \nep \ , especially the methodologies surrounding a ``separation of 
concerns'', coupling technologies, the
use of mixed precision arithmetic and fault tolerance. In addition, project 
\nep \  will benefit from an exploration of
the convergence of HPC and AI (e.g. the use of Neural Network PDE solvers
}\href{https://arxiv.org/abs/1906.01200}{\textstyleInternetlink{{[21]}}}\textsty
leInternetlink{{
or for advanced preconditioners}}{) and parallel in time methods
}\href{https://parallel-in-time.org/}{\textstyleInternetlink{{[22]}}}{. These 
areas
of common ground and opportunities for interdisciplinary working therein shall 
be explored and agreed by partners as
outlined in the Met Office Science Case.}


\bigskip

\textbf{\textcolor[rgb]{0.12156863,0.28627452,0.49019608}{Emerging
Technologies}}\textcolor[rgb]{0.12156863,0.28627452,0.49019608}{ (Cross-cutting 
themes,
}\textbf{\textcolor[rgb]{0.12156863,0.28627452,0.49019608}{XC-WP2}}\textcolor[rg
b]{0.12156863,0.28627452,0.49019608}{)}


\bigskip

{The examples listed for XC-WP2 in the Met Office Science Case [1] are again 
highly aligned with the
Fusion use case. Other technologies that are deserving of consideration include 
High Bandwidth Memory (HBM
}\href{https://en.wikipedia.org/wiki/High_Bandwidth_Memory}{\textstyleInternetli
nk{{[23]}}}{),
novel exascale targeted IO/storage }{technology (e.g. H2020 project SAGE II
}\href{http://sagestorage.eu/about/overview}{\textstyleInternetlink{{[24]}}}{ 
led by
Seagate wherein UKAEA is a partner), next generation accelerator technology 
(Nvidia Volta Next, A64fx) and ARM systems
for efficient scaling of Flops/Watt. It is not clear at this stage which 
technologies offer the best route forward for
the FM-WP2 and FM-WP3 close coupled system but it is clear that the UK is well 
placed to explore all of them (e.g.
through Isambard II). As stated earlier, a core theme within the project is to 
build solutions around a ``separation of
concerns'' philosophy (using tools such as Kokkos and Open-API) in order to 
develop an agile environment that can adapt
rapidly to disruptive emergent HPC technology.}


\bigskip


\bigskip


\bigskip


\bigskip


\bigskip


\bigskip


\bigskip


\bigskip


\bigskip


\bigskip

\textbf{\textcolor[rgb]{0.12156863,0.28627452,0.49019608}{References}}


\bigskip

{[1]\ \ \ \ }Strategic Priorities Fund: ExCALIBUR Met Office Science Plan and 
Activities

[2]\ \ \ \ \url{https://www.iter.org}

[3]\ \ \ \ 
\url{https://www.gov.uk/government/news/uk-to-take-a-big-step-to-fusion-electric
ity}

\textstyleInternetlink{{[4]}\ \ \ \ https://en.wikipedia.org/wiki/Tokamak}

[5]\ \ \ \ \url{https://www.euro-fusion.org/programme/demo/}

[6]\ \ \ \ \url{https://reuters.screenocean.com/record/1366443}

[7]\ \ \ \ 
\url{https://bout-dev.readthedocs.io/en/latest/user_docs/introduction.html}

[8]\ \ \ \ \url{https://en.wikipedia.org/wiki/Particle-in-cell}

[9]\ \ \ \ \url{https://en.wikipedia.org/wiki/Gyrokinetics}

[10]\ \ \url{https://proxyapps.exascaleproject.org/app/}

[11]\ \ \url{https://society-rse.org}

[12]\ \ 
\url{https://cfwebprod.sandia.gov/cfdocs/CompResearch/docs/Kokkos-Multi-CoE.pdf}

[13]\ \ \url{https://en.wikipedia.org/wiki/Domain-specific_language}

[14]\ \ \url{https://en.wikipedia.org/wiki/Spectral_element_method}

[15]\ \ 
\url{https://www.sciencedirect.com/topics/engineering/discontinuous-galerkin}

[16]\ \ \url{https://royalsocietypublishing.org/doi/10.1098/rsta.2018.0144}

[17]\ \ 
\url{https://www.sciencedirect.com/science/article/pii/S0021999115003228}

[18]\ \ \url{https://www.olcf.ornl.gov/center-projects/adios/}

[19]\ \ \url{https://www.tandfonline.com/doi/abs/10.13182/FST47-172}

[20]\ \ 
\url{https://www.york.ac.uk/physics/news/departmentalnews/plasma-fusion/major-gr
ant-award-supports-fusion-energy-research/}

{[21]\ \ }\url{https://arxiv.org/abs/1906.01200}

{[22]\ \ }\url{https://parallel-in-time.org}

{[23]\ \ }\url{https://en.wikipedia.org/wiki/High_Bandwidth_Memory}

{[24]\ \ }\url{http://sagestorage.eu/about/overview}


\bigskip


\bigskip
\end{document}
