\textbf{FM-WP1} will focus upon the suitability of available numerical 
algorithms (or the development of new algorithms)
for Exascale targeted plasma modelling. As stated above, elements of this work 
package together with FM-WP4 will also
surround the ``coupling technology'' that will be required to connect the 
edge/pedestal region of the plasma (addressed
by FM-WP2) and the neutral gas/impurity model (FM-WP3). An agile approach will 
be adopted around the legacy codes that
might be necessary for helping to develop a roadmap for the FM-WP1 referent 
models, e.g. through community workshops
(the first being in Feb 2020).

The ideal numerical algorithms for forming the Exascale edge plasma codes of 
the future will have (but will not be
restricted to) the following properties:


\bigskip


\bigskip

\liststyleWWNumviii
\begin{enumerate}
\item %
%Added prefix P to indicate desirable property
%Wayne Arter
%November 30, 2019 3:24 PM
Accurate solution of hyperbolic problems.
\item Ability to deliver efficient and accurate solutions of corresponding 
elliptic problems.
\item Accurate modelling of highly anisotropic dynamics. 
\item Accurate representation of first wall geometry (face normals to within 
0.1$^\circ$\textsubscript{--}), and
correspondingly of complex magnetic field geometries.
\item Accurate representation of velocity (phase) space.
\item Preservation of conservation properties of the underlying equations.
\item Scalability to likely Exascale architectures:
\end{enumerate}
\liststyleWWNumxiii
\begin{enumerate}
\item interaction between models of different dimensionality,
\item interaction between particle and fluid models,
\item dynamic construction of surrogates.
\end{enumerate}
\liststyleWWNumviii
\begin{enumerate}
\item Performance portability to allow rapid deployment upon emerging hardware.
\end{enumerate}

\bigskip

It is difficult at this state to rank the importance of these properties, or to 
identify the cost associated with
achieving each in a timely fashion -- a clearer understanding of immediate 
needs and achievable SMART deliverables will
emerge as the project matures via extensive community engagement and team 
building across the UK partners -- this
exercise will start in Feb 2020 via an open workshop. 


\bigskip

It is clear however that the choice of geometrical representation and numerical 
scheme will have a profound impact upon
almost all areas of the project. Options will be identified by means of 
literature and code surveys and consultation
with UKRI and industry experts. Research will be commissioned where necessary 
to eliminate unsuitable choices as early
as possible. Relatively small development tasks will initially be undertaken to 
test remaining candidate methods for
accuracy, stability and HPC scalability potential. \ This task, along with 
FM-WP4 will have a prioritised start to
provide initial inputs into FM-WP2 and FM-WP3 developments and work to be 
defrayed in year 2. Tasks FM-WP1-3 will
incorporate numerical, finite element and other plasma physics libraries of 
suitable quality and exascale
applicability.


\bigskip

Options, which do not preclude consideration of others, have been tentatively 
identified for initial investigation as
follows:


\bigskip

\liststyleWWNumix
\begin{enumerate}
\item Spectral/hp {element
}\href{https://en.wikipedia.org/wiki/Spectral_element_method}{\textstyleInternet
link{{[14]}}}{,
combined }with Discontinuous {Galerkin
}\href{https://www.sciencedirect.com/topics/engineering/discontinuous-galerkin}{
\textstyleInternetlink{{[15]}}}{,
to }meet P1,P3 and possibly P5 above.
\item Multigrid methods for P2.
\item Nekmesh for P4.
\item For Exascale (P7):

\begin{enumerate}
\item matrix-based approaches, hierarchical geometric structures,
\item kinetic enslavement, multi-index Monte-Carlo methods,
\item physics based Neural Network approaches.
\end{enumerate}
\item {MUSCLE 2 etc. as referenced in
}\href{https://royalsocietypublishing.org/doi/10.1098/rsta.2018.0144}{\textstyle
Internetlink{{[16]}}}{,
MUI
}\href{https://www.sciencedirect.com/science/article/pii/S0021999115003228}{\tex
tstyleInternetlink{{[17]}}}{,
ADIOS
}\href{https://www.olcf.ornl.gov/center-projects/adios/}{\textstyleInternetlink{
{[18]}}}{
etc. for code }coupling P6, P6,7.
\end{enumerate}

\bigskip

As per the algorithm requirements, these will be ranked via the community 
workshop planned for Feb 2020. Activities for
defrayment will then be defined around SMART deliverables together with an 
integrated delivery plan for the entire
project.

