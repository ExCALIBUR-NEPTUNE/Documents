\section{{\nep\  Meeting: 20 April 2021 2-3pm BST}}

\emph{Present}

\begin{itemize}
\item Chair: Wayne Arter, UKAEA
\item Felix Parra, Oxford
\item Michael Barnes, Oxford
\item Sarah Newton, UKAEA
\item John Omotani, UKAEA
\item Joseph Parker, UKAEA
\item Ed Threlfall, UKAEA

\end{itemize}

\section{Minutes}

WA started by discussing the recent report~`03' by Oxford, {\it Physics in the edge 
of fusion devices}~\cite{2047357-TN-03}.  (Two other
reports~\cite{2047357-TN-01,2047357-TN-02}  have already been
produced.) He commended the very good report, 
praising an accessible (ie.\ to other \nep\  grant holders) introduction, 
followed by an authoritative literature survey (4 pages of references).  The 
reviewer had passed the report subject to only minor details.  WA thought the 
focus on drift-ordered fluid models interesting and FP confirmed that this was 
the approach taken in most of the current state-of-the-art models.  WA said the 
fundamental dimension is the size of the flow ie.\ the size of the boundary 
sheath, but mentioned there could be an alternative expansion for higher flows 
- FP agreed and cited the case where there is a potential difference across the 
wall (some machines do this eg.\ rotating mirrors, but not tokamaks; he also 
mentioned biasing the divertor though this idea seems to have been abandoned 
years ago for reasons that were unclear).  WA said potential differences made 
performance worse and that the pedestal does not `like' having flow.

WA moved on to discuss the novel research content of the report; it was not 
clear what model would be best ie.\ full gyrokinetics, or drift kinetics with 
small corrections (easier).  FP added that drift-plus-corrections might be 
acceptable; WA stated that the main worry with this approach is a proper energy 
conservation relation; however, a cheap model is certainly needed, in order to 
perform many runs with the aim of assessing the effect of damping terms.  WA 
mentioned that lots of diverging types of calculations were performed in the 
1990s.  WA suggested formulating the new physics in the form of a variational 
principle, compatible with new work involving adding a variational-based DSL to 
\F{Nektar++} as part of \exc\  cross-cutting theme work (this could allow 
the easy passage of the new physics into an exascale-enabled code in the next 
three years).  FP mentioned that there are many variational approaches, 
automatic differentiation etc. so many possible ways of formulating physics 
models.

WA returned to his own questions regarding the report and asked FP how he 
planned to proceed next.  FP responded that he proposes to explore three areas 
where existing work is lacking:
\begin{enumerate}
\item Address neutrals, with a charge-exchange operator;
\item Wall boundary conditions;
\item Calculation of the electric field (difficult).
\end{enumerate}

The electric field is obtained from a pressure balance equation and there is a 
trick in gyrokinetics for calculating it from a small polarization density 
term.  Material on the pressure balance approach is included in FP's first 
report (the new thing is to try this in the higher-moment approach, meaning the 
electric field can be derived from the parallel momentum equation).  FP is not 
sure it will work as it involves a rather artificial `split' of an equation.  
WA then asked about similarities to the asymptotic-preserving schemes as used 
by Patrick Farrell; FP said this problem is similar - it involves the 
asymptotic behaviour of a small parameter (here, the gyroradius) though it does 
not clearly fall into the strict ambit of asymptotic-preserving problems.  WA 
mentioned that even if implemented using asymptotic-preserving methods, PF's 
work shows that a degree of numerical ingenuity is still required.  It seemed 
that FP had talked with PF after the last progress meeting to revisit a 
problem FP had worked on (without success) in the past - it seems they are 
working on this but there were no details.

WA steered the discussion back to the report; he was a little confused by the 
discussion of particle methods (a kinetic model may be attacked as a 5-D fluid 
system or as a particle system).  FP stated that the continuum equations are 
integrable using particles; WA said the the physics suggests this is so but 
that electromagnetism is then a challenge.  FP said his method is not exclusive 
to fluid or particle approaches; fluid methods are attractive as it is clear 
how to do a spectral implementation, as in MB's work (there was some confusion 
about whether a spectral particle code is even possible).  The \I{HAGIS} code~\cite{Pi98HAGI}
(particles coupled to spectral force fields) was cited - Fourier spectral field 
representation provides a nice cut-off here.  WA said he had brought up 
particles because of sheath effects meaning it is not clear what boundary 
conditions to use for fluids near the wall.  In this context, FP has extended 
work by Geraldini et al~\cite{Ge19Depe}  to derive the distribution of ions at a domain end 
(extension involves replacing adiabatic electrons with kinetic ones); this 
distribution is input into a (complicated) code by Geraldini that 
calculates the potential at the wall, and which electrons come back, given the 
ion distribution at the presheath. WA wanted to understand how this all fitted 
together.  FP said for \exc\ , the full sheath code would be replaced by 
something simpler eg.\ assuming ions always leave and that this assumption 
determines the potential drop, hence a simplified boundary condition.  The 
motivation here was not to start simulations with a full-blown code whose true 
results are unknown.  WA mentioned that PIC modelling could be used to check 
the outputs of this boundary code.  FP agreed and also said the Geraldini 
boundary code might prove unstable.  Regarding PIC coupling, WA asked about how to 
couple the different representations used in different domains, having talked 
previously to a worker on the \I{XGC} boundary code who had admitted that the 
determination of the sizes of overlap regions was done largely by trial and 
error.  WA had previously mentioned methodological concerns 
about instabilities in the US \I{GENE} simulation code.
FP expressed his own confusion why there is currently work to 
couple \I{XGC} and \I{GENE}, given that these are both kinetic codes (\I{XGC} 
is gyrokinetic PIC and \I{GENE} treats a 5-D continuum model.  WA mentioned that NASA need 
to solve the particles / fluids dichotomy; he thought \I{XGC} was best for 
boundaries and \I{GENE} for bulk.  WA agrees that one can use a continuum 
representation but worries that collision operators are problematic; further to 
this, FP asked whether particle methods were better for collisions. WA said 
that collisions have been done for many years and the trade-off is accurate 
microphysics but much noise / sampling error; he mentioned it might not be 
necessary for us to get the momentum balance strictly correct.  FP thinks 
collisions are a problem in PIC too; he sees why the neutronics field uses 
particle codes but thinks either particle or continuum appropriate for plasmas. 
WA mentions collisions in fluids are harder because both particles are moving and generally 
a hard problem to identify a particle `collision' without using a mesh.
FP said particle vs. continuum is not considered in his 
report.  WA said he has never looked into rigorous maths behind this in detail 
and opined that a definitive theory is needed here.  WA then mentioned that in 
classical fluids, the mean free path is used to determine what overlap is 
needed (ie.\ between regions simulated using continuum and particle 
descriptions) - can this be applied in plasmas?  FP opined that he favours 
fluid / drift models for two reasons: 1) the 11-moment equation can be used to 
compute the electric field and 2) one has better control over the energy; in 
different regions of different temperatures, their grids can rescale 
`automatically'.  There is scope for a diagnostic in this code that detects 
that the distribution is Maxwellian and triggers use of Braginskii theory.  WA 
mentioned that in space applications, he thought he recollected that the
mean free path is used to determine 
whether a particle description is needed (it is clearly more attractive to 
simulate a fluid by default and switch unresolved regions to use a more 
expensive particle description as required, rather than defaulting to 
particles); NASA applications involve compression and heating near a nose cone 
giving a hot, collisionless gas (and indeed a plasma).  WA asked whether there 
exists a simple problem using FP's formalism in which the need for a particle 
description can be detected.  FP answered that, in his experience, the need for 
kinetics is determined heuristically (he used the phrase `fudge factor'); he 
noted that even if the mean free path is relatively small, collisionless 
processes involve high energy particles which can nevertheless contribute
significantly to transport eg.\ heat fluxes.

WA invited FP to talk about his plans for future work and FP handed over to MB, 
who has implemented a 1+1-D drift kinetic solver which evolves the distribution 
function, see the '02'~report~\cite{2047357-TN-02}.  The results from this agree with kinetic solvers (once they apply 
`corrections').  Technical details: implemented in Julia using the FFTW3
library; Chebyshev pseudospectral elements; no explicit artificial viscosity 
but uses an upwind flux at element boundaries so it has slight numerical 
dissipation: apparently they deal with this by `re-normalizing' the integral 
over velocity space so it remains constant.  They still want to test solving 
for the electromagnetic field in this code (this would conclude their work with 
periodic boundaries).  WA mentioned that a similar code could be implemented in 
\F{Matlab} but that this would run slowly; MB added that JO had tested the 
Julia code vs. C++ and found little difference in speed.

There followed a brief discussion on DSLs (unfortunately FP and MB missed the 
recent DSL meeting due to other commitments).  WA mentioned that \F{Firedrake} 
is build around Python and it calls C code, giving a possible challenge for 
integrating \F{Nektar++} into the DSL of \F{Firedrake} as part of a forthcoming 
cross-cutting \exc\  bid.  MB added that Julia can call C++ or Fortran code.

WA closed the meeting.  FP checked it was OK for JO to be a co-author on their 
publication; WA said there was no issue here so long as the bidding process is 
seen to be transparent and fair.  He reiterated that the aim of UKAEA is to make codes that come 
out of \nep\  as widely-available as possible.

