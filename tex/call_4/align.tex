\begin{itemize}
\item This task focuses upon the suitability of available numerical 
algorithms (or the development of new algorithms)
for Exascale targeted Model Order Reduction. As stated in the Fusion Modelling System
Science Plan
\begin{itemize}
\item Quality control, verification, validation and uncertainty quantification
(VVUQ, eg.\ via intrusive or ensemble-based methods) will be embedded across all
areas of the project to ensure numerical predictions are ``actionable''. 
\item The overall aim, 
\ldots of the project \ldots is to build a hierarchy of models that are capable of
representing edge plasma behaviour to within a specific level of uncertainty.
\end{itemize}

\item The ideal numerical algorithms for forming the Exascale edge plasma codes of 
the future will have preferably at least the following properties:
\begin{itemize}
\item[P1] Accurate solution of hyperbolic problems.
%\item[P2] Ability to deliver efficient and accurate solutions of corresponding 
%elliptic problems.
\item[P3] Accurate modelling of highly anisotropic dynamics. 
%\item[P4] Accurate representation of first wall geometry (face normals to
%within~$0.1^{0}$), and correspondingly of complex magnetic field geometries.
\item[P5] Accurate representation of velocity (phase) space.
%\item[P6] Preservation of conservation properties of the underlying equations.
\item[P7] Scalability to likely Exascale architectures:
\begin{itemize}
\item[a] interaction between models of different dimensionality,
\item[b] interaction between particle and fluid models,
\item[c] dynamic construction of surrogates.
\end{itemize}
\item[P8] Performance portability to allow rapid deployment upon emerging hardware.
\end{itemize}

\item It is unlikely that any algorithm will be optimal in all categories,
and part of the exercise
will be to rank the importance of these properties.
\end{itemize}

\begin{enumerate}
\item Reduced Order Modelling~(ROM) describes a range of techniques
designed to replace a complex time dependent system usually modelled by PDEs
with a relatively small system of ODEs. Fundamental to many approaches
is the SVD of a matrix generated by autocorrelation of a time series.
ROM is often associated with the use of Gaussian Processes (GP)
to interpolate noisy data, and is beginning to be associated
with the use of Machine Learning techniques to help fit
parameters to the small system.
\item Three basic approaches to ROM are
\begin{itemize}
\item Proper Orthogonal Decomposition~(POD), aka Karhunen-Loeve,
and Principal Components Analysis
that applies SVD to the autocorrelation matrix of the discretely
sampled time series. A variant of this is DEIM which works with
nonlinear functions of the series.
\item Proper Generalised Decomposition~(PGD)
that seeks to replace solutions as low rank separable expansions,
ie. as sums of products of functions
of a single variables, often achieved by repeated application of SVD pairwise
to the dependent variables. (This is representative of a class of methods
known as tensor-train for which the mathematical basis is Kolmogorov's superposition
theorem, cf.\ the work of Trefethen and collaborators~\cite{To15Cont}.)
\item Reduced Basis Methods~(RB) that use sample complete solutions
to represent only selected aspects of the solution efficiently,
eg.\ the average temperature reached by a heated surface.
\end{itemize}
\end{enumerate}

Suggested sources
\begin{enumerate}
\item R.C. Smith, ``Uncertainty quantification: theory, implementation, and applications" \cite{smithUQ}.
\item S.L. Brunton and J.N. Kutz, ``Data-driven science and engineering", \cite[\S\,IV]{bruntonkutz}.
\item Chinesta et al, PGD~\cite{chinestaetal}.
\item B. Haasdonk, Reduced Basis Methods tutorial~\cite{Ha17Redu}.
\end{enumerate}

