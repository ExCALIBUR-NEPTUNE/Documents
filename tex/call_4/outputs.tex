\begin{enumerate}

\item Investigation of at least three different approaches to producing a
ROM that show promise of suitability for HPC / Exascale implementation particularly
in the context of a spectral element code. It is expected that at least
two of these will be drawn from those described at the end of Section~1.2.2.

\item Potentially suitable approaches will be evaluated by means of 
literature and code surveys and consultation
with UKRI and industry experts.
Relatively small development tasks may initially be undertaken to 
test candidate methods for
accuracy, stability and HPC scalability potential.
At minimum, evaluation should produce an objective assessment as to
how suitable each ROM is for each of the four roles described in 
Section~1.3. For each too at minimum should be similarly assessed
whether an approach is suitable for:
\begin{itemize}
\item Immediate implementation in \nep\ .
\item Recognition by the \nep\ design so that deferred implementation is facilitated.
\item Implementation only in a restricted class of machine architectures.
\end{itemize}

\item The bidder should indicate how, over the duration of the grant, they intend to engage
in community building and development, for example by assisting in the production
of one or more \papp s to demonstrate key features of preferred approaches.

\end{enumerate}

A commitment is expected to build a fully connected community
across UKRI and Academia, centred on 
a team of around 10-12FTE of UK experts, to meet
the grand challenge goal of developing a
state-of-the-art, Exascale targeted, UK based plasma physics simulation 
capability for the tokamak plasma ``edge'' (see Science Plan~\cite{sciplan}).

