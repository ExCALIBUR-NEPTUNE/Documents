This report has provided a brief overview of each of three important approaches to model order reduction: reduced basis methods, proper orthogonal decomposition and proper generalized decomposition.  
Specific recommendations for methods expected to work well in the magnetically-confined fusion~(MCF) use case are not yet possible, though it is clear that methods capable of handling nonlinearity are indicated, inviting a further study of POD methods and also indicating the need for techniques such as empirical interpolation to preserve the online efficiency of reduced basis methods.  
Another, though perhaps longer-term, consideration is the requirement for an MCF control system to be able to quantify and mitigate extreme events in order to prevent damage to a fusion machine and in this sense, algorithms of the greedy type are useful in that they seek to identify worst-case events rather than focussing on a global least-squares minimization.

Existing MOR implementations encapsulate much of the theory and computational machinery outlined in this report and are able to interface with external PDE solvers.  One simple initial proposal is therefore to apply tools such as {\it pyMOR} to simulations of fluid turbulence (a simple model with a handful of inputs and a single main physically-relevant output in the time-averaged quasi-steady-state heat flux across the domain but a large number of internal degrees of freedom); such models are a proxy for heat transport near the outer boundary of a tokamak.

A subsequent section presented an overview of data assimilation and sketched the two main approaches of Kalman filtering and variational data assimilation.  
Some particular issues in NWP were highlighted, given that similar problems are expected in the case of tokamak edge physics modelling, key shared features being nonlinearity, multiple scales and turbulence; the main interest is how to extend the techniques of DA to work in cases where the errors are non-Gaussian and where the dynamical model is not linear.  
It also revealed that NWP models use what is spiritually a reduced-order model in the inner loop of a perturbation forecast model for estimating model variance in non-Gaussian scenarios.  
The ensemble Kalman filter was highlighted as a technique for fitting not only the model state but also as a tool for Bayesian parameter estimation.
