\begin{itemize}
\item This task focuses upon the suitability of available numerical 
algorithms (or the development of new algorithms)
for Exascale targeted plasma modelling. As stated in the Fusion Modelling System
Science Plan, elements of this work package together with FM-WP4 will also
surround the ``coupling technology'' that will be required to connect the 
edge/pedestal region of the plasma (addressed
by FM-WP2) and the neutral gas/impurity model (FM-WP3).

\item The ideal numerical algorithms for forming the Exascale edge plasma codes of 
the future will have preferably at least the following properties:
\begin{itemize}
\item[{\bf P1}] Accurate solution of hyperbolic problems.
\item[{\bf P2}] Ability to deliver efficient and accurate solutions of corresponding 
elliptic problems.
\item[{\bf P3}] Accurate modelling of highly anisotropic dynamics. 
\item[{\bf P4}] Accurate representation of first wall geometry (face normals to
within~$0.1^{0}$), and correspondingly of complex magnetic field geometries.
\item[{\bf P5}] Accurate representation of velocity (phase) space.
\item[{\bf P6}] Preservation of conservation properties of the underlying equations.
\item[{\bf P7}] Scalability to likely Exascale architectures:
\begin{itemize}
\item[a] interaction between models of different dimensionality,
\item[b] interaction between particle and fluid models,
\item[c] dynamic construction of surrogates.
\end{itemize}
\item[{\bf P8}] Performance portability to allow rapid deployment upon emerging hardware.
\end{itemize}

\item It is unlikely that any algorithm will have all the above, and part of the exercise
will be to rank the importance of these properties.

\item Options were tentatively identified for initial investigation,
but have since been revised as indicated in the Milestone Report
CD/EXCALIBUR-FMS/0013.
\begin{enumerate}
\item The edge region is in parts a very good vacuum, so
that none of the neutral or charged species in the plasma typically thermalizes fully.
Indeed, collision timescales typically vary $\tau\propto T^{3/2}/N$ where $T$~is the temperature of the species
and~$N$ is its number density, so that a species may be treatable as a fluid over say microsecond
timescales in cooler, denser regions, but not on shorter timescales or in parts closer to the core.
The edge may also contain relatively small numbers of impurity species with different charges,
the spectra of which are important for diagnosing local plasma properties.
There is the additional complication of sheath formation at the edge, where the preferential
loss of electrons leads to strong electric fields and consequently flows close to sonic.

\item The cooler parts of the tokamak edge plasma may contain large numbers of neutral atoms,
Neutrals are important because classical plasma transport coefficients~$\kappa$, in
addition to a~$1/\tau$ dependence, are anisotropic because of the strong magnetic field,
to the extent that collisions with neutrals may become an important limiting mechanism
for transport along the field. Another aspect of the neutral species is that
these collisions may represent a large source of plasma.

\end{enumerate}


%\begin{enumerate}
%\item Kinetic or Particle Enslavement (KE),
%\item Quasi-Monte-Carlo (QMC) methods,
%\item Multi-Index Monte-Carlo (MIMC) methods, to include Multi-Level Monte-Carlo (MLMC).
%\end{enumerate}
\end{itemize}
