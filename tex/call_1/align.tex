%The  successful bidder will work closely with other community members 
%(especially UKAEA and the successful bidder to Grant Call~T/NA083/20) to produce
%a roadmap to reach the long-term goals of the project. The bidder 
%should indicate how they intend to help drive co-design of all the constituent 
%elements of \nep \   in support of the Fusion Modelling System use case goals 
%and also the \exc \   overarching pillars.

This task will focus upon the suitability of a) available libraries and 
b) numerical algorithms (or possibly the development of new algorithms and/or 
libraries) for the project with a particular focus upon methods for modelling 
anisotropic heat transport. %in the tokamak plasma edge in close proximity to a 
%complex, 3-D first wall.
Summarising the requirements from the Fusion Modelling System Science Plan, of 
particular note for this task is that:

%\item This workpackage will focus upon the suitability of available numerical 
%algorithms (or the development of new algorithms)
%for Exascale targeted plasma modelling, especially anisotropic
%heat transport. As stated in the Fusion Modelling System
%Science Plan, elements of this work package together with FM-WP4 will also
%surround the ``coupling technology'' that will be required to connect the 
%edge/pedestal region of the plasma (addressed
%by FM-WP2) and the neutral gas/impurity model (FM-WP3).
%An agile approach will be adopted around the legacy codes that
%might be necessary for helping to develop a roadmap for the FM-WP1 referent models.
\begin{itemize}
\item  ``\ldots there are a large range of effects in the 
tokamak plasma edge that prevent the plasma from isotropising to a Maxwellian 
distribution \ldots the strong 
imposed and directed magnetic field leads to anisotropy since charged particles 
move rapidly in tight, spiral orbits (gyro-orbits) along the closed field lines 
and only slowly normal to the field", and


\item the ideal numerical algorithms for forming the Exascale edge plasma codes of 
the future will have preferably at least the following properties:
\begin{itemize}
\item[{\bf P1}] Accurate solution of hyperbolic problems.
\item[{\bf P2}] Ability to deliver efficient and accurate solutions of corresponding 
elliptic problems.
\item[{\bf P3}] Accurate modelling of highly anisotropic dynamics. 
\item[{\bf P4}] Accurate representation of first wall geometry (face normals to
within~$0.1^{0}$), and correspondingly of complex magnetic field geometries.
\item[{\bf P5}] Accurate representation of velocity (phase) space.
\item[{\bf P6}] Preservation of conservation properties of the underlying equations.
\item[{\bf P7}] Scalability to likely Exascale architectures:

\begin{itemize}
\item[a)] interaction between models of different dimensionality,
\item[b)] interaction between particle and fluid models,
\item[c)] dynamic construction of surrogates.
\end{itemize}

\item[{\bf P8}] Performance portability to allow rapid deployment upon emerging hardware.
\end{itemize}

\item It is unlikely that any algorithm will have all of the above; part of the exercise
will therefore be to rank the importance of these properties.

%It is difficult at this state to rank the importance of these properties, or to 
%identify the cost associated with
%achieving each in a timely fashion -- a clearer understanding of immediate 
%needs and achievable SMART deliverables will
%emerge as the project matures via extensive community engagement and team 
%building across the UK partners -- this
%exercise will start in Feb 2020 via an open workshop. 

\item It should also be stressed that 
the choice of geometrical representation and numerical 
scheme will have a profound impact upon
almost all areas of the project.

%Options will be identified by means of 
%literature and code surveys and consultation
%with UKRI and industry experts. Research will be commissioned where necessary 
%to eliminate unsuitable choices as early
%as possible. Relatively small development tasks will initially be undertaken to 
%test remaining candidate methods for
%accuracy, stability and HPC scalability potential. \ This task, along with 
%FM-WP4 will have a prioritised start to
%provide initial inputs into FM-WP2 and FM-WP3 developments and work to be 
%defrayed in year 2. Tasks FM-WP1-3 will
%incorporate numerical, finite element and other plasma physics libraries of 
%suitable quality and exascale
%applicability.

\item Options have been tentatively 
identified for initial investigation as
follows:
\begin{enumerate}
\item Spectral/hp element methods
used with Galerkin and/or Discontinuous Galerkin schemes
to meet {\bf P1},{\bf P3} and possibly {\bf P5} above.
\item Nekmesh to meet {\bf P4} above.
\end{enumerate}

\item A critical ``feature" of the fluid referent model prioritised for implementation is
\begin{enumerate}
\item 2-D model of anisotropic heat transport.
\end{enumerate}
\end{itemize}

Elements of work package FM-WP1 together with work managed though FM-WP4 will 
inevitably surround the ``coupling technology" that will be required to 
connect the edge/pedestal region of the plasma (addressed by work managed under 
FM-WP2) and the neutral gas/impurity model (via FM-WP3) -- see Science Plan~\cite{sciplan}. 
Any libraries or algorithms selected must therefore be suitable for the 
Exascale targeted close coupling of models (potentially including both fluid 
and particle-based approaches) that will be essential to deliver a performant 
and actionable infrastructure in the long term.

