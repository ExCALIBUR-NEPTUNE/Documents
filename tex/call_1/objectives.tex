Simple geometrical models of first wall power deposition exhibit a 
dependence proportional to $\sin{\theta}$ where $\theta$ is the
angle between the surface normal and the direction of the particle
flux. Assuming the particle flux is directed within
magnetic-flux surfaces along the lines of magnetic
field, to maintain peak power deposition within acceptable
limits, $\theta$ must be $2^o$ or less. Therefore,
to be useful, a grid generator must be capable of producing
discrete representations of plasma facing component~(PFC) or
flux surfaces so that their surface normals have
an accuracy much better than $2^o$, say to~$0.1^o$.
Even given an accurately generated mesh, since particle flux
is believed to be strongly anisotropic, it is necessary to
establish to what extent numerical effects lead to
unphysical transport normal to the field direction,
and what methods if necessary are required to ameliorate
these effects.

Edge plasmas present a numerical scheme with many challenges
found in both viscous and nearly inviscid computational fluid
dynamics, together with some which are not, namely
application to large particle sources, sonic outflow to engineered
surfaces, and velocity-phase space effects. It is necessary
that a candidate scheme be shown to be capable of meeting all these
challenges, preferably directly or by interfacing to other schemes that
employ say a particle representation.

The above leads to the additional objective to produce an implementation of the model equations,
geometry and boundary conditions described in  Section~2.1,
entitled `2-D model of anisotropic heat transport' of the 
document ``Equations for \exc/\nep \ \Papp s"~\cite{pappeqs} that will:
\begin{itemize}
\item[$\bullet$] deliver an ``as accurate as possible solution" that will provide an early indication 
as to whether or not the eventual code based upon this method/solver/library combination will 
cater for problem sizes predicted to deliver an ``actionable solution" on Exascale hardware, 
while giving a ``significant" performance improvement over software using second order representations
without mesh adaptation.
 
\item[$\bullet$] For estimating the likely achievable accuracy, performance and tractable problem sizes/resolution
etc.\ on the eventual Exascale platform, the bidder should indicate how they intend to use the \nep \ \papp(s),
scientific literature, information from trustworthy websites,  analytic theory/solutions, and profiling tools that
perform eg.\  roofline analysis, to identify a viable solution and steer the development towards it.
They should outline their vision (and proposed timeline) for achieving the capability requirements
specified in the \exc\ \nep\  Science Plan~\cite{sciplan}.
%\item delivers an accurate and efficient solution, and
%\item captures the scaling characteristics of the full models being targeted
%in the long term 
\end{itemize}
Subject to prior agreement with UKAEA, the bidder may substitute model equations presenting equal or
greater relevant challenges to those described above, for implementation in the \Papp.
