\documentclass[11pt]{article}
\usepackage{graphicx}
\usepackage{verbatim}
\usepackage{multicol}
\usepackage{amsmath}
\usepackage{amssymb}
\usepackage{color}
\usepackage{hyperref}            %insert hyper-links
%\usepackage{sectsty}
%\sectionfont{\small}
%\subsectionfont{\small}
\oddsidemargin=-0.5in
\evensidemargin=-0.5in

\textwidth=7.0in
\setlength{\parindent}{0cm}
\setlength{\parskip}{0.3cm}

% \renewcommand{\floatpagefraction}{0.10}

\topmargin=-2.0cm \textheight=25cm
\newcommand{\nep}{\textsc{NEPTUNE}}
\newcommand{\exc}{\textsc{E}x\textsc{CALIBUR}}
\newcommand{\Papp}{Proxyapp}
\newcommand{\papp}{proxyapp}


\begin{document}

\vspace{0.2cm}

\centerline{\bf \large \textbf{\nep\  DSL workshop 8 April 2021}}

\section{Agenda}

\begin{itemize}
\item Chair / intro (Wayne Arter, UKAEA)
\item {\tt OP2} / {\tt OPS} and \nep\  work (Gihan Mudalige, Warwick)
\item What makes a good DSL? (Will Saunders, UKAEA)
\item TBA (Exeter / Imperial)
\item 10-minute break, followed by general discussion
\end{itemize}
\emph{(The above was subsequently modified by removal of the break and
inclusion of spur-of-the-moment
presentations by most of the external attendees.)}

\section{Minutes}

The meeting was opened by Wayne Arter (chair), who showed the agenda, followed 
by announcements:

\begin{itemize}
\item There is new material on the \nep\  documents GitHub repository; a new 
branch {\tt y2end} contains the presentations / minutes of the March progress 
meeting.  There is still scope for contract holders to check this as the branch 
has yet to be merged.
\item The next RAG progress meeting is scheduled for 10-11am of Thursday 29 
April (see email of 29/3 - which hopefully everybody received).  Volunteers are 
requested for presentations 11am-12pm of the same day.  A Doodle poll for a 
regular fortnightly slot is to follow.
\item This meeting is being recorded for minute-taking purposes.
\end{itemize}

WA continued with a slide indicating the general proposed software strategy: 
spectral / hp methods with UQ, that is made attractive to three broad classes 
of user:

\begin{enumerate}
\item Engineer or physicist using the code as a `black box', e.g. a plasma 
physicist.
\item High-level programmer using e.g. Python or Julia.
\item Writer of new problem-specific code in e.g. C++.
\end{enumerate}

The code needs to be modifiable and extensible to stand the test of 30 years' 
future use.  WA cited matlab and NAG as models and said these might have even 
more grades of potential user.  

WA then displayed slides of a set of equations governing the scrape-off layer 
(SOL) dynamics, after F. Riva (single-fluid), then showed the 13-moment model 
after Zhdanov (fluid model extended to allow for low collisionality); he gave a 
brief explanation of these e.g. replacing pressure stress tensor in a fluid by a 
higher-order dynamical completion.  This was to illustrate the complexity of 
the equation systems describing the SOL plasma.

\subsection{Bridging the complexity gap in exascale simulation software 
development through high-level DSLs, Gihan Mudalige (Warwick)}

This was an overview of many years' work by GM.  A motivation for the work was 
given: the need for parallelizable code capable of working on a diverse 
hardware landscape: a large `zoo' of accelerator types and many different 
programming methods (OpenMP, SIMD, CUDA, ROCm / HIP, MPI, PGAS ...). Open 
standards attempting to keep up with the range of hardware and complicated by a 
lack of overall consensus and companies' vested interests in optimizing code 
for their own hardware; also the issue of large legacy codes (Fortran was 
mentioned).

Raising the level of abstraction to solve the problem of a varied hardware 
landscape: GM explained that a classical compiler has two basic themes: 
analysis (syntax, semantics, ... , polyhedra) and synthesis (parallelization, 
tiling, vectorization) (WA asked for clarification of `polyhedra' in this 
context: it means things like cache-blocking and tiling).  The problem is 
really that for a general program the compiler cannot easily optimize in terms 
of layouts and memory spaces (e.g. if running a computation on an unstructured 
mesh).  The proposed solution is to raise the level of abstraction from that of 
a general programming language to e.g. a communication skeleton ({\tt OP2} / 
{\tt OPS}) and further to a specific numerical method e.g. {\tt Firedrake}.  
With a narrower problem ambit (e.g. just FEM), there is scope to re-use a known 
set of optimizations (or sets, given that there are multiple target platforms). 
 This would work as a domain-specific API (the `contract' with the user) being 
embedded in e.g. Python or C++.  A quote from Mike Giles was cited: {\tt OP2} / 
{\tt OPS} `straitjacket' the user and prevent them from writing bad code.  So 
one has a scheme in which a problem is declared by the user and a set of 
automated routines produce an optimized implementation.  Examples of handling 
unstructured mesh code (which looked to be finite-difference based) were shown; 
the problem here was sorting out data races if multiple edges tried to update 
the same node.

A nice application structure diagram was shown and it was made clear that e.g. 
{\tt OP2} acts as a parser: it is a source-to-source translator and it outputs 
human-readable code, which then goes into a general compiler and can be used 
with general debugging / profiling tools (g++, gdb, Allinea MAP).  The DSL 
layer here handles automatic parallelization, load-balancing, checkpointing, 
and runtime (JIT) compilation.

The talk necessarily accelerated here due to time constraints; some {\tt SYCL} 
examples were shown (a useful citation is a 2021 publication by GM, Jarvis, 
Powell, Owenson, Reguly
\url{https://warwick.ac.uk/fac/sci/dcs/people/gihan_mudalige/op2-mgcfd.pdf}).

Mention was given to the ASIMOV project: re-engineering existing codes with 
{\tt OP2}; e.g. one which is 50k lines of Fortran with over 300 parallel loops.

It was emphasized that getting the correct abstraction (i.e. higher-level 
description) is the main thing and that this has more mileage in it than the 
particular implementation using the technology of the moment (this paraphrases 
a quote from Alfred Aho and Jeffrey Ullman).

WA asked how to minimize the workload of a future re-engineering exercise such 
as in ASIMOV; GM replied that \nep\  is in a good position as it starts code 
from scratch - just get the abstractions correct.

Patrick Farrell asked whether the DSL acts at compile or runtime.  GM replied 
that historically it has been done at compile time due to e.g. not having the 
compiler available on the HPC compute nodes, but some stuff can be done at 
runtime e.g. {\tt Firedrake} does loop nesting and tiling.

\subsection{What makes a good DSL, Will Saunders (UKAEA)}

WS advised he would provide a more high-level discussion, starting with some 
(ubiquitous!) examples of DSLs e.g. \LaTeX, SQL, HTML, APIs.  Desirable 
properties are:

\begin{itemize}
\item Ease of use.
\item Communication: allows accurate problem description and allows enforcing 
of explicit standards from third parties e.g. a publisher's \LaTeX 
conventions.
\item Abstractions to give separation of concerns (e.g. \LaTeX users are 
agnostic as to {\it how} exactly their PDF is generated).
\end{itemize} 

Two types of DSLs were considered: 1) External, implemented by a specific 
interpreter or compiler (e.g. \LaTeX, SQL, {\tt Make}), which are flexible 
but involve the hard work of writing a compiler; 2) Embedded, which extend an 
existing host language (i.e. an API) (e.g. {\tt SYCL}, {\tt UFL} - Unified Form 
Language).  The latter allows the use of the host ecosystem, but restricts the 
DSL to use the lexicon of the host language (e.g. Python does not support 
overloading the assignment operator).

WS presented three characteristics of a `good' DSL:

\begin{itemize}
\item Provides the correct abstraction to describe domain tasks (concurring 
with GM's preceding talk).
\item Ease of use i.e. intuitive for domain user.
\item Composable, as DSLs are rarely used in isolation.
\end{itemize}

WS finished by discussing the question of what we want from a DSL:

\begin{itemize}
\item This is really an open question for all levels of our (prospective) user 
community.
\item Separation of concerns - a hierarchy.
\item Performance portability over likely HPC targets.
\item Offering interoperability between components i.e. acting as a gluing 
language. 
\end{itemize}

\subsection{Discussion of {\tt Nektar++} from a DSL standpoint (my title), 
Spencer Sherwin}

SS explained that {\tt Nektar++} did not initially use DSLs.  He gave then an 
overview of the structure of the code in the latest version (refined in the 
light of knowledge gained writing earlier spectral / hp codes).  Currently the 
developers are pushing a top-level Python interface (also, I'd add that the xml 
session file is a DSL of sorts).

In a discussion between SS and WA it became clear that the fluids community had 
relatively little need for a complex DSL because fluid equations are fairly 
standardized things (in contrast to the wide range of models used in fusion).  
WA emphasized that the new terms added to fusion models (e.g. sources) had the 
potential to cause a wide range of numerical issues, and so it was clear that 
these might need a relatively deep integration with the source code.

\subsection{{\tt UFL} / {\tt Firedrake} (my title), Patrick Farrell}

The question of what happens if the DSL is too restrictive to add a certain 
piece of new physics was raised by SS; PF explained that {\tt Firedrake} 
circumvents this issue by allowing pieces of code from other languages to be 
included (via `escape hatches'), so there can be C++ or Python `bolt-ons'.  PF 
showed then some slides taken from a FEM theory course he teaches showing the 
use of {\tt UFL} in {\tt Firedrake} (one very nice feature is a simple API to 
generate regular meshes - not present in {\tt Nektar++}).  This showed how easy 
it is to specify weak-form PDEs e.g. {\tt G = inner(grad(u), grad(v))*dx - 
inner(f,v)*dx} then {\tt Solve(G==0,u,bc)}.  He then showed a more complex 
linear elasticity example and explained that these toy problems could 
nevertheless be scaled to billions of degrees of freedom of ARCHER.  WA asked 
to see an example of some bolt-on code (PF prepared some - see \ref{discussion} 
below).

\subsection{DSLs in {\tt BOUT++} (my title), Ben Dudson}

BD's experience with coming to existing code and finding discrepancies
between the code and the documentation, as well as fusion physics'
lack of completely-specified models, led to the aim to make it easy to
add new physics to {\tt BOUT++}, and easy to read what equations are
being solved.  He explained the code structure i.e. method-of-lines
time integration with all the physics in a module that computes the
time-derivatives, and explained that there are two DSLs used in
BOUT++:
\begin{itemize}
  \item The physics equations are written in C++ e.g. {\tt ddt(n) = 
    -vE\_Grad(n,phi)+Div\_par(Jpar)+2*DDZ(n) / R\_c}.
  \item An input configuration file format. This evolved from a simple
    configuration file (INI format), and now it can e.g. parse complex
    mathematical formulae, and is Turing-complete. This evolution was
    driven by the need for increasingly complex configurations, in
    particular testing with MMS. In hindsight it might have been
    better to adopt a standardised interpreted input language, rather
    than evolve a unique one.  WA asked whether the next bit of code shown
    was C++ but BD replied it is an external DSL that is interpreted
    at runtime by an interpreter inside BOUT++.
\end{itemize}

%1h 15m

BD explained how the code had been made performant and cited work by
Joseph Parker (2018) vectorizing the kernel inner loops (subsequently,
the bottleneck became the elliptic inversion).  Another problem
overcome was that of too many small loops not parallelizing
efficiently on GPUs (need more work per unit of loaded data to get the
efficiency).  Showed example of merging loops to form a single
outerloop.  This became hard to debug when using techniques such as
C++ templates and code generation, and adding debugging framework code
wrecked the performance. To address this, an idea was borrowed from
{\tt SYCL} - unsafe but lightweight wrappers and doing the runtime
checks outside the loops. This enables efficient code which retains
readability, and can be more easily debugged. There was a brief
discussion about the need for code tuning and manual vectorization, as
compilers cannot always do this well, and interaction with threading
only complicates things.

\subsection{General discussion} \label{discussion}

PF showed {\tt Firedrake} with bolt-on code snippets e.g. example of converting 
non-periodic mesh to periodic in loopy syntax (https://documen.tician.de/loopy/)
and the other as a normal C function  to exploit tensor product structure.
The kernels (either plain C or loopy) are inputs along with sets, maps and
access descriptors to PyOP2. PyOP2 is a large Python code integral to
{\tt Firedrake} which generates the C source code for a shared library
which is written to disk and then compiled with a C compiler of choice (usually GCC).

WA raised a couple of points:

\begin{itemize}
\item From BD talk: it was clear that not all plasma physicists will know about 
FEM.
\item {\tt BOUT++} uses method of lines, and also elliptic solvers: what if we 
needed to solve coupled elliptic problems?  BD explained that such things could 
be transformed into something that can be solved.
\end{itemize}

SS raised issue of whether a DSL could encourage good practice - clearly with a 
DSL the added flexibility means more scope to get nonsense results out (though, 
presumably, VVUQ techniques would flag up bad calculations).  WA agreed that is 
a big question ... SS mentioned that in finite difference, the only adjustable 
parameter is the global refinement (whereas, in finite element, one can locally 
refine the mesh, or do p-refinement; WA mentioned it had to be done dynamically 
as shocks can form during simulations).

SS asked whether the user could be warned if they had introduced a term that 
was likely to cause the simulation to fail; WA responded that the equations for 
fusion tended not to be that bad in this regard (second derivatives, 
collisions and transport) and that the difficulty really came from the sheer 
number of different terms and possible species.  PF asked what a warning (or 
straitjacketing) system would mean in practice given that it might reduce 
overall freedom (e.g. clearly \LaTeX allows the user to write 
mathematically incorrect equations).  SS replied that his intent was to help 
the user understand how simulation results are affected by the various solver 
options.  PF put it that it should be made easy for the user to do things they 
`should' be doing (i.e. use the interface to `nudge' users in the right 
direction).  BD agreed, saying that this was the thinking behind {\tt BOUT++}.  
PF and SS agreed that error checking by means of the method of manufactured 
solutions was useful.  PF then said his concern with {\tt UFL} was that it is 
designed for FEM, not plasmas: he plans to talk to BD about what equations are 
needed for \nep\  and also the wider cross-cutting ExCALIBUR themes.  
Specifically, BD asked whether {\tt UFL} can handle 5D and 6D phase spaces 
(i.e. including velocity space), to which PF replied that {\tt UFL} can handle 
this but the solvers ({\tt Firedrake}) currently cannot.

WA said it seemed many people were happy with {\tt UFL} but many in the plasma 
community are unfamiliar with FEM; he thought some in the community will have 
an equation they wish to solve, so good if they can use the DSL to implement it 
on their own (separation of concerns between the equation and the FEM used to 
solve it).  WA asked whether the \nep\  community should put effort behind 
{\tt UFL}, given that most {\tt UFL} users are not in the fusion field.  BD 
said {\tt UFL} was promising.  PF added that {\tt UFL} was becoming the 
language of choice in the FEM community and is good for comparing FEM runtimes.

BD asked about methods for transforming higher-level mathematics into the weak 
form used in {\tt UFL}.  PF cautioned that there are many possible variational 
forms for the same mathematical equations, different Sobolev spaces etc., so 
automating this is probably impossible.  WA questioned whether something like 
this existed - biased to providing a `robust' solver in all cases - PF unsure, 
but added that a least-square coercive approach would be robust but 
conditioning and performance would always be suboptimal (WA agreed and there 
was an astrophysical application by Wiegelmann which might be admitted
to be at least 100~times slower).  PF opined 
that the approach should be to try to `automate out' the computer programmer 
and not the mathematician or the physicist.

WA raised the issue that the fusion equations may become very large and prone to 
errors / typos in implementation; PF agreed that it might be much quicker to 
implement the equations in {\it Firedrake} that for the mathematician to try to 
debug the equation system by inspection.

WA said one goal was to educate the user community about aspects of FEM, 
therefore expose some of the options (e.g. element order, basis type).  PF 
agreed and said we should try to give enough education that users can avoid 
common pitfalls. SS mentioned some of the options in FEM and added that people 
like the strong form and the nodal basis (as `easier to think about').  PF 
mentioned that not all possible discretizations are stable, citing 
compatibility conditions.  WA mentioned that workers on the European 
Boundary Code project are getting good results with Discontinuous
Galerkin, as this is typically more stable than classical Galerkin - SS 
added that there is a large literature on DG stability, Riemann fluxes etc.

WA wrapped up the session, stating that this meeting was more about discussion 
than reaching firm conclusions (and added that we have yet to define a DSL).  
GM made the point that there is a higher-level maths / physics interface as 
well as a lower-level hardware abstraction layer dealing with loops as its 
input (GM's work concerns the latter of these two).  PF mentioned that {\tt 
Firedrake} takes a {\tt UFL} input and produces computer code output so acts to 
separate concerns.  WA added that there is a great deal of scope for additional 
physics / more species in our equations - is there a need for more software
in consequence?
WA asked whether new functionality can be added to {\tt OP2} if we need (e.g. 
PIC codes); GM said he had yet to try PIC codes and that these need their own 
abstractions; it was mentioned that Steven Wright has worked on PIC via Kokkos, 
so a discussion to be had there.

WA concluded by saying that discussion between grantees is encouraged although UKAEA 
would like to be kept in the loop; also to let him know if the grantees 
thought that UKAEA's organising another session on DSLs would be of benefit.

\section{Attempt at Conclusion (written after the meeting by the minute-taker )}
%\textcolor{red}{remove if desired})}

It became clear that there are really {\it two} separate issues here: the DSL 
as a user interface and the DSL as a hardware abstraction API.

The DSL as a user interface inherits all the usual problems of interface design 
e.g. learning curve, flexibility versus complexity ... The usual holy grail of 
the code being able to determine internally the optimal parameters for 
performing a given calculation (e.g. mesh resolution, level of p-refinement and 
many, many more) seems something to strive toward, rather than representing an 
attainable goal.  The use of `escape hatches' as demonstrated by PF seems a 
very good idea.  {\tt UFL} seems a reasonable early candidate on which to base 
something, though it does not currently do everything we might want (e.g. 
strong form) and it relies on at least some user knowledge of FEM.

The hardware abstraction part seems to be a matter of working out the loop 
structure and implementing this using e.g. {\tt OP2} or considering direct 
implementations in e.g. {\tt SYCL}.  The hope is that these standards will 
render time-consuming performance tuning, e.g. manual vectorization, 
unnecessary.

\end{document}
